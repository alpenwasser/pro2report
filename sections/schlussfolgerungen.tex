% was laeuft, was laeuft nicht, was kann verbessert werden
Das Projekt  wurde erfolgreich abgeschlossen: Das Tool  ist funktionst\"uchtig
und alle Punkte des urspr\"unglichen Auftrages sowie einige der im Verlauf des
Projekts  optionalen  Erweiterungen  wurden erf\"ullt. Aus  den  Eingabewerten
wird  das dynamische  Verhalten des  geschlossenen Regelkreises  berechnet und
graphisch dargestellt. Die Erweiterungen  erm\"oglichen das Dimensionieren des
Reglers anhand der Phasengangmethode und die Optimierung der Reglerwerte.


Die   im   Pflichtenheft   optional  geplanten,   graphischen   Ausgaben   des
Amplitudengangs und der Strecke sowie  die Schieberegler f\"ur das Ver\"andern
der  Reglerwerte wurden  weggelassen. Diese Zus\"atze  wurden geplant,  um das
Dimensionieren  des  Reglers zu  erleichtern  und  die Eingabewerte  graphisch
darzustellen. Der  Grund f\"ur  den Entscheid,  diese zus\"atzlichen  Ausgaben
wegzulassen, war die M\"oglichkeit durch die hohe Rechenleistung den Regler in
Echtzeit zu optimieren, womit die Funktionen \"uberfl\"ussig wurden.


Alternativ kann  die Schrittantwort mit zwei  Schiebereglern manuell optimiert
werden. Durch  das Ver\"andern  des  \"Uberschwingens und  das Einstellen  der
Optimierung kann die Kurve von  Hand angepasst werden.  Als Erweiterung w\"are
es m\"oglich die  Streckenwerte auszugeben und die  Strecke graphisch zur\"uck
zu geben. Das  automatische Einlesen  der Streckenwerte aus  einer graphischen
Form k\"onnte hinzugef\"ugt werden. Das Tool  w\"urde an die eingelesene Kurve
die Wendetangente  legen und die  Werte ablesen. Das Ablesen k\"onnte  mit der
Monte-Carlo-Studie verfeinert werden. Die Monte-Carlo-Studie ist ein Verfahren
aus  der  Wahrscheinlichkeitsrechung, welche  aus  einer  endlichen Schar  von
Kurven die numerisch wahrscheinlichste L\"osung findet.
