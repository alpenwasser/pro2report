% Ausgangslage
Im   Rahmen  des   Projektes  soll   ein  Tool   entwickelt  werden,   welches
einen   PI-  respektive   einen   PID-Regler  mittels   der  von   Prof. Jakob
Zellweger  entwickelten  Phasengangmethode  \todo{referenz  script  Zellweger}
dimensioniert. Zum Vergleich  soll der entsprechende Regler  ebenfalls mittels
verschiedenen Faustformeln berechnet werden.

 % Anforderungen
Die Phasengangmethode ist eine graphische Methode, die bis anhin mit Stift und
Papier  durchgef\"uhrt wurde. Folglich  ist die  Ausf\"uhrung zeitaufw\"andig,
speziell   wenn   Schrittantworten   mit   unterschiedlichen   Parameterwerten
durchgespielt werden sollen. Das  Tool soll ausgehend von  drei Parametern aus
der  Schrittantwort  der  Strecke (Verst\"arkung  $K_s$,  Anstiegszeit  $T_g$,
Verz\"ogerungszeit  $T_u$) mittels  der  Phasengangmethode m\"oglichst  ideale
Regelparameter  berechnen  sowie  die Schrittantwort  des  darauf  basierenden
geschlossenen Regelkreises graphisch  darstellen. Die Benutzeroberfl\"ache der
Software soll  intuitiv sein, sodass sich  auch mit dem Thema  nicht eingehend
vertraute Regelungstechniker einfach zurechtfinden.

 % Umsetzung
Die  erforderlichen  Algorithmen  wurden   zuerst  in  Matlab  als  Prototypen
implementiert  und   anschliessend  vollst\"andig   in  Java\todo{mehr/anderer
Inhalt?}  konvertiert. Die  graphische   Benutzeroberfl\"ache  baut  ganz  auf
Java. Um optimale Wartbarkeit, \"Ubersichtlichkeit und Modularit\"at des Codes
zu gew\"ahrleisten,  ist die Software  gem\"ass Model-View-Controllern-Pattern
aufgebaut.

 % Ergebnisse
Nach der Implementierung  in Matlab wurde klar, dass die  Berechnung durch die
hohe Rechenleistung  sehr schnell  durchgef\"uhrt werden  kann und  somit eine
Dimensionierung des geschlossenen Regelkreises anhand dieser Methode von Herrn
Zellweger m\"oglich ist.

% Aufbau des Berichts
Der  Bericht gliederte  sich in  zwei Teile: Der  ersten Teil  erl\"autert die
theoretischen  Grundlagen und  darauf  aufbauend stellt  der  zweite Teil  der
Aufbau der Software dar.
