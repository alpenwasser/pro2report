% section Software
\todo{Verweis auf Klassendiagramm}

Zweck  der Applikation  ist die  Dimensionierung eines  Reglers ausgehend  von
einer  Strecke  und  der  zugeh\"origen  Schrittantwort. Abschliessend  werden
die  numerischen Parameter  des dimensionierten  Reglers ausgegeben  sowie die
Schrittantwort des geschlossenen Regelkreises grafisch dargestellt.

\todo{Erkl\"arung geschlossener Regelkreis, kurz}

Die   Software    ist   im    bekannten   \emph{Model-View-Controller}-Pattern
aufgebaut. Die   \emph{View}  ist   verantwortlich   f\"ur   den  Aufbau   der
Benutzeroberfl\"ache sowie \todo{sonstige Aufgaben der View}.

Der  \emph{Controller} fungiert  als  Schnittstelle  zwischen \emph{View}  und
\emph{Model}, kontrolliert Benutzereingaben und gibt diese an das \emph{Model}
weiter.

Im \emph{Model} werden s\"amtliche Berechnungen ausgef\"uhrt. Diese beinhalten
die Bestimmung der gesuchten Regelparameter  sowie die Aufbereitung der Daten,
die zur grafischen Darstellung des geschlossenen Regelkreises notwendig sind.


% ------------------------------------------------------------------------------
\subsection{View}
% ------------------------------------------------------------------------------
Die  \emph{View} ist  aus  zwei \"ubergeordneten  Panels aufgebaut. Im  linken
Panel  befinden  sich  Ein-  und Ausgabefelder  f\"ur  numerische  Werte,  das
rechte  Panel  beheimtated  die  Plots  sowie  das  Optimierungs-Panel  (siehe
Abbildung~\ref{fig:toolStartPI}).

\begin{figure}[h!, width=\pagewidth]
    \centering
    \includegraphics[width=\textwidth]{images/toolStartPI.jpg}
    \caption{Grundlegender Aufbau der Benutzeroberfl\"ache}
    \label{fig:toolStartPI}
\end{figure}

Das    rechte   Panel    kann   mittels    der   Check-Box    \emph{erweitert}
(Abbildung~\ref{fig:toolStartPIDSmallErweitert})    ein-   und    ausgeblendet
werden,       die      reduzierte       Benutzeroberfl\"ache      ist       in
Abbildung~\ref{fig:toolStartPIDSmall} zu sehen.

\begin{minipage}[t][][b]{0.45\textwidth}
    \centering

    \begin{minipage}[c][][b]{\textwidth}
        \includegraphics[width=\textwidth]{images/toolStartPIDSmall.jpg}
        \captionof{figure}{Benutzeroberfl\"ache reduziert auf linkes Panel}
        \label{fig:toolStartPIDSmall}
    \end{minipage}

    \begin{minipage}[c][][b]{\textwidth}
        \centering
        \includegraphics[width=\textwidth]{images/toolStartPIDSmallErweitert.jpg}
        \captionof{figure}{Checkbox zum Zu- und Wegschalten des rechten Haupt-Panels}
        \label{fig:toolStartPIDSmallErweitert}
    \end{minipage}

\end{minipage}
\begin{minipage}[t][][b]{0.45\textwidth}
    \centering
    \begin{minipage}[c][][b]{\textwidth}
        \centering
        \includegraphics[width=\textwidth]{images/toolStartPIDSmallSchrittantwort.jpg}
        \captionof{figure}{Bereich zum Eingeben der Streckenparameter}
        \label{fig:toolStartPIDSmallSchrittantwort}
    \end{minipage}

    \begin{minipage}[c][][b]{\textwidth}
        \centering
        \includegraphics[width=\textwidth]{images/toolStartPIDSmallButtons.jpg}
        \captionof{figure}{Auswahl zwischen PI- und PID-Regler}
        \label{fig:toolStartPIDSmallButtons}
    \end{minipage}

    \begin{minipage}[c][][b]{\textwidth}
        \centering
        \includegraphics[width=\textwidth]{images/toolStartPIDSmallRegler.jpg}
        \captionof{figure}{Ausgabewerte des dimensionierten Reglers}
        \label{fig:toolStartPIDSmallRegler}
    \end{minipage}

\end{minipage}

Im             Bereich             \emph{Schrittantwort             vermessen}
(Abbildung~\ref{fig:toolStartPIDSmallSchrittantwort}) werden die Parameter der
vermessenen  Strecke eingegeben. Darunter  befinden sich  die Schalftfl\"achen
zur   Wahl   zwischen  der   Dimensionierung   eines   PI-  respektive   eines
PID-T1-Reglers (Abbildung~\ref{fig:toolStartPIDSmallButtons}).

Das  Panel   \emph{Reglerwerte}  (Abbildung~\ref{fig:toolStartPIDSmallRegler})
dient   der   Ausgabe   der    berechneten   Reglerwerte   der   verschiedenen
Berechnungsmethoden. Ebenfalls  kann f\"ur  die Phasengangmethode,  sofern ein
PID-Regler dimensioniert wird, die  Zeitkonstante $T_p$ spezifiziert werden.

Das  Optimierungs-Panel  (Abbildung~\ref{fig:optimierungen}  )beinhaltet  zwei
Slider  zur  Eingabe  des   gew\"unschten  \"Uberschwingens  respektive  f\"ur
die  Optimierung   des  Reglers  der  Phasengang-Methode. \"Uber   den  Slider
\emph{Optimierung} kann der Phasenrand beeinflusst werden.

\begin{figure}[h!, width=\pagewidth]
    \centering
    \includegraphics[width=0.6\textwidth]{images/tool20UeberschwingenPIDOptimierungen.jpg}
    \caption{Slider im Optimierungs-Panel}
    \label{fig:optimierungen}
\end{figure}

\begin{figure}[h!, width=\pagewidth]
    \centering
    \includegraphics[width=0.6\textwidth]{images/tool20UeberschwingenPIDPlots.jpg}
    \caption{Plots der Schrittantworten, PID-Regler berechnet nach Phasengangmethode, 20\% \"Uberschwingen}
    \label{fig:tool20Plots}
\end{figure}

Unterhalb   des    Optimierungs-Panels   werden   die   Plots    der   mittels
Faustformeln     und     Phasengangmethode    errechneten     Schrittantworten
graphisch    ausgegeben   (Abbildung~\ref{fig:tool20Plots}). Diese    k\"onnen
mittels    Check-Boxen     zu-    und    weggeschaltet     werden. Zu    jeder
Faustformel     wird     die    zugeh\"orige     Schrittantwort     abgebildet
(Abbildung~\ref{fig:toolFaustFormeln}).

Die Resultate der Phasengangmethode werden durch drei Kurven dargestellt. Eine
Kurve   benutzt   den   Standardwert  des   Phasenrands   gem\"ass   Zellweger
($-90\degree$  f\"ur PI-Regler,  $-135\degree$ f\"ur  PID-Regler), die  beiden
anderen Kurven  basieren auf Benutzereingaben  f\"ur einen oberen  und unteren
Offset  des  Phasenrandes  der   \"uber  den  Schiebregler  \emph{Optimierung}
eingestellt werden kann.

\begin{figure}[h!, width=\pagewidth]
    \centering
    \includegraphics[width=\textwidth]{images/toolStartFaustFormeln.jpg}
    \caption{Plots der Schrittantworten nach Faustformeln}
    \label{fig:toolFaustFormeln}
\end{figure}

\begin{figure}[h!, width=\pagewidth]
    \centering
    \includegraphics[width=\textwidth]{images/tool2UeberschwingenOptimierungPID.jpg}
    \caption{Plots der Schrittantworten nach Phasengangmethode, PID-Regler, 2\% \"Uberschwingen}
    \label{fig:tool2UberschwingenPID}
\end{figure}

\clearpage



% ------------------------------------------------------------------------------
\subsection{Controller}
% ------------------------------------------------------------------------------

\todo{GUIController Bezeichnung}

Leserf\"uhrung Controller.
Ausschnitt Klassendiagramm, Verweis auf gesamtes Diagramm.

Regler  heisst \"ubersetzt  auf Englisch  \emph{Controller}. Aus diesem  Grund
heisst die generische Reglerklasse  in unserer Software \code{Controller}. Die
Klasse,   welche   die   Rolle   des   \emph{Controllers}   im   Kontext   von
\emph{Model-View-Controller} wahrnimmt, heisst  daher \code{GUIController}, um
Namenskonflikte zu vermeiden.


Der \code{GUIController} dient als  Schnittstelle zwischen der \code{View} und
dem  \code{Model}. Werden  Eingaben auf  der  \code{View}  durch den  Benutzer
gemacht, wird  der \code{GUIController}  dar\"uber informiert. Er  pr\"uft die
Daten  auf  Zul\"assigkeit  und   leitet  berechnungsrelevante  Daten,  sofern
sie  die  Pr\"ufung bestanden  haben,  an  das \code{Model}  zur  Verarbeitung
weiter. Zus\"atzlich  \"ubernimmt der  \code{GUIController} die  Steuerung der
\code{View}, indem  er Objekte  ein- und ausblendet  sowie Meldungen/Warnungen
\"uber die \code{View} ausgibt.



% ------------------------------------------------------------------------------
\subsection{Model}
% ------------------------------------------------------------------------------

Leserf\"uhrung Model.
Ausschnitt Klassendiagramm, Verweis auf gesamtes Diagramm.

Im Model wird f\"ur jede Berechnungsart, also entwerder f\"ur die Faustformeln
oder die Phasengangmethode, Objekt des Typs \code{CloseLoop} erzeugt.


\subsubsection*{Controller}

Der  Controller  bildet  die  Oberklasse   f\"ur  alle  Faustformeln  und  die
Phasengangmehode. Sie  beinhaltet  die  abstrakte  Methode  \code{calculate()}
sowie alle n\"otigen Getter und Setter.


\subsubsection*{Faustformeln}

\code{Chien20},   \code{ChienApper},   \code{Oppelt},   \code{Rosenberg}   und
\code{ZieglerNichols} sind die Klassen  zu den zugeh\"origen Fausformeln.  Die
abstrakte Methode  \code{calculate()} aus der Klasee  \code{Controller} ist in
jeder dieser Klassen  implementiert. Sie liest die ben\"otigten  Werte aus der
Strecke (Klasse  \code{Path}) aus  und f\"uhrt  die Berechnungen  gem\"ass der
entsprechenden Faustformel aus.


\subsubsection*{ClosedLoop}

In  der  Klasee  \code{ClosedLoop}   wird  f\"ur  jede  Berechungsart  jeweils
ein  \code{Controller}  erstellt. Auch  diese Klasse  enth\"alt  eine  Methode
\code{calculate()}. In   dieser  wird   einerseits   f\"ur  alle   Rechenarten
die  Berechung   der  Schrittantwort   mittels  \code{calculateStepResponse()}
ausgel\"ost   und   f\"ur   die  Phasengangmethode   wird   zus\"atzlich   das
\"Uberschwingverhalten durch \code{overShootOptimazation()} optimiert.


\subsubsection*{PhaseRespsonseMethod}

\todo{Dieser  Abschnitt  sollte  vermutlich nomals  Aufmerksamkeit  von  einem
Experten zum Code erhalten.}

In dieser Klasse  wird die Frequenzachse in Abh\"angigkeit  vom Phasengang und
des ben\"otigten Winkelbereiches erstellt.   Ausserdem finden wir verschiedene
calculate Methoden.

Einerseits  ist da  calculate()  , in  dieser  wird die  UTF  Strecke aus  der
Strecke(path)  geholt  und  die  Omega-Achse  Methode  aufgeruffen. In  dieser
wird Sie  Abh\"angigkeit vom  Phasengang und des  ben\"otigten Winkelbereiches
erstellt.

Anschlissend werden die Werte f\"ur Hs und phiS berehnet.

Andererseits  sind  da die  Methoden,  calculateTnvTnk()  um  Tnk und  Tnv  zu
berechnen, calculateKrk um Krk zu berechnen, calculatecontrollerConf() um eine
umrechung  von Bodekonformen  Werten  zu Reglerkonformen  Werten  in der  calc
Klasse  auszul\"osen, calculatePhaseMargin()  bestimmt je  nach Reglertyp  den
Phasenrand  und zum  schluss calculateOverShoot(),  hier wird  je nach  vorher
berechnetem \"Uberschwingen, dem  Wert phiu einen der  4 vordefinierten static
final Werte zugewiesen.



% ------------------------------------------------------------------------------
\subsection{Benutzungs-Beispiel (Use-Case)}
% ------------------------------------------------------------------------------

Leserf\"uhrung Use-Case.
Ausschnitt Klassendiagramm, Verweis auf gesamtes Diagramm.

Das Zusammenspiel der einzelnen Komponenten der Applikation wird im Folgenden anhand eines Beispiels erkl\"art.
Beim Programmstart werden durch das Model drei \textit{closedLoops}
(geschlossenen Regelkreise) f\"ur die Phasengang-Methode sowie vier weitere f\"ur die Faustformeln erzeugt. Jeder \textit{closedLoops} ist von Beginn an einem Berechnungstyp (Phasengang-Methode, Faustformel) zugewiesen. Er ist bereit, Daten aufzunehmen und zu verarbeiten.
\"Uber die drei Eingabefelder $K_s$, $T_u$ und $T_g$ werden die Werte der vermessenen Regelstrecke durch den Benutzer eingegeben. Durch Dr\"ucken des Buttons „Berechnen“ werden die Eingaben durch den \textit{GUIController} auf Zul\"assigkeit \"uberpr\"uft. Erf\"ullen sie die erforderlichen Kriterien nicht, wird eine Benachrichtigung mit Hinweis auf den Fehler oberhalb des Buttons ausgegeben und die Berechnung nicht ausgel\"ost.
Haben die Eingaben die \"Uberpr\"ufung durch den \textit{GUIController} bestanden, fragt dieser zus\"atzlich die aktuellen Werte/Zust\"ande der Slider f\"ur \"Uberschwingen und Optimierung sowie den Reglertyp auf dem GUI ab und leitet alle Daten mittels \textit{setData()} an das Model weiter. Dieses erzeugt einen Path (Strecke) aus den Eingabewerten. Das Model errechnet den Optimierungs-Offset und weist die Daten den entsprechenden \textit{closedLoops} der Phasengang-Methode sowie der Faustformeln mittels der\textit{setData()}  Methode zu. Jeder \textit{closedLoops} leitet die Daten an den zugeh\"origen Controller weiter, der die Reglerwerte berechnet. Zu Beginn betrachten wir den Regler nach Oppelt genauer.
Die Klasse \textit{Oppelt}  erbt von der abstrakten Klasse Controller und besitzt somit alle setter- und getter-Methoden der Oberklasse. Als Input stehen die Informationen der Strecke sowie der Reglertyp (PI, PID) zur Verf\"ugung. Je nach gew\"ahltem Berechnungstyp werden die Reglerwerte „reglerkonform“ berechnet und gespeichert. Weiter werden die Werte in die „bodekonforme“ Darstellung umgerechnet und ebenfalls abgespeichert.
Die Berechnungstypen Rosenberg, Chien/Hrones/Reswick (20%) sowie Chien/Hrones/Reswick (aperiod.) funktionieren analog dem Berechnungstyp Oppelt  und werden nicht weiter ausgef\"uhrt.
Ein spezielles Augenmerk richten wir nun auf die Berechnung der Phasengang-Methode. Auch die Klasse \textit{phaseResponseMethod }  erbt von der Klasse Controller und bringt die bereits erw\"ahnten setter- und getter-Methoden mit. Die Input-Werte sind analog derjenigen der Faustformeln. Zus\"atzlich werden die Informationen zum \"Uberschwingen, der Optimierung sowie Tp in die Berechnung mit einbezogen und die gesetzten Input-Werte werden den lokalen Attributen zugewiesen. calculateOvershoot() wird ausgel\"ost und setzt das Attribut phiU, welches f\"ur das korrekte \"Uberschwingen ben\"otigt wird. Darauf folgend wird calculate() aufgerufen. Diese Methode berechnet anhand der Methode createOmegaAxis() die diskrete Omega-Achse in Abh\"angigkeit der Zeitkonstante der Regelstrecke. Die \"Ubertragungsfunktion in s der Regelstrecke wird f\"ur alle Punkte von Omega berechnet. calculateTnk() wird ausgel\"ost. Diese Methode berechnet Tnk und Tvk unter Zuhilfenahme der diskreten Werte nach dem Prinzip der Phasengang-Methode. Daraus resultiert die \"Ubertragungs-Funktion des Reglers. Falls Tp = 0 aus der Eingabe \"ubergeben wurde, so wird an dieser Stelle Tp berechnet. Krk wird gem\"ass der Phasengang-Methode mittels calculateKrk() berechnet. Zudem beinhaltet. die Methode calculateKrk() den Aufruf von calculateControllerConf() und setUTF(). calculateControllerConf transformiert die Werte in die „reglerkonforme“ Darstellung. setUTF() setzt die \"Ubertragungs-Funktion des Reglers.
Im closedLoop wird nun die Methode calculate() ausgef\"uhrt, welche mittels calculateStepResponse() die Schrittanwort des geschlossenen Regelkreises berechnet. Falls es sich bei der Berechnung des Reglers um die Phasengang-Methode handelt, wird zus\"atzlich die overShootOptimization() aufgerufen. Diese Methode \"andert den Wert von Krk so lange, bis das gew\"unschte \"Uberschwingen erreicht wird.
Sobald die Berechnungen aller \textit{closedLoops}abgeschlossen sind, wird im Model \textit{notifyObserver()} ausgel\"ost. Hiermit wird die View  dar\"uber informiert, dass \"Anderungen im Model vorgenommen wurden und die Methode \textit{update()} in den jeweiligen View-Unter-Klassen aufgerufen werden soll. Somit aktualisiert sich die View und die neu berechneten Regler-Werte sowie die Strecken-Ordnung werden ausgegeben und die Plot-Daten werden aktualisiert. Im Speziellen f\"ur die Phasengang-Methode, werden auch Start-Werte f\"ur $T_p$ gesetzt.
Der Benutzer hat nun die M\"oglichkeit die Resultate der Phasengang-Methode weiter zu optimieren. \"Uber das Panel \textit{Optimierungen} stehen die Slider ’’\"Uberschwingen“ sowie ’’Optimierung“ zur Verf\"ugung. Das \"Uberschwingen kann in vorgegebenen Schritten in Prozenten bestimmt werden. Die Optimierung schiebt den Phasenrand bzw. den Regler-Knickpunkt in die positive sowie negative Richtung zugleich, und wird mittels zwei separaten Plots dargestellt. Weiter k\"onnen die Werte f\"ur $T_p$ nachtr\"aglich f\"ur jede der drei kurven individuell angepasst werden.
Sobald einer der drei Parameter (\"Uberschwingen, Optimierung, $T_p$) ver\"andert wird, wird \"uber den \textit{GUIController} die jeweilige setter-Methode im Model aufgerufen. Das Model gibt die Daten an den jeweiligen \textit{closedLoop} weiter, der diese wiederum den Methoden der Phasengang-Methode weiterleitet. Sobald die neu berechneten Werte vorliegen wird \textit{notifyObservers()} aufgerufen und die View aktualisiert.
Die Plots sowie die Optimierungs-Schaltfl\"achen sind nur dann sichtbar, wenn die CheckBox ’’Erweiter“ aktiviert ist. Durch deaktivieren dieser CheckBox, kann das Programm in einer Klein-Ansicht, ohne grafische Ausgabe, bedient werden. Die einzelnen Plots k\"onnen \"uber CheckBoxen unterhalb des Plot-Bereichs dazu- oder weggeschaltet werden.
\"Uber den Button „L\"oschen“ k\"onnen alle Regler- sowie Plot-Daten gel\"oscht werden.

