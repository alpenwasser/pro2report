Im Rahmen dieses Projektes soll ein  Tool entwickelt werden, welches einen PI-  % Ausgangslage
respektive einen PID-Regler mittels der von Prof. Jakob Zellweger entwickelten
Phasengangmethode                                                               % TODO: referenz script Zellweger
dimensioniert. Zum Vergleich  soll der entsprechende Regler  ebenfalls mittels
verschiedenen  Faustformeln ausgelegt  werden. Die  Phasengangmethode ist  ein
graphisches Werkzeug,  das normalerweise  mit Stift und  Papier durchgef\"uhrt
wird. Folglicherweise  ist seine  Ausf\"uhrung zeitaufw\"andig,  speziell wenn
verschiedene  Szenarien  mit unterschiedlichen  Parameterwerten  durchgespielt
werden sollen. Ein Tool zur Automatisierung  dieses Prozesses ist bisher nicht
verf\"ugbar; unsere Software soll diese Marktl\"ucke f\"ullen.

Das Tool soll ausgehend von drei Parametern aus der Schrittantwort der Strecke  % Anforderungen
($K_s$ Verst\"arkung, $T_g$: Anstiegszeit, $T_u$: Verz\"ogerungszeit)           % TODO CHECK
mittels  der  Phasengangmethode  m\"oglichst ideale  Regelparameter  berechnen
sowie  die Schrittantwort  des darauf  basierenden geschlossenen  Regelkreises
graphisch  darstellen. Die  Benutzeroberfl\"ache  der Software  soll  intuitiv
sein,   sodass   sich  auch   mit   dem   Thema  nicht   eingehend   vertraute
Regelungstechniker einfach zurechtfinden.

Die  erforderlichen   Algorithmen  wurden  eigenst\"andig  zuerst   in  Matlab % Umsetzung. TODO: mehr/anderer Inhalt?
als  Prototypen   implementiert  und   anschliessend  vollst\"andig   in  Java
konvertiert. Auch die  graphische Benutzeroberfl\"ache baut ganz  auf Java. Um
optimale  Wartbarkeit,  \"Ubersichtlichkeit  und Modularit\"at  des  Codes  zu
gew\"ahrleisten,  ist  die  Software  gem\"ass  Model-View-Controllern-Pattern
aufgebaut.

Nach  der Berechung  in  Matlab  wurde klar,  dass  die  Berechnung durch  die % Ergebnisse
hohe Rechenleistung  sehr schnell  durchgef\"uhrt werden  kann und  somit eine
Dimensionierung des geschlossenen Regelkreises anhand dieser Methode von Herrn
Zellweger m\"oglich ist.

Das  Projekt gliederte  sich prim\"ar  in  zwei Teile. In  einer ersten  Phase % Aufbau des Berichts.
wurden  die  theoretischen  Grundlagen erarbeitet,  darauf  aufbauend  bestand
die  zweite  Phase  vor  allem   aus  der  Implementierung  der  Software. Der
vorliegende Bericht entspricht in seinem  Aufbau diesem Verlauf und beschreibt
haupts\"achlich die erarbeitete Theorie und die entwickelte Software.
