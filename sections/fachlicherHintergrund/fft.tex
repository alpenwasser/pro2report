In  diesem  Projekt ist  im  speziellen  die Schrittantwort  de  geschlossenen
Regelkreises von  Bedeutung. Um auf die  Schrittantwort zu gelangen  wurde der
Weg \"uber die IFFT (Englisch:  \emph{Inverse Fast Fourier Transform}, inverse
schnelle Fourier-Transformation)  gew\"ahlt, da diese relativ  einfach in Java
zu implementieren ist.

F\"ur diese Berechnung sind folgende Schritte notwendig:

\begin{itemize}
    \item
        \"Ubertragungsfunktion des geschlossenen Regelkreises bilden
    \item
        bestimmen der Abstastfrequenz $f_s$ sowie der Anzahl zu berechnender Punkte.
    \item
        diskreten Frequenzgang berechnen
    \item
        Frequenzgang f\"ur die IFFT vorbereiten
    \item
        Frequenzgang mittels IFFT r\"ucktransformieren
    \item
        Schrittantwort aus der Impulsantwort bilden
\end{itemize}


% --------------------------------------------------------------------------- %
\subsubsection*{\"Ubertragungsfunktion des geschlossenen Regelkreises bilden}
% --------------------------------------------------------------------------- %

Da  die \"Ubertragungsfunktionen  des Reglers  sowie der  Regelstrecke bekannt
sind,  kann aus  ihnen  die gesamte  \"Ubertragungsfunktion $H_g(s)$  gebildet
werden. Dazu  werden jeweils  die Z\"ahler-  sowie die  Nennerpolynome mittels
diskreter  Faltung  zusammengef\"uhrt. Zus\"atzlich   muss  das  resultierende
Z\"ahlerpolynom zum Nennerpolynom addiert werden.

\begin{gather} \label{eq:fft:hg}
    \begin{split}
        H_g(s)  & = \frac{ZAH(s)}{NEN(s)} \\
                & = \frac{ZAH_{Regler} \cdot ZAH_{Strecke}}{NEN_{Regler} \cdot NEN_{Strecke} + ZAH_{Regler} \cdot ZAH_{Strecke}}
    \end{split}
\end{gather}

wobei die Z\"ahler- und Nennerpolynome folgende Form haben:

\begin{align*}
    ZAH_{Regler}  & = b_{R_1} s^n  + b_{R_2} s^{n-1}  + \dotsb  + b_{R_{n-1}}     ~|~ n \in  \mathbb{N} \\
    ZAH_{Strecke} & = b_{S_1} s^k  + b_{S_2} s^{k-1}  + \dotsb  + b_{S_{k-1}}     ~|~ k \in  \mathbb{N}  \\
    NEN_{Regler}  & = a_{R_1} s^m  + a_{R_2} s^{m-1}  + \dotsb  + a_{R_{m-1}}     ~|~ m \in  \mathbb{N}  \\
    NEN_{Strecke} & = a_{S_1} s^l  + a_{S_2} s^{l-1}  + \dotsb  + a_{S_{l-1}}     ~|~ l \in  \mathbb{N}
\end{align*}


 --------------------------------------------------------------------------- %
\subsubsection*{Bestimmen der Abtastrate fs sowie der Anzahl zu berechnender Punkte.}
% --------------------------------------------------------------------------- %
F\"ur die Berechnung von $f_s$ durften  wir eine Methode verwenden, welche uns
von  Prof. Dr. Richard Gut  zur  Verf\"ugung  gestellt wurde. Diese  berechnet
$f_s$ anhand der Nullstellen des  Nennerpolynoms und ermittelt gleichzeitig in
Abh\"angigkeit von $f_s$  die Anzahl $N$ zu berechnender  Punkte, deren Anzahl
stets eine  Potenz von  2 ist. Daher  wird an dieser  Stellt nicht  weiter auf
diesen Schritt  eingegangen; das  Matlab-File kann  in Anhang~\ref{app:fftgut}
gefunden werden.


% --------------------------------------------------------------------------- %
\subsubsection*{Diskreten Frequenzgang berechnen}
% --------------------------------------------------------------------------- %

Nun  werden  f\"ur  $s$  genau $\frac{N}{2}$  Werte  eingesetzt,  welche  sich
geichm\"assig zwischen 0  und $f_s \cdot \pi$  verteilen. Dadurch entsteht ein
diskreter Frequenzgang $H_g[k]$ mit $\frac{N}{2}$ Werten.

\begin{equation*}
    H_g[k] = H_g(s) ~|~ s=j \cdot \frac{f_s \cdot \pi}{\frac{N}{2} - 1} \cdot k, 0 \leq k < \frac{N}{2}
\end{equation*}


% --------------------------------------------------------------------------- %
\subsubsection*{Frequenzgang f\"ur die inverse Fourier-Transformation vorbereiten}
% --------------------------------------------------------------------------- %
Damit nach der Transformation in den Zeitbereich ein reelles Resultat vorliegt
muss der diskrete Frequenzgang $H_g[k]$ folgende Symmetriebedingung erf\"ullen
($L$ ist die L\"ange des Arrays $H_g$):

\begin{equation*}
    H_g[L-k]=H_g^*[k]  ~|~ 0 \leq k \leq L
\end{equation*}

Deshalb  wird $H_g[\frac{N}{2}]  = 0$  gesetzt  und zweite  Teil von  $H_g[k]$
jeweils durch die entsprechende Konjugiert-Komplexe $H^*[k]$ erg\"anzt.

% --------------------------------------------------------------------------- %
\subsubsection*{Frequenzgang r\"ucktransformieren}
% --------------------------------------------------------------------------- %
Die  diskrete inverse  Fourier-transformation wird  gem\"ass folgender  Formel
durchgef\"uhrt:

\begin{equation*}
    \vec{h}=\text{ifft}(\vec{H})
\end{equation*}
\todo{Formel}

Die  in diesem  Projekt verwendetete  inverse schnelle  Fourier-Transformation
ist   ein  verbessertes   Verfahren  der   oben  erkl\"akrten   IFFT.   Dieses
erm\"oglicht eine schnellere Berechnung durch  das Zerlegen der Punkte in ihre
Primfaktoren. Eine genauere Erl\"auterung wird an dieser Stelle nicht gemacht,
da dieses Verfahren  in Matlab sowie in Java als  fertige Funktion verf\"ugbar
ist.


% --------------------------------------------------------------------------- %
\subsubsection*{Schrittantwort aus der Impulsantwort bilden}
% --------------------------------------------------------------------------- %
Aus  der  Impulsantwort $h[k]$  kann  nun  in  einem einfachen  Verfahren  die
Schrittantwort  $y[k]$ gebildet  werden.  Dabei  entspricht jedes  Element der
Schrittantwort  $y[k]$ der  Summe der  Elemente der  Impulsantwort $h[k]$  vom
ersten  bis zum  k-ten Element.   Zus\"atzlich  wird die  zweite H\"alfte  der
Impulsantwort  weggeschnitten,  um  die  zuvor  als  konjugert-komplexe  Werte
hinzugef\"ugten \"Uberbleibsel zu eliminieren.

\begin{equation*}
    y[k]= \sum_{n=0}^k h[n] ~|~ 0 \leq k < \frac{L}{2}
\end{equation*}
