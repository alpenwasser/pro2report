
In der Regelungstechnik ist die korrekte Dimensionierung der Regler grundlegend. Die zu regelnde Strecke kann nur durch richtig eingestellte Regelwerte wie gewünscht verändert werden. 

Dieses Projekt hat sich zum Ziel gesetzt ein Softwaretool zu entwickeln, welches aus den eingegebenen Streckenwerten $K_s$, $T_u$ und $T_g$ PI- und PID-Regler dimensioniert. Das Tool berechnet die Reglerwerte mit der sogenannten Phasengangmethode, welche von Jakob Zellweger, ehemaliger Dozent der Fachhochschule Nordwestschweiz, stammt. Diese Methode wurde ursprünglich für die graphische Auswertung des Phasengangs entwickelt und durch Handarbeit mit Geodreieck und Bleistift angewendet. Das Softwaretool ermöglicht die bewährte Methode effizient durch ein automatisiertes Verfahren anzuwenden.

Das entwickelte Softwaretool simuliert neben den berechneten Reglerwerten auch die Schrittantwort des geschlossenen Regelkreises. Zusätzlich wurde die Berechnung dreier gängigen Faustformeln mit implementiert. Zum Vergleich können die Schrittantworten der Faustformeln im gleichen Plot angezeigt werden.
Für die Phasengangmethode werden gleichzeitig drei Graphen abgebildet. Einer mit der Standart-Reglerknickfrequenz gemäss Phasengangmethode und zwei weitere welche sich in der Reglerknickfrequenz um ± 10° unterscheiden.

Zusätzlich kann die Reglerknickfrequenz und das Überschwingen mittels Schiebregler manuell verändert werden. Dies ermöglicht die im Vergleich, zu der ursprünglichen Handarbeit, enorm hohe Geschwindigkeit. Somit wird eine Echtzeit Optimierung der bestehenden Phasengangmethode getätigt.



