% was laeuft, was laeuft nicht, was kann verbessert werden
Das Projekt wurde erfolgreich abgeschlossen und alle Punkte des ursprünglichen Auftrages erfüllt. Aus den Eingabewerten wird das dynamische Verhalten des geschlossenen Regelkreises berechnet und graphisch dargestellt. Die Auftragserweiterungen wurden eingebaut, sodass man den Regler mit der Phasengangmethode dimensionieren und die Knickfrequenz und das Überschwingen manuell verändert werden kann. \\Die vor der Erweiterung des Auftrags geplanten graphische Ausgabe des Amplitudengangs und der Strecke sowie die Schiebregler für das Verändern der Reglerwerte wurden weggelassen. Der Grund für diesen Entscheid war die Möglichkeit durch die hohe Rechenleistung den Regler echtzeit zu dimensioniern.\\
\\
Als Erweiterung könnten die Streckenwerte ausgegeben und die Strecke gezeichnet werden. Ausserdem könnte das automatische Einlesen der Streckenwerte aus einer graphisch Form hinzugefügt werden. Das Tool würde an die eingelesene Kurve die Wendetangente legen und die Werte ablesen. Das Ablesen könnte mit der Monte-Carlo-Studie verfeinert werden.