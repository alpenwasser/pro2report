\todo{ganz \"uberarbeiten}
% was laeuft, was laeuft nicht, was kann verbessert werden
Das   Projekt   wurde   erfolgreich   abgeschlossen  und   alle   Punkte   des
urspr\"unglichen   Auftrages  erf\"ullt. Aus   den   Eingabewerten  wird   das
dynamische Verhalten  des geschlossenen  Regelkreises berechnet  und graphisch
dargestellt. Einige Erweiterungen wurden eingebaut,  sodass man den Regler mit
der Phasengangmethode dimensioniert und zus\"atzlich.\todo{Satz abgebrochen}


Die vor der  Erweiterung des Auftrags optional  geplanten graphischen Ausgaben
des Amplitudengangs der  Strecke sowie die Schiebregler  f\"ur das Ver\"andern
der Reglerwerte wurden  weggelassen. Der Grund f\"ur diesen  Entscheid war die
M\"oglichkeit,  durch  die  hohe  Rechenleistung den  Regler  in  Echtzeit  zu
optimieren, womit diese Funktionen \"uberfl\"ussig wurden.\todo{Wieso wird der
Amplitudengang der Strecke \"uberfl\"ussig durch Echtzeigoptimierung?}


Alternativ kann  die Schrittantwort  mit zwei Schiebreglern  manuell optimiert
werden. Durch  das Ver\"andern  des  \"Uberschwingens und  das Einstellen  der
Optimierung kann die Kurve per  Hand angepasst werden.  Als Erweiterung w\"are
es  m\"oglich,  die  Streckenwerte  auszugegeben  und  die  Strecke  graphisch
zur\"uckgegeben wird. Das  automatische Einlesen  der Streckenwerte  aus einer
graphisch Form k\"onnte hinzugef\"ugt werden. Das Tool w\"urde an die eingelesene
Kurve die Wendetangente legen und  die Werte ablesen. Das Ablesen k\"onnte mit
der Monte-Carlo-Studie verfeinert werden.\todo{Wenn  wir das schon erw\"ahnen,
sollte man nicht auch erkl\"aren, was das ist?}
