% section Software
\todo{Verweis auf Klassendiagramm}

Zweck  der Applikation  ist die  Dimensionierung eines  Reglers ausgehend  von
einer  Strecke  und  der  zugeh\"origen  Schrittantwort. Abschliessend  werden
die  numerischen Parameter  des dimensionierten  Reglers ausgegeben  sowie die
Schrittantwort des geschlossenen Regelkreises grafisch dargestellt.

Die   Software    ist   im    bekannten   \emph{Model-View-Controller}-Pattern
aufgebaut.


% ------------------------------------------------------------------------------
\subsection{View}
% ------------------------------------------------------------------------------
Die  \emph{View} ist  aus  zwei \"ubergeordneten  Panels aufgebaut. Im  linken
Panel befinden sich Ein- und  Ausgabefelder f\"ur numerische Werte, das rechte
Panel  beheimtated die  Plots sowie  das Optimierungs-Panel  und kann  mittels
Check-Box "erweiter" ein- und ausgeblendet werden.

\todo{Image Gesamt-GUI}

\todo{Image Panel Schrittantwort vermessen}


Im  Bereich 1  \todo{Image Referenzen}  werden die  Parameter der  vermessenen
Strecke  eingegeben. Darunter  befinden  sich die  Schalftfl\"achen  zur  Wahl
\todo{Image Butttons  PI-, PID-T1-Regler}  zwischen der  Dimensionierung eines
PI- respektive eines PID-T1-Reglers.

Das   Panel    \emph{Reglerwerte}   dient   der   Ausgabe    der   berechneten
Reglerwerte  der   verschiedenen  Berechnungsmethoden. Ebenfalls   kann  f\"ur
die  Phasengang-Methode,   sofern  ein  PID-Regler  dimensioniert   wird,  die
Zeitkonstante  $T_p$  \todo{Check:  korrekter  Begriff}  spezifiziert  werden.
\todo{Image Panel Phasengangmethode}

Das Optimierungs-Panel  beinhaltet zwei  Slider zur Eingabe  des gew\"unschten
\"Uberschwingens   respektive   f\"ur   die  Optimierung   des   Reglers   der
Phasengang-Methode.

Unterhalb des  Optimierungs-Panels werden  die Plots der  mittels Faustformeln
und Phasengangmethode errechneten Schrittantworten graphisch ausgegeben. Diese
k\"onnen   mittels  Check-Boxen   zu-  und   weggeschaltet  werden. Zu   jeder
Faustformel wird die zugeh\"orige Schrittantwort abgebildet. Die Resultate der
Phasengangmethode  werden durch  drei Kurven  dargestellt. Eine Kurve  benutzt
den  Standardwert des  Phasenrands gem\"ass  Zellweger \todo{Einf\"ugen  Wert,
Referenz}, die beiden anderen Kurven basieren auf Benutzereingaben f\"ur einen
oberen  und  unteren  Offset  des Phasenrandes  der  \"uber  den  Schiebregler
\emph{Optimierung} eingestellt werden kann.



% ------------------------------------------------------------------------------
\subsection{Controller}
% ------------------------------------------------------------------------------
Der  \emph{Controller}   ist  verantwortlich  f\"ur  die   Steueraufgaben  und
erzeugt    den    Regler. Da    \emph{Regler}   \"ubersetzt    auf    Englisch
\emph{Controller} ist, heisst die  generische Reglerklasse in unserer Software
\emph{Controller}. Die  Klasse, welche  die  die  Rolle des  \emph{Controller}
im   Kontext   von   \emph{Model-View-Controller}  wahrnimmt,   heisst   daher
\emph{GUIController}, um Namenskonflikte zu vermeiden.



% ------------------------------------------------------------------------------
\subsection{Model}
% ------------------------------------------------------------------------------

Leserf\"uhrung Model.
Ausschnitt Klassendiagramm, Verweis auf gesamtes Diagramm.

Im Model wird f\"ur jede Berechnungsart, also entwerder f\"ur die Faustformeln
oder die Phasengangmethode, Objekt des Typs \code{CloseLoop} erzeugt.


\subsubsection*{Controller}

Der  Controller  bildet  die  Oberklasse   f\"ur  alle  Faustformeln  und  die
Phasengangmehode. Sie  beinhaltet  die  abstrakte  Methode  \code{calculate()}
sowie alle n\"otigen Getter und Setter.


\subsubsection*{Faustformeln}

\code{Chien20},   \code{ChienApper},   \code{Oppelt},   \code{Rosenberg}   und
\code{ZieglerNichols} sind die Klassen  zu den zugeh\"origen Fausformeln.  Die
abstrakte Methode  \code{calculate()} aus der Klasee  \code{Controller} ist in
jeder dieser Klassen  implementiert. Sie liest die ben\"otigten  Werte aus der
Strecke (Klasse  \code{Path}) aus  und f\"uhrt  die Berechnungen  gem\"ass der
entsprechenden Faustformel aus.


\subsubsection*{ClosedLoop}

In  der  Klasee  \code{ClosedLoop}   wird  f\"ur  jede  Berechungsart  jeweils
ein  \code{Controller}  erstellt. Auch  diese Klasse  enth\"alt  eine  Methode
\code{calculate()}. In   dieser  wird   einerseits   f\"ur  alle   Rechenarten
die  Berechung   der  Schrittantwort   mittels  \code{calculateStepResponse()}
ausgel\"ost   und   f\"ur   die  Phasengangmethode   wird   zus\"atzlich   das
\"Uberschwingverhalten durch \code{overShootOptimazation()} optimiert.


\subsubsection*{PhaseRespsonseMethod}

\todo{Dieser  Abschnitt  sollte  vermutlich nomals  Aufmerksamkeit  von  einem
Experten zum Code erhalten.}

In dieser Klasse  wird die Frequenzachse in Abh\"angigkeit  vom Phasengang und
des ben\"otigten Winkelbereiches erstellt.   Ausserdem finden wir verschiedene
calculate Methoden.

Einerseits  ist da  calculate()  , in  dieser  wird die  UTF  Strecke aus  der
Strecke(path)  geholt  und  die  Omega-Achse  Methode  aufgeruffen. In  dieser
wird Sie  Abh\"angigkeit vom  Phasengang und des  ben\"otigten Winkelbereiches
erstellt.

Anschlissend werden die Werte f\"ur Hs und phiS berehnet.

Andererseits  sind  da die  Methoden,  calculateTnvTnk()  um  Tnk und  Tnv  zu
berechnen, calculateKrk um Krk zu berechnen, calculatecontrollerConf() um eine
umrechung  von Bodekonformen  Werten  zu Reglerkonformen  Werten  in der  calc
Klasse  auszul\"osen, calculatePhaseMargin()  bestimmt je  nach Reglertyp  den
Phasenrand  und zum  schluss calculateOverShoot(),  hier wird  je nach  vorher
berechnetem \"Uberschwingen, dem  Wert phiu einen der  4 vordefinierten static
final Werte zugewiesen.



% ------------------------------------------------------------------------------
\subsection{Benutzungs-Beispiel (Use-Case)}
% ------------------------------------------------------------------------------

Leserf\"uhrung Use-Case.
Ausschnitt Klassendiagramm, Verweis auf gesamtes Diagramm.

Das Zusammenspiel der einzelnen Komponenten  der Applikation wird im Folgenden
anhand  eines   Beispiels  erkl\"art. Die  hier   dargelegten  Zusammenh\"ange
k\"onnen   auch   gut   im   Klassendiagramm   (Anhang~\ref{app:classdiagram})
betrachtet werden.

Beim   Programmstart   werden   durch  das   Model   drei   \code{closedLoops}
(geschlossener  Regelkreis) f\"ur  die Phasengangmethode  sowie vier  weitere
f\"ur  die Faustformeln  erzeugt. Jeder \code{closedLoops}  ist von  Beginn an
einem  Berechnungstyp  (Phasengangmethode,  Faustformel)  zugewiesen. Er  ist
bereit, Daten aufzunehmen und zu verarbeiten.

\"Uber die  drei Eingabefelder  $K_s$, $T_u$  und $T_g$  werden die  Werte der
vermessenen Regelstrecke  durch den  Benutzer eingegeben. Durch  Dr\"ucken des
Buttons  „Berechnen“ werden  die Eingaben  durch den  \code{GUIController} auf
Zul\"assigkeit  \"uberpr\"uft. Erf\"ullen  sie  die  erforderlichen  Kriterien
nicht,  wird eine  Benachrichtigung mit  Hinweis auf  den Fehler  oberhalb des
Buttons ausgegeben und die Berechnung nicht ausgel\"ost.

Haben  die   Eingaben  die  \"Uberpr\"ufung  durch   den  \code{GUIController}
bestanden,  fragt  dieser  zus\"atzlich  die  aktuellen  Werte/Zust\"ande  der
Slider  f\"ur \"Uberschwingen  und  Optimierung sowie  den  Reglertyp auf  dem
GUI  ab  und   leitet  alle  Daten  mittels  \code{setData()}   an  das  Model
weiter. Dieses erzeugt  einen Path (Strecke) aus  den Eingabewerten. Das Model
errechnet  den  Optimierungs-Offset und  weist  die  Daten den  entsprechenden
\code{closedLoops} der  Phasengangmethode sowie der Faustformeln  mittels der
Methode \code{setData()}  zu. Jeder \code{ClosedLoop} leitet die  Daten an den
zugeh\"origen  Controller weiter,  der  die  Reglerwerte berechnet. Zu  Beginn
betrachten wir den Regler nach Oppelt genauer.

Die  Klasse \code{Oppelt}  erbt  von der  abstrakten Klasse  \code{Controller}
und  besitzt  somit  alle  Setter-  und  Getter-Methoden  der  Oberklasse. Als
Input  stehen die  Informationen der  Strecke  sowie der  Reglertyp (PI,  PID)
zur  Verf\"ugung. Je nach  gew\"ahltem Berechnungstyp  werden die  Reglerwerte
reglerkonform  berechnet  und  gespeichert. Weiter  werden die  Werte  in  die
bodekonforme  Darstellung umgerechnet  und  ebenfalls  abgespeichert (mehr  zu
diesem Thema in Abschnitt~\ref{subs:bode_regler}).

Die    Berechnungstypen   Rosenberg,    Chien/Hrones/Reswick   (20\%)    sowie
Chien/Hrones/Reswick (aperiod.) funktionieren analog dem Berechnungstyp Oppelt
und werden nicht weiter ausgef\"uhrt.

Ein   spezielles  Augenmerk   richten   wir  nun   auf   die  Berechnung   der
Phasengangmethode. Auch  die   Klasse  \code{PhaseResponseMethod}   erbt  von
der  Klasse  \code{Controller}  und  bringt die  bereits  erw\"ahnten  Setter-
und   Getter-Methoden  mit. Die   Input-Werte  sind   analog  derjenigen   der
Faustformeln. Zus\"atzlich  werden  die   Informationen  zum  \"Uberschwingen,
der   Optimierung   sowie   $T_p$   in  die   Berechnung   miteinbezogen   und
die   gesetzten  Input-Werte   werden  den   lokalen  Attributen   zugewiesen.
\code{calculateOvershoot()}   wird   ausgel\"ost   und  setzt   das   Attribut
\code{phiU},   welches   f\"ur   das   korrekte   \"Uberschwingen   ben\"otigt
wird. Darauf   folgend  wird   \code{calculate()}  aufgerufen. Diese   Methode
berechnet   anhand   der   Methode   \code{createOmegaAxis()}   die   diskrete
Omega-Achse  in  Abh\"angigkeit  der  Zeitkonstanten  der  Regelstrecke.   Der
Wert  der  \"Ubertragungsfunktion  der  Regelstrecke wird  f\"ur  alle  Punkte
der Frequenzachse  ausgewertet. \code{calculateTnk()}  wird ausgel\"ost. Diese
Methode  berechnet   $T_nk$  und  $T_vk$  unter   Zuhilfenahme  der  diskreten
Werte   nach  dem   Prinzip  der   Phasengangmethode. Daraus  resultiert   die
\"Ubertragungsfunktion  des   Reglers. Falls  $T_p   =  0$  aus   der  Eingabe
\"ubergeben   wurde,  wird   an  dieser   Stelle  $T_p   =  \frac{T_{vk}}{10}$
gesetzt. $K_{rk}$    wird     gem\"ass    der     Phasengangmethode    mittels
\code{calculateKrk()}     berechnet. Zudem      beinhaltet     die     Methode
\code{calculateKrk()}   den    Aufruf   von   \code{calculateControllerConf()}
und   \code{setUTF()}.   \code{calculateControllerConf()}  transformiert   die
Werte   in  die   reglerkonforme   Darstellung.   \code{setUTF()}  setzt   die
\"Ubertragungs-Funktion des Reglers.

Im    \code{ClosedLoop}    wird    nun    die    Methode    \code{calculate()}
ausgef\"uhrt,  welche   mittels  der   Methode  \code{calculateStepResponse()}
die   Schrittanwort  des   geschlossenen   Regelkreises  berechnet. Falls   es
sich  bei  der  Berechnungsmethode   des  Reglers  um  die  Phasengangmethode
handelt,  wird  zus\"atzlich \code{overShootOptimization()}  (dokumentiert  in
Anhang~\ref{app:algo:oversh}) aufgerufen. Diese Methode  \"andert den Wert von
$K_{rk}$ so lange, bis das gew\"unschte \"Uberschwingen erreicht wird.

Sobald die  Berechnungen aller \code{ClosedLoops} abgeschlossen  sind, wird im
\code{Model}  die  Methode \code{notifyObserver()}  ausgel\"ost. Hiermit  wird
die  \code{View}  dar\"uber  informiert,  dass  \"Anderungen  im  \code{Model}
vorgenommen  wurden   und  die  Methode  \code{update()}   in  den  jeweiligen
Unterklassen der Klasse  \code{View}aufgerufen werden soll. Somit aktualisiert
sich die \code{View} und  die neu berechneten Reglerwerte. Die Streckenordnung
wird  ausgegeben   und  die   Plot-Daten  werden  aktualisiert. Im   Fall  der
Phasengangmethode werden auch Startwerte f\"ur $T_p$ gesetzt.

Der Benutzer  hat nun die  M\"oglichkeit, die Resultate  der Phasengangmethode
weiter  zu  optimieren. \"Uber  das   Panel  \code{Optimierungen}  stehen  die
Slider  \emph{\"Uberschwingen} sowie  \emph{Optimierung} zur  Verf\"ugung. Das
\"Uberschwingen  kann  in  vorgegebenen   Schritten  in  Prozenten  festgelegt
werden. Die  Optimierung  schiebt  $\varphi_{start}$  in  die  positive  sowie
negative   Richtung   zugleich  und   wird   mittels   zwei  separaten   Plots
dargestellt. Weiter k\"onnen  die Werte f\"ur $T_p$  nachtr\"aglich f\"ur jede
der drei Kurven individuell angepasst werden.

Sobald  einer   der  drei  Parameter  (\"Uberschwingen,   Optimierung,  $T_p$)
ver\"andert  wird,   wird  \"uber   den  \code{GUIController}   die  jeweilige
Setter-Methode im \code{Model} aufgerufen. Das  \code{Model} gibt die Daten an
den jeweiligen \code{ClosedLoop}  weiter, der diese wiederum  den Methoden der
Phasengangmethode  weiterleitet. Sobald die  neu berechneten  Werte vorliegen
wird \code{notifyObservers()} aufgerufen und die \code{View} aktualisiert.

Die  Plots  sowie die  Optimierungs-Schaltfl\"achen  sind  nur dann  sichtbar,
wenn die  CheckBox \emph{erweitert}  aktiviert ist. Durch  Deaktivieren dieser
CheckBox  kann das  Programm  in einer  Klein-Ansicht  ohne grafische  Ausgabe
bedient werden. Die  einzelnen Plots k\"onnen \"uber  CheckBoxen unterhalb des
Plot-Bereichs dazu- oder weggeschaltet werden.

\"Uber  den Button  \emph{L\"oschen}  k\"onnen alle  Regler- sowie  Plot-Daten
gel\"oscht werden.



% ------------------------------------------------------------------------------
\subsection{Beschreibung der Algorithmen}
% ------------------------------------------------------------------------------

% ---------------------------------------------------------------------------- %
\subsection{Sani}
% ---------------------------------------------------------------------------- %

\subsubsection*{Input}

\begin{tabular}{p{40mm}l}
    $ T_u $ & Verzugszeit \\
    $ T_g $ & Anstiegszeit
\end{tabular}

\subsubsection*{Output}
\begin{tabular}{p{40mm}l}
    $ n $ & Ordnung der Regelstrecke \\
    $ T $ & Zeitkonstante
\end{tabular}

\subsubsection*{Algorithmus}
\begin{enumerate}
    \item
        Ung\"ultige Eingaben warden abgefangen und ein Fehler zur\"uckgegeben.
    \item
        L\"adt Werte f\"ur Tu und Tg.
    \item
        Erstellt 50 Werte zwischen 0 und 1 f\"ur ri.
    \item
        Bestimmt die Ordnung der Regelstrecke.
    \item
        Spline f\"ur r und w
    \item
        T(n) wird aus w*tg berechnet.
    \item
        Umspeichern \& Sortieren
\end{enumerate}

\subsubsection*{Matlab-Code}
\lstinputlisting[style=Matlab-editor-2]{mfiles/p2Sani.m}


\clearpage
% ---------------------------------------------------------------------------- %
\subsection{Umrechnung von reglerkonformer in bodekonforme Darstellung}
% ---------------------------------------------------------------------------- %

\subsubsection*{Input}
\begin{tabular}{p{40mm}l}
    $ T_v $        & Vorhaltezeit \\
    $ T_n $        & Nachstellzeit \\
    $ T_p $        & Periodendauer \\
    $ K_r $        & Verst\"arkungsfaktor des Reglers \\
      Reglertyp    & Typ des Reglers (P, PI, PID)
\end{tabular}

\subsubsection*{Output}
\begin{tabular}{p{40mm}l}
    $ T_{nk} $ & Nachstellzeit \\
    $ T_{vk} $ & Vorhaltezeit \\
    $ K_{rk} $ & Verst\"arkungsfaktor des Reglers
\end{tabular}

\subsubsection*{Algorithmus}
\begin{enumerate}
    \item
        W\"ahlt je nach Reglertyp die Umrechnungsformel.
    \item
        Falls der I-Regler gew\"ahlt wird, gibt der Algorithmus einen Fehler zur\"uck, da der I-Regler nicht implementiert ist.
    \item
        PI-Regler:
        $T_{nk} = T_n$, $K_{rk} = K_r$, $T_{vk} = 0$
    \item
        F\"ur PID-Regler:
        \begin{equation*}
            \varepsilon =\frac{\sqrt{1-(4 \cdot T_n \cdot (T_v-T_p))}}{(T_n+T_p)^2}
        \end{equation*}

        \begin{equation*}
            T_{nk} = \frac{(T_n+T_p) \cdot (1+\varepsilon)}{2}
        \end{equation*}

        \begin{equation*}
            K_{rk} = \frac{K_r \cdot (\frac{1+T_p}{T_{nk}}) \cdot (1+\varepsilon)}{2}
        \end{equation*}

        \begin{equation*}
            T_{vk} = \frac{(T_n+T_p)*(1+\varepsilon)}{2}
        \end{equation*}
\end{enumerate}
\subsubsection*{Matlab-Code}
\lstinputlisting[style=Matlab-editor-2]{mfiles/p2Bodekonf.m}


% ---------------------------------------------------------------------------- %
\subsection{Umrechnung von bodekonformer in reglerkonforme Darstellung}
% ---------------------------------------------------------------------------- %

\subsubsection*{Input}

\begin{tabular}{p{40mm}l}
    $ T_p  $ & Periodendauer \\
    $ T{nk} $ & Nachstellzeit \\
    $ T{vk} $ & Vorhaltezeit \\
    $ K{rk} $ & Verst\"arkungsfaktor des Reglers \\
      Reglertyp   & Reglertyp (P, PI, PID)
\end{tabular}

\subsubsection*{Output}
\begin{tabular}{p{40mm}l}
    $ T_n $ & Nachstellzeit \\
    $ T_v $ & Vorhaltezeit \\
    $ K_r $ & Verst\"arkungsfaktor des Reglers
\end{tabular}

\subsubsection*{Algorithmus}
\begin{enumerate}
    \item
        W\"ahlt je nach Reglertyp die Umrechnungsformel.
    \item
        Falls der I-Regler gew\"ahlt wird, gibt der Algorithmus einen Fehler zur\"uck, da der I-Regler nicht implementiert ist.
    \item
        PI-Regler:
        $T_{n} = T_{nk}$, $K_{k} = K_{rk}$, $T_{v} = 0$
    \item
       F\"ur PID-Regler:

       \begin{equation*}
           K_r= K_{rk} \cdot \frac{1+T_{vk}}{T_{nk}}
       \end{equation*}

       \begin{equation*}
           T_n= T_{nk}+T_{vk}-T_p
       \end{equation*}

       \begin{equation*}
           T_v = \frac{T_{nk} \cdot T_{vk}}{T_{nk}+T_{vk}-T_p)-T_p}
       \end{equation*}

\end{enumerate}

\subsubsection*{Matlab-Code}
\lstinputlisting[style=Matlab-editor-2]{mfiles/p2Reglerkonf.m}


\clearpage
% ---------------------------------------------------------------------------- %
\subsection{utfController}
% ---------------------------------------------------------------------------- %

\subsubsection*{Input}

\begin{tabular}{p{40mm}l}
    $ T_p $        & Verzugszeit \\
    $ T_{nk} $     & Nachstellzeit \\
    $ T_{vk} $     & Vorhaltezeit \\
    $ K_{rk} $     & Verst\"arkungsfaktor des Reglers \\
      Reglertyp    & Reglertyp (P, PI, PID)
\end{tabular}

\subsubsection*{Output}
\begin{tabular}{p{40mm}l}
    Z\"ahler & Reeller Z\"ahler \\
    Nenner   & Reeller Nenner
\end{tabular}\todo{korrekt? Oder einfach reelle Koeffizienten?}

\subsubsection*{Algorithmus}
\begin{enumerate}
    \item
        W\"ahlt je nach Reglertyp die korrekten Formeln.
    \item
        \todo{Matrix}
\end{enumerate}

\subsubsection*{Matlab-Code}
\lstinputlisting[style=Matlab-editor-2]{mfiles/p2UTFRegler.m}

\subsection{Faustformel Oppelt}

\subsubsection*{Input}

\begin{tabular}{p{40mm}l}
    $ T_p $        & Verzugszeit \\
    $ T_u $        & Anstiegszeit \\
    $ K_s $        & Verst\"arkung der Strecke \\
      Reglertyp    & Reglertyp (P, PI, PID)
\end{tabular}

\subsubsection*{Output}
\begin{tabular}{p{40mm}l}
    $ K_p $ & Proportionalit\"atsfaktor \\
    $ T_n $ & Nachstellzeit \\
    $ T_v $ & Vorhaltezeit
\end{tabular}

\subsubsection*{Algorithmus}
\begin{enumerate}
    \item
        W\"ahlt je nach Reglertyp die korrekten Formeln.
    \item
        F\"ur PI:
        \begin{equation*}
             K_p= \frac{0.8}{K_s} \cdot \frac{T_g}{T_u}
        \end{equation*}

        \begin{equation*}
             T_n=3 \cdot T_u
        \end{equation*}

        \begin{equation*}
              t_v=0
        \end{equation*}
    \item
        F\"ur PID:
        \begin{equation*}
            K_p = \frac{1.2}{K_s} \cdot \frac{T_g}{T_u}
        \end{equation*}

        \begin{equation*}
            T_n=2 \cdot T_u
        \end{equation*}

        \begin{equation*}
            T_v=0.42*T_u
        \end{equation*}
\end{enumerate}

\subsubsection*{Matlab-Code}
\lstinputlisting[style=Matlab-editor-2]{mfiles/p2Ffoppelt.m}


\clearpage
% ---------------------------------------------------------------------------- %
\subsection{Faustformel Rosenberg}
% ---------------------------------------------------------------------------- %

\subsubsection*{Input}

\begin{tabular}{p{40mm}l}
    $ T_p $        & Verzugszeit \\
    $ T_u $        & Anstiegszeit \\
    $ K_s $        & Verst\"arkung der Strecke \\
      Reglertyp    & Reglertyp (P, PI, PID)
\end{tabular}

\subsubsection*{Output}
\begin{tabular}{p{40mm}l}
    $ K_p $ & Proportionalit\"atsfaktor \\
    $ T_n $ & Nachstellzeit \\
    $ T_v $ & Vorhaltezeit
\end{tabular}

\subsubsection*{Algorithmus}
\begin{enumerate}
    \item
        W\"ahlt je nach Reglertyp die korrekten Formeln.
    \item
    \item
        F\"ur PI:

        \begin{equation*}
            K_p= \frac{0.91}{K_s} \cdot \frac{T_g}{T_u}
        \end{equation*}

        \begin{equation*}
            T_n=3.3 \cdot T_u
        \end{equation*}

        \begin{equation*}
            T_v=0
        \end{equation*}
    \item
        F\"ur PID:

        \begin{equation*}
            Kp = \frac{1.2}{K_s} \cdot \frac{T_g}{T_u}
        \end{equation*}  \\

        \begin{equation*}
            T_n=2 \cdot T_u
        \end{equation*}

        \begin{equation*}
            T_v=0.45 \cdot T_u;
        \end{equation*}
\end{enumerate}

\subsubsection*{Matlab-Code}
\lstinputlisting[style=Matlab-editor-2]{mfiles/p2Ffrosenberg.m}


\clearpage
% ---------------------------------------------------------------------------- %
\subsection{Faustformel Ziegler}
% ---------------------------------------------------------------------------- %

\subsubsection*{Input}

\begin{tabular}{p{40mm}l}
    $ T_p $        & Verzugszeit \\
    $ T_u $        & Anstiegszeit \\
    $ K_s $        & Verst\"arkung der Strecke \\
      Reglertyp   & Reglertyp (P, PI, PID)
\end{tabular}

\subsubsection*{Output}
\begin{tabular}{p{40mm}l}
    $ K_p $ & Proportionalit\"atsfaktor \\
    $ T_n $ & Nachstellzeit \\
    $ T_v $ & Vorhaltezeit
\end{tabular}

\subsubsection*{Algorithmus}
\begin{enumerate}
    \item
        W\"ahlt je nach Reglertyp die korrekten Formeln.
    \item
        F\"ur PI:
        \begin{equation*}
            K_p= \frac{0.9}{K_s} \cdot \frac{T_g}{T_u}
        \end{equation*}

        \begin{equation*}
            T_n=3.33 \cdot T_u
        \end{equation*}

        \begin{equation*}
            T_v=0
        \end{equation*}
    \item
        F\"ur PID:
        \begin{equation*}
            K_p = \frac{1.2}{K_s} \cdot \frac{T_g}{T_u}
        \end{equation*}

        \begin{equation*}
            T_n=2 \cdot T_u
        \end{equation*}

        \begin{equation*}
            T_v = 0.5 \cdot T_u
        \end{equation*}
\end{enumerate}

\subsubsection*{Matlab-Code}
\lstinputlisting[style=Matlab-editor-2]{mfiles/p2Ffziegler.m}

\clearpage
% ---------------------------------------------------------------------------- %
\subsection{Faustformel Chien}
% ---------------------------------------------------------------------------- %

\subsubsection*{Input}

\begin{tabular}{p{40mm}l}
    $ T_p $             & Verzugszeit \\
    $ T_u $             & Anstiegszeit \\
    $ K_s $             & Verst\"arkung der Strecke \\
      Reglertyp         & Reglertyp (P, PI, PID) \\
    \"Uberschwingen     & Flag f\"ur \"Uberschwingeng %\TODO{specifics $
\end{tabular}

\subsubsection*{Output}
\begin{tabular}{p{40mm}l}
    $ K_p $ & Proportionalit\"atsfaktor \\
    $ T_n $ & Nachstellzeit \\
    $ T_v $ & Vorhaltezeit
\end{tabular}

\subsubsection*{Algorithmus}
\begin{enumerate}
    \item
        W\"ahlt je nach Reglertyp die korrekten Formeln.
    \item
        F\"ur PI, ohne \"Uberschwingen:
        \begin{equation*}
            K_p= \frac{0.35 \cdot T_g}{K_s \cdot T_u}
        \end{equation*}

        \begin{equation*}
            T_n=1.2 \cdot T_u
        \end{equation*}

        \begin{equation*}
            T_v=0
        \end{equation*}

    \item
        F\"ur PI, 20\% \"Uberschwingen:

        \begin{equation*}
            K_p= \frac{0.7 \cdot T_g}{K_s \cdot T_u}
        \end{equation*}

        \begin{equation*}
            T_n=3 \cdot T_u
        \end{equation*}

        \begin{equation*}
            T_v=0
        \end{equation*}
    \item
        F\"ur PID, ohne \"Uberschwingen:

        \begin{equation*}
            K_p = \frac{0.9 \cdot Tg}{Ks \cdot Tu}
        \end{equation*}

        \begin{equation*}
            T_n=2.4 \cdot T_u
        \end{equation*}

        \begin{equation*}
            T_v=0.42 cdot T_u;
        \end{equation*}
    \item
        F\"ur PID, 20\% \"Uberschwingen:

        \begin{equation*}
            K_p = \frac{1.2 \cdot T_g}{K_s \cdot T_u}
        \end{equation*}

        \begin{equation*}
            T_n=2 \cdot T_u
        \end{equation*}

        \begin{equation*}
            T_v=0.42 \cdot T_u;
        \end{equation*}
\end{enumerate}

\subsubsection*{Matlab-Code}
\lstinputlisting[style=Matlab-editor-2]{mfiles/p2Ffchien.m}


\clearpage
% ---------------------------------------------------------------------------- %
\subsection{diskDiff}
% ---------------------------------------------------------------------------- %

\code{diskDiff} berechnet  die Steigung einer Funktion  (repr\"asentiert durch
zwei Arrays) an einem bestimmten Array-Index.

\subsubsection*{Input}

\begin{tabular}{p{40mm}l}
    x-Array        & Array mit x-Werten \\
    y-Array        & Array mit zugeh\"origen Funktionswerten \\
    Index          & Index, an dem die Steigung berechnet werden soll
\end{tabular}

\subsubsection*{Output}
\begin{tabular}{p{40mm}l}
    Steigung   & Steigung an gesuchter Stelle
\end{tabular}

\subsubsection*{Algorithmus}
\begin{enumerate}
    \item
        Pr\"ufen, ob Index innerhalb des Arrays liegt.
    \item
        Steigungsdreieck  zwischen  Element  and  Index  und  den  unmittelbar
        daneben liegenden Array-Elementen bilden.
    \item
        Durchschnitt der beiden Steigungsdreiecke ausrechnen.
    \item
        Falls Steigung an erster Array-Stelle verlangt ist: Steigungsdreieck
        mit dem zweiten Element bilden und Steigung zur\"uckgeben.
    \item
        Falls Steigung an letzter Array-Stelle verlangt ist: Steigungsdreieck
        mit zweitletztem Array-Element bilden und Steigung zur\"uckgeben.
\end{enumerate}

\subsubsection*{Java-Code}
\lstinputlisting[style=java]{java/diskDiff.java}


\clearpage
% ---------------------------------------------------------------------------- %
\subsection{schrittIfft}
% ---------------------------------------------------------------------------- %

\code{schrittIfft}   berechnet  die   Schrittantwort   im  Zeitbereich   einer
\"Ubertragungsfunktion. \todo{korrekt?}

\subsubsection*{Input}

\begin{tabular}{p{40mm}l}
    Z\"ahler & Z\"ahler der \"Ubertragungsfunktion                             \\
    Nenner   & Nenner der \"Ubertragungsfunktion                               \\
    Frequenz & Frequenz, bis zu der die Frequenzachse ausgewertet werden soll. \\
    n        & Granulierung der Frequenzachse
\end{tabular}

\subsubsection*{Output}
\begin{tabular}{p{40mm}l}
    Resultat & \parbox[t][4em][s]{0.7\textwidth}{Zweidimensionales Array mit Zeitachse und zugeh\"origen Funktionswerten}
\end{tabular}

\subsubsection*{Algorithmus}
\begin{enumerate}
    \item
        Array mit f\"ur Frequenzachse generieren.
    \item
        Frequenzgang der \"Ubertragungsfunktion berechnen.
    \item
        Impulsantwort im Frequenzbereich berechnen.
    \item
        In den Zeitbereich zur\"ucktransformieren.
    \item
        Aus   den   Realteilen    des   Resultat-Arrays   die   Schrittantwort
        zusammensetzen    (aufsummieren    des   Realteils    des    aktuellen
        Array-Elements    mit    den     Realteilen    aller    vorhergehenden
        Array-Elementen).
\end{enumerate}

\subsubsection*{Java-Code}
\lstinputlisting[style=java]{java/schrittIfft.java}


\clearpage
% ---------------------------------------------------------------------------- %
\subsection{overShootOptimisation}
% ---------------------------------------------------------------------------- %

\code{overShootOptimisation}   optimiert    das   \"Uberschwingverhalten   des
generierten Reglers.

\subsubsection*{Input}

\begin{tabular}{p{40mm}l}
    Keine Eingabewerte &
\end{tabular}

\subsubsection*{Output}
\begin{tabular}{p{40mm}l}
    Keine R\"uckgabewerte &
\end{tabular}

\subsubsection*{Algorithmus}
\begin{enumerate}
    \item
        Das Maximum in der Schrittantwort des geschlossenen Regelkreises finden.
    \item
        Diesen Wert mit dem vom Benutzer gew\"unschten maximalen \"Uberschwingen vergleichen.
    \item
        Falls   zu   starkes   \"Uberschwingen: Reglerverst\"arkung   $K_{rk}$
        schrittweise reduzieren, bis gew\"unschtes Verhalten eingehalten wird.
    \item
        Falls   zu  schwaches   \"Uberschwingen: Reglerverst\"arkung  $K_{rk}$
        schrittweise erh\"ohen, bis gew\"unschtes Verhalten eingehalten wird.
\end{enumerate}

\subsubsection*{Java-Code}
\lstinputlisting[style=java]{java/overshootoptimisation.java}


\clearpage
% ---------------------------------------------------------------------------- %
\subsection{Berechnung von $T_{vk}$ und $T_{nk}$}
% ---------------------------------------------------------------------------- %
\label{app:algo:tnktvk}

\code{calculateTnkTvk}    bestimmt    $T_{nk}$    und    $T_{vk}$. Kern    des
Algorithmus  ist  die  Automatisierung   der  Berechnung  von  $\beta$  (siehe
Anhang~\ref{app:beta} f\"ur die manuelle Rechnung);

\subsubsection*{Input}

\begin{tabular}{p{40mm}l}
    Keine Eingabewerte &
\end{tabular}

\subsubsection*{Output}
\begin{tabular}{p{40mm}l}
    Keine R\"uckgabewerte &
\end{tabular}

\subsubsection*{Algorithmus}
\begin{enumerate}
        \item
            Bestimmen der Steigung der Strecke bei Frequenz $\omega_{pid}$
        \item
            Unteren  Wert   f\"ur  $\beta$   festlegen: $10^{-12}$. Wert  muss
            gr\"sser  als null,  aber  sehr  klein sein,  damit  der Rest  des
            Algorithmus funktioniert.
        \item
            Obere Grenze f\"ur $\beta$ auf 1 legen.
        \item
            \"Uber 20 Iterationen folgende Arbeitsschritte ausf\"uhren:
            \begin{enumerate}
                \item
                    Aktuellen Testwert f\"ur $\beta$ definieren: $\beta_{oben}
                    - \beta_{unten} \cdot 0.5 + \beta_{unten}$.
                \item
                    Mit diesem  Wert f\"ur  $\beta$ nun $T_{nk}$  und $T_{vk}$
                    berechnen gem\"ass Gleichung~\ref{eq:pid:beta:start}.
                \item
                    $T_{nk}$  und  $T_{vk}$   in  die  Reglergleichung  (siehe
                    Gleichung~\ref{eq:pid:t_nk_t_vk_initial_results}) einsetzen.
                    Man  beachte,   dass  der  Algorithmus   nicht  symbolisch
                    rechnet, sondern mit  Werte-Arrays f\"ur die Frequenzachse
                    und Funktionswerte der \"Ubertragungsfunktion.
                \item
                    Steigung    des   Phasengangs    des   Reglers    an   der
                    Stelle  $\omega_{pid}$  bestimmen,  aufsummieren  mit  der
                    Steigung  der  Strecke   (bereits  bekannt)  zur  Steigung
                    des   Phasengangs   des    offenen   Regelkreises   (siehe
                    Gleichung~\ref{eq:pid:phi_sum}).
                \item
                    Falls   die    Steigung   $\varphi_o$   an    der   Stelle
                    $\omega_{pid}$  kleiner   ist  als   $-\frac{1}{2}$,  muss
                    $\beta$ vergr\"ossert  werden.  In diesem Fall  die untere
                    Grenze $\beta_{unten}$ auf  den aktuellen Testwert $\beta$
                    setzen.
                \item
                    Falls   die    Steigung   $\varphi_o$   an    der   Stelle
                    $\omega_{pid}$  gr\"osser  ist  als  $-\frac{1}{2}$,  muss
                    $\beta$   verkleinert    werden. Daher   neue   Obergrenze
                    $\beta_{oben}$ auf den aktuellen Testwert $\beta$ setzen.
            \end{enumerate}
\end{enumerate}

\subsubsection*{Java-Code}
\lstinputlisting[style=java]{java/calculateTnkTvk.java}

