Beim   Programmstart   werden   durch  das   Model   drei   \emph{closedLoops}
(geschlossene  Regelkreise) f\"ur  die Phasengang-Methode  sowie vier  weitere
f\"ur die  Faustformeln erzeugt. Jeder \emph{closedLoop} weiss  nun von Beginn
an, welcher  Berechnungstyp (Phasengang-Methode,  Faustformel) er  ist. Er ist
nun bereit, Daten aufzunehmen und zu verarbeiten.

\"Uber  die  drei  Eingabefelder  f\"ur  $K_s$, $T_u$  und  $T_g$  werden  die
Werte  der  vermessenen  Regelstrecke  durch  den  Benutzer  eingegeben. Durch
Dr\"ucken  des   Buttons  \emph{Berechnen}  werden  die   Eingaben  durch  den
\code{GUIController}  auf  Zul\"assigkeit  \"uberpr\"uft. Erf\"ullen  sie  die
erforderlichen Kriterien nicht, wird eine Benachrichtigung mit Hinweis auf den
Fehler oberhalb des Buttons ausgegeben und die Berechnung nicht ausgel\"ost.

Haben  die Eingaben  die \"Uberpr\"ufung  durch den  GUI-Controller bestanden,
fragt   dieser   zus\"atzlich   die   aktuellen  Werte/   der   Slider   f\"ur
\"Uberschwingen  und  Optimierung sowie  den  Reglertyp  auf  dem GUI  ab  und
leitet  alle  Daten  mittels  \code{setData()}  an  das  Model  weiter. Dieses
erzeugt einen \code{Path} (Strecke) aus den Eingabewerten. Das Model errechnet
den  richtigen  Optimierungs-Offset und  weist  die  Daten den  entsprechenden
\code{closedLoops} der  Phasengang-Methode sowie der Faustformeln  mittels der
\code{setData()}-Methode zu. Jeder  \code{closedLoop} leitet die Daten  an den
zugeh\"origen Controller weiter, der die Reglerwerte berechnet.
