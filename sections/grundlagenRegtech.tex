\subsection{Regelstrecke}
In  der  Regelungstechnik  wird  die  zu  regelnde  Strecke  als  Regelstrecke
bezeichnet. Die  Regelstrecke wird  durch  ihr Zeitverhalten  charakterisiert,
welches den Aufwand und die G\"ute der Regelung bestimmt. Um das Zeitverhalten
zu  beschreiben,  verwendet  man  die Sprungantwort,  welche  zeigt,  wie  die
Regelgr\"osse  auf  Stellgr\"ossen\"anderung  reagiert. Mit  der  entstehenden
Regelgr\"osse werden verschiedene Regelstrecken unterschieden:

\begin{itemize}
 \item  P-Regelstrecke
 \item I-Regelstrecke
 \item    Strecken mit einer Totzeit
 \item Strecken mit Energiespeicher
\end{itemize}

Dieses  Projekt   besch\"aftigt  sich   mit  den  PTn-Strecken,   welche  eine
Kombination einer Strecke  mit proportionalen Verhalten und  einer mit Totzeit
sowie der Angabe der Ordnung n der Strecke sind.

\subsubsection*{P-Regelstrecke}
Bei  der  Regelstrecke mit  proportionalem  Verhalten  folgt die  Regelstrecke
proportional der  Stellgr\"osse ohne  Verz\"ogerung. Dies kommt in  der Praxis
nicht vor,  da immer eine  Verz\"ogerung vorhanden ist. Ist diese  jedoch sehr
klein spricht man von  einer P-Strecke. Der Proportionalit\"atsfaktor wird mit
$K_p$ abgek\"urzt. Wird  $K_p<1$ wirkt $K_p$ nicht  mehr verst\"arkend sondern
abschw\"achend.
\todo{Bild Blockschaltbild P-Strecke}

\subsubsection*{Strecken mit Totzeit}
\"Andert sich die Stellgr\"osse, wirkt sich diese \"Anderung bei einer Strecke
mit Totzeit erst nach einer gewissen Zeit auf die Regelgr\"osse aus. Mit $T_t$
wird das Mass der Totzeit gekennzeichnet.

Totzeiten     verursachen     schnell     Schwingungen,    da     sich     die
Stellgr\"osse\"anderung  zeitverz\"ogert auf  die Regelgr\"osse  auswirkt. Die
Schwingungen  entstehen  wenn sich  die  Stellgr\"osse  und die  Regelgr\"osse
periodisch \"andern.

\todo{Bild Blockschaltbild Totzeit}

\subsubsection*{I-Regelstrecke}
Die  I-Regelstrecke  antwortet  auf eine  Stellgr\"ossen\"anderung  mit  einer
fortw\"ahrenden   \"Anderung   in   steigende  oder   fallende   Richtung. Die
Begrenzung dieses  Vorganges ist mit den  systemgegeben Schranken gegeben. Die
Integrierzeit  $T_i$  ist  ein  Mass  f\"ur  die  Anstiegsgeschwindigkeit  der
Regelgr\"osse.

\todo{Bild Blockschaltbild I-Strecke}

\subsection{Regler}
Die  Aufgabe  eines  Reglers  besteht   die  zu  regelnde  Strecke  mit  einem
Stellsignal  so  zu  beeinflussen,  dass der  Wert  der  Regelgr\"osse  gleich
dem  Wert  der F\"uhrungsgr\"osse  entspricht. Der  Regler  besteht aus  einem
Vergleichsglied,  welches  die  Reglerdifferenz  aus  der  Differenz  zwischen
F\"uhrungs-  und Reglergr\"osse  bildet und  dem Reglerglied. Das  Reglerglied
erzeugt aus der Reglerdifferenz die Stellgr\"osse.

Es wird zwischen P-, I- und  D-Regler unterschieden.  In diesem Projekt werden
die PI- und  PID-Regler, welche Kombinationen der oben  genannten Regler sind,
behandelt.

\subsubsection*{PI-Regler}
Der  PI-Regler  besteht  aus  einer   Parallelschaltung  eines  P-  und  eines
I-Reglers. Durch  diese   Kombination  werden  die  Nachteile   beider  Regler
kompensiert und die Vorteile (schnell, stabil) hervorgehoben.
\todo{Bild Blockschaltbild PI-Regler}

\subsubsection*{PID-Regler}
Wird   dem  PI-Regler   ein   D-Anteil  parallel   geschaltet,  entsteht   der
PID-Regler. Der  PID-Regler ist  ein  sehr oft  verwendeter  Regler, da  durch
den  D-Anteil  die  Regelgr\"osse  rascher   den  Sollwert  erreicht  und  der
Einschwingvorgang  schneller abgeschlossen  ist. Der  PID-Regler ist  geeignet
f\"ur Strecken h\"oheren Ordung, welche m\"oglichst schnell und ohne bleibende
Regelabweichung geregelt werden m\"ussen.
\todo{Bild Blockschaltbild PID-Regler}

\subsection{Die Steuerung}
\todo{bild offener Regelkreis}
Unter einer Steuerung  versteht man eine offen Wirkungskette  wie in Abbildung
XX \todo{Referenz Bild  offener Regelkreis}, dass heisst  die Wirkglieder sind
ketten\"ahnlich aufgereiht und  besitzen keine R\"uckkopplung. Die Steuerkette
wird  genau  f\"ur  eine  Steuerung  ausgelegt und  kann  nur  einer  Art  von
St\"orgr\"osse entgegenwirken. Ohne die  R\"uckkopplung wird das Ausgangsignal
nicht  mit dem  Eingangssignal verglichen  und es  k\"onnen keine  Korrekturen
vorgenommen werden.

\subsection{Der geschlossene Regelkreis}
Die      Aufgabe     eines      geschlossenen     Regelkreises      (Abbildung
\ref{fig:geschlossenerRegelkreis})  ist  es,   einen  vorgegeben  Sollwert  zu
erreichen und diesen  auch bei St\"orungen aufrecht  zu erhalten. Dabei sollen
die unten  genannten dynamischen  Anforderungen eingehalten werden,  damit die
Stabilit\"at des  Regelsystems garantiert ist. Die wichtigste  Bedingung f\"ur
die Schrittantwort ein geschlossenen  Regelkreis heisst, dass der Regelfehler,
die Differenz zwischen  Ist- und Sollwert, gleich Null  oder m\"oglichst klein
ist.


\begin{figure}[!h!, width=\pagewidth]
    \centering
    \includegraphics[width=0.7\textwidth]{images/geschlRegelkreis}
    \caption{Geschlossener Regelkreis}
    \label{fig:geschlossenerRegelkreis}
\end{figure}

%Name Bild Struktur eines allgemeinen Regelkreises
\begin{itemize}
    \item
        $y_soll$ bezeichnet den Sollwert der Regelgr\"osse.
    \item
        $e$ Regelabweichung (Regelfehler)
    \item
        $u$ Steuergr\"osse
    \item
        $x$ Stellgr\"osse
    \item
        $y$ Regelgr\"osse
    \item
        $z$ St\"orgr\"ossen werden in diesem Projekt nicht ber\"ucksichtigt
    \item
        $y_ist$  ist der  Ist-Wert der  Regelgr\"osse  und wird  auch als  die
        Schrittantwort des Regelkreises bezeichnet.
\end{itemize}


Grunds\"atzlich  k\"onnen  f\"unf   Anforderungen  f\"ur  einen  geschlossenen
Regelkreis und deren Schrittantworten zusammengefasst werden:

\begin{enumerate}
    \item
        Der Regelkreis muss stabil sein:  F\"ur das Regelsystem heisst stabil,
        dass es in seinen Gleichgewichtszustand zur\"uckgef\"uhrt werden kann.
    \item
        Der Regelkreis muss gen\"ugend ged\"ampft sein.
    \item
        Der   Regelkreis   muss   eine  bestimmte   station\"are   Genauigkeit
        aufweisen: Das bedeutet, der Regelfehler e(t)  soll f\"ur t-> oo gegen\todo{math mode}
        Null gehen.
    \item
        Der  Regelkreis muss  hinreichend  schnell sein:   Ist die  D\"ampfung
        zu  stark   oder  zu  schwach,  braucht   der  Einschwingvorgang  mehr
        Zeit. Hierbei  muss  darauf  geachtet werden,  dass  die  spezifischen
        Anforderungen an das Regelsystem eingehalten werden.
    \item
        Der  Regelkreis muss  robust  sein: Der Regelkreis  muss so  ausgelegt
        werden,  dass  das  Regelsystem  auch im  schlimmsten  Fall  (je  nach
        Regelsystem situationsabh\"angig) in der Lage ist, das System zur\"uck
        in den stabilen Zustand (vgl. Punkt 1) zu regeln.
\end{enumerate}


\subsubsection*{Die Schrittantwort des geschlossenen Regelkreises}

\todo{Bild Schrittantworten passend zu Aufz\"ahlung unten}

Als Schrittantwort  eines geschlossenen  Regelkreises wird  das Ausgangssignal
$y(t)$ bezeichnet. Im Zusammenhang mit  den Anforderungen an den geschlossenen
Regelkreis, werden an die Schrittantwort folgende Forderungen gestellt:

\begin{enumerate}
    \item
        Die Schrittantwort eines stabilen Regelkreises darf nach dem Erreichen
        des eingeschwungenen Zustand kein erneutes \"Uberschwingen auftreten.
    \item
        Die  D\"ampfung  der  Schrittantwort  soll so  stark  sein,  dass  der
        eingeschwungene Zustand m\"oglichst rasch  erreicht wird ohne dass das
        \"Uberschwingen des Systems zu stark wird.
    \item
        Die Schrittantwort muss f\"ur ein t->oo gleich $y_soll$ sein.
    \item
        Die Schnelligkeit des Einschwingvorganges der Schrittantwort ist stark
        von der  D\"ampfung abh\"angig. Wenn  diese zu  stark oder  zu schwach
        ist, ist der Regelkreis zu langsam.
\end{enumerate}
