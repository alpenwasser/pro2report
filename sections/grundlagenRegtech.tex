In diesem Kapitel wird auf die Grundlagen der Regelungstechnik eingegangen und
die wichtigsten Grundbegriffe erl\"autert.

% ---------------------------------------------------------------------------- %
\subsection{Die Steuerung}
% ---------------------------------------------------------------------------- %

Unter  einer   Steuerung  versteht  man   eine  offene Wirkungskette   wie  in
Abbildung~\ref{fig:Steuerung}, das heisst die Wirkglieder sind ketten\"ahnlich
aufgereiht und besitzen keine R\"uckkopplung. Die Steuerkette wird genau f\"ur
eine Steuerung ausgelegt und kann  nur auf Steuergr\"ossen reagieren. Ohne die
R\"uckkopplung wird das Ausgangsignal  nicht mit dem Eingangssignal verglichen
und es k\"onnen Abweichungen korrigiert werden.

\begin{figure}[!h!, width=\pagewidth]
    \centering
    \includegraphics[width=0.5\textwidth]{images/Steuerung}
    \caption{Steuerung}
    \label{fig:Steuerung}
\end{figure}


% ---------------------------------------------------------------------------- %
\subsection{Der geschlossene Regelkreis}
\label{subs:grundl:geschlossenerRegelkreis}
% ---------------------------------------------------------------------------- %
Die          Aufgabe         eines          geschlossenen         Regelkreises
(Abbildung~\ref{fig:geschlossenerRegelkreis})   ist   es,   einen   vorgegeben
Sollwert  zu   erreichen  und   diesen  auch   bei  St\"orungen   aufrecht  zu
erhalten. Dabei   sollen  die   unten   genannten  dynamischen   Anforderungen
eingehalten  werden,  damit  die   Stabilit\"at  des  Regelsystems  garantiert
ist. Daraus  folgt  auch die  wichtigste  Bedingung  f\"ur die  Schrittantwort
eines geschlossenen  Regelkreises: Der Regelfehler,  das heisst  die Differenz
zwischen Ist- und Sollwert, soll m\"oglichst schnell ausgeglichen werden.  Die
einzelnen Elemente werden in der Aufz\"ahlung weiter unten erkl\"art.


\begin{figure}[!h!, width=\pagewidth]
    \centering
    \includegraphics[width=0.5\textwidth]{images/geschlRegelkreis}
    \caption{Geschlossener Regelkreis}
    \label{fig:geschlossenerRegelkreis}
\end{figure}

%Name Bild Struktur eines allgemeinen Regelkreises
\begin{itemize}
    \item
        $y_{soll}$: Sollwert der Regelgr\"osse
    \item
        $e$: Regelabweichung (Regelfehler)
    \item
        $u$: Steuergr\"osse
    \item
        $x$: Stellgr\"osse
    \item
        $y$: Regelgr\"osse
    \item
        $z$: St\"orgr\"ossen (werden in diesem Projekt nicht ber\"ucksichtigt)
    \item
        $y_{ist}$: Ist-Wert  der Regelgr\"osse (wird auch  als die
        Schrittantwort des Regelkreises bezeichnet)
\end{itemize}


Grunds\"atzlich  k\"onnen  f\"unf   Anforderungen  f\"ur  einen  geschlossenen
Regelkreis und seine Schrittantwort zusammengefasst werden:
\begin{enumerate}
    \item
        Der Regelkreis muss stabil  sein: F\"ur das Regelsystem heisst stabil,
        dass es in seinen Gleichgewichtszustand zur\"uckgef\"uhrt werden kann.
    \item
        Der Regelkreis muss gen\"ugend ged\"ampft sein.
    \item
        Der   Regelkreis   muss   eine  bestimmte   station\"are   Genauigkeit
        aufweisen: Das   bedeutet,   der   Regelfehler   $e(t)$   soll   f\"ur
        $t\rightarrow\infty$ gegen null gehen.
    \item
        Der  Regelkreis  muss  hinreichend schnell  sein: Ist  die  D\"ampfung
        zu  stark   oder  zu  schwach,  braucht   der  Einschwingvorgang  mehr
        Zeit. Hierbei  muss  darauf  geachtet werden,  dass  die  spezifischen
        Anforderungen an das Regelsystem eingehalten werden.
    \item
        Der  Regelkreis muss  robust  sein: Der Regelkreis  muss so  ausgelegt
        werden,  dass  das  Regelsystem  auch im  schlimmsten  Fall  (je  nach
        Regelsystem situationsabh\"angig) in der Lage ist, das System zur\"uck
        in den stabilen Zustand (vgl. 1.) zu regeln.
\end{enumerate}


% ---------------------------------------------------------------------------- %
\subsubsection*{Die Schrittantwort des geschlossenen Regelkreises}
% ---------------------------------------------------------------------------- %

\begin{figure}[h!, width=\pagewidth]
    \begin{center}
    \includegraphics[width=0.8\textwidth]{images/schrittantworten.png}
    \caption{Schrittantworten verschiedener Charakteristiken}
    \label{fig:stepresponse}
    \end{center}
\end{figure}

Als  Schrittantwort  eines  geschlossenen   Regelkreises  wird  der  zeitliche
Verlauf  des   Ausgangssignals  $y(t)$  bezeichnet,  welches   entsteht,  wenn
$y_{soll}$   zum   Zeitpunkt  $t=0$   von   $0$   auf  $1$   springt. In   der
Abbildung~\ref{fig:stepresponse}  werden  drei  verschiedene  Schrittantworten
gezeigt.   Im   Zusammenhang  mit  den  Anforderungen   an  den  geschlossenen
Regelkreis, werden an die Schrittantwort folgende Forderungen gestellt:

\begin{enumerate}
    \item
        Die Schrittantwort eines stabilen Regelkreises darf nach dem Erreichen
        des eingeschwungenen Zustands kein erneutes Schwingen aufweisen.
    \item
        Die  D\"ampfung  der  Schrittantwort  soll so  stark  sein,  dass  der
        eingeschwungene  Zustand m\"oglichst  rasch erreicht  wird ohne,  dass
        das  \"Uberschwingen des  Systems zu  stark wird. Die  pinke Kurve  in
        Abbildung~\ref{fig:stepresponse} zeigt eine Schrittantwort, welche vor
        dem Erreichen des eingeschwungenen stark Zustands \"uberschwingt.
    \item
        Die   Schrittantwort  muss   f\"ur  ein   $t\rightarrow\infty$  gleich
        $y_{soll}$ sein.
    \item
        Die  Schnelligkeit  des  Einschwingvorganges  der  Schrittantwort  ist
        stark  von  der  D\"ampfung   abh\"angig. Wenn  diese  zu  stark  oder
        zu   schwach   ist,  ist   der   Regelkreis   zu  langsam. Die   gelbe
        Kurve  in  Abbildung~\ref{fig:stepresponse}  zeigt  eine  zu  langsame
        Schrittantwort.
\end{enumerate}

Die  Berechnung   der  Schrittantwort  des  geschlossenen   Regelkreises  wird
in  unserem   Tool  mittels  der  inversen   schnellen  Fourier-Transformation
durchgef\"uhrt, genauere  Informationen dazu sind  in Abschnitt~\ref{subs:fft}
und Anhang~\ref{app:algo:ifft} zu finden.


% ---------------------------------------------------------------------------- %
\subsection{Regelstrecken}
% ---------------------------------------------------------------------------- %

In  der  Regelungstechnik  wird  die  zu  regelnde  Strecke  als  Regelstrecke
bezeichnet. Die  zu   regelnde  Strecke   ist  zum  Beispiel   die  Temperatur
im  Raum  oder  die  Luftfeuchtigkeit  in  der  Sauna. Die  Regelstrecke  wird
durch  ihr   Zeitverhalten  charakterisiert,  welches  den   Aufwand  und  die
G\"ute   der   Regelung   bestimmt. Um  das   Zeitverhalten   zu   beschreiben
verwendet  man die  Schrittantwort,  welche zeigt,  wie  die Regelgr\"osse auf
Stellgr\"ossen\"anderung reagiert. Mit  der entstehenden  Regelgr\"osse werden
verschiedene Regelstrecken unterschieden:

\begin{itemize}
    \item
        P-Regelstrecke
    \item
        I-Regelstrecke
    \item
        Strecken mit einer Totzeit
    \item
        Strecken mit Energiespeicher
\end{itemize}

Dieses  Projekt  besch\"aftigt  sich   mit  den  PT$n$-Strecken,  welche  eine
Kombination  aus einer  Strecke  mit proportionalem  Verhalten  und einer  mit
Totzeit ist. Die Ordnung der Strecke wird durch die Variable $n$ angegeben.


% ---------------------------------------------------------------------------- %
\subsubsection*{P-Regelstrecke}
% ---------------------------------------------------------------------------- %
Bei  der  Regelstrecke mit  proportionalem  Verhalten  folgt die  Regelstrecke
proportional der  Stellgr\"osse ohne  Verz\"ogerung. Dies kommt in  der Praxis
nicht vor,  da immer eine  Verz\"ogerung vorhanden ist. Ist  die Verz\"ogerung
jedoch  sehr  klein  spricht  man   von  einer  P-Strecke. Das  Verhalten  der
Strecke ist in ihrem Blockschaltbild (Abbildung~\ref{fig:PStrecke}) symbolisch
dargestellt. Der  Proportionalit\"atsfaktor wird  mit $K_p$  abgek\"urzt. Wird
$K_p<1$ wirkt $K_p$ nicht mehr verst\"arkend sondern abschw\"achend.

\begin{figure}[h!, width=\pagewidth]
    \centering
    \includegraphics[width=0.2\textwidth]{images/PStrecke}
    \caption{Blockschaltbild von P-Strecke}
    \label{fig:PStrecke}
\end{figure}


% ---------------------------------------------------------------------------- %
\subsubsection*{Strecken mit Totzeit}
% ---------------------------------------------------------------------------- %
\"Andert  sich  die  Stellgr\"osse,  wirkt sich  diese  \"Anderung  bei  einer
Strecke  mit Totzeit  erst  nach  einer gewissen  Zeit  auf die  Regelgr\"osse
aus. Mit $T_t$  wird das  Mass der Totzeit  gekennzeichnet. Im Blockschaltbild
(Abbildung~\ref{fig:TotZeit})  wird die  Totzeit durch  eine Verz\"ogerung  am
Anfang dargestellt.

Totzeiten     verursachen    schnelle     Schwingungen,     da    sich     die
Stellgr\"ossen\"anderung zeitverz\"ogert auf die Regelgr\"osse auswirkt. Diese
Schwingungen  entstehen  wenn sich  die  Stellgr\"osse  und die  Regelgr\"osse
periodisch \"andern.

\begin{figure}[h!, width=\pagewidth]
    \centering
    \includegraphics[width=0.2\textwidth]{images/Totzeit}
    \caption{Blockschaltbild von Strecke mit Totzeit}
    \label{fig:TotZeit}
\end{figure}


% ---------------------------------------------------------------------------- %
\subsubsection*{I-Regelstrecke}
% ---------------------------------------------------------------------------- %

Die  I-Regelstrecke  antwortet  auf eine  Stellgr\"ossen\"anderung  mit  einer
stetigen  \"Anderung in  steigender  oder  fallender Richtung. Die  Begrenzung
dieses   Vorganges  ist   mit  den   systembedingten  Schranken   gegeben. Die
Integrierzeit  $T_i$  ist  ein  Mass  f\"ur  die  Anstiegsgeschwindigkeit  der
Regelgr\"osse und das Blockschaltbild (Abbildung~\ref{fig:IStrecke}) zeigt das
Verhalten sinnbildlich.

\begin{figure}[h!, width=\pagewidth]
    \centering
    \includegraphics[width=0.2\textwidth]{images/IStrecke}
    \caption{Blockschaltbild von I-Strecke}
    \label{fig:IStrecke}
\end{figure}


% ---------------------------------------------------------------------------- %
\subsection{Regler}
% ---------------------------------------------------------------------------- %

Die Aufgabe  eines Reglers  besteht darin  die zu  regelnde Strecke  mit einem
Stellsignal  so  zu  beeinflussen,  dass der  Wert  der  Regelgr\"osse  gleich
dem  Wert  der F\"uhrungsgr\"osse  entspricht. Der  Regler  besteht aus  einem
Vergleichsglied,  welches  die  Reglerdifferenz  aus  der  Differenz  zwischen
F\"uhrungs-  und Reglergr\"osse  bildet und  dem Reglerglied. Das  Reglerglied
erzeugt aus der Reglerdifferenz die Stellgr\"osse.

Es wird zwischen P-, I- und D-Regler unterschieden.

In diesem Projekt werden die PI- und PID-Regler, welche Kombinationen der oben
genannten Regler sind, behandelt.

% ---------------------------------------------------------------------------- %
\subsubsection*{P-Regler}
% ---------------------------------------------------------------------------- %
Der  P-Regler regelt  die  Stellgr\"osse $x$  proportional zur  Regeldifferenz
$e$. Aus  diesem  Grund reagiert  der  P-Regler  ohne Verz\"ogerung  auf  eine
Ausgangs\"anderung. Das  Proportionale Verhalten  ist in  seinem Blockdiagramm
(Abbildung~\ref{fig:PRegler}) dargestellt.

\begin{figure}[h!, width=\pagewidth]
    \centering
    \includegraphics[width=0.4\textwidth]{images/PRegler}
    \caption{Blockschaltbild des P-Reglers}
    \label{fig:PRegler}
\end{figure}

% ---------------------------------------------------------------------------- %
\subsubsection*{I-Regler}
% ---------------------------------------------------------------------------- %
Der  I-Regler ist  ein integrierender  Regler, welcher  Regelabweichungunen zu
jedem  Betriebspunkt ausregelt. Die  Steigung  des I-Reglers  h\"angt von  der
Reglerabweichung und  der Integrationszeit  $T_i$ ab. Hier gilt  die Regel: Je
gr\"osser  die  Integrationszeit ist,  umso  langsamer  steigt die  Kurve. Das
Verhalten des I-Reglers ist  im Blockschaltbild in Abbildung~\ref{fig:IRegler}
ersichtlich.

\begin{figure}[h!, width=\pagewidth]
    \centering
    \includegraphics[width=0.4\textwidth]{images/IRegler}
    \caption{Blockschaltbild des I-Reglers}
    \label{fig:IRegler}
\end{figure}

% ---------------------------------------------------------------------------- %
\subsubsection*{D-Regler}
% ---------------------------------------------------------------------------- %
Der Differentialregler bildet seine  Stellgr\"osse $x$ aus der Geschwindigkeit
der \"Anderung der Regeldifferenz. Aus diesem  Grund erzeugt der D-Regler auch
bei kleiner Reglerdifferenz  f\"ir eine kurze Zeit  eine grosse Stellamplitude
und  reagiert dadurch  schneller als  der P-Regler. Dieses  Verhalten wird  in
seinem Blockschaltbild (Abbildung~\ref{fig:DRegler}) symbolisch dargestellt.

\begin{figure}[h!, width=\pagewidth]
    \centering
    \includegraphics[width=0.4\textwidth]{images/DRegler}
    \caption{Blockschaltbild des D-Reglers}
    \label{fig:DRegler}
\end{figure}

% ---------------------------------------------------------------------------- %
\subsubsection*{PI-Regler}
% ---------------------------------------------------------------------------- %
Der  PI-Regler  besteht  aus  einer   Parallelschaltung  eines  P-  und  eines
I-Reglers  (Abbildung~\ref{fig:PIRegler1}). Durch   diese  Kombination  werden
die   Nachteile  beider   Regler  kompensiert   und  die   Vorteile  (schnell,
stabil)  hervorgehoben. Sein Verhalten  wird  bildlich  im Blockschaltbild  in
Abbildung~\ref{fig:PIRegler2} dargestellt.

\begin{figure}[h!, width=\pagewidth]
    \centering
    \includegraphics[width=0.4\textwidth]{images/PIRegler1}
    \caption{Parallelschlatung von P-Regler und I-Regler}
    \label{fig:PIRegler1}
\end{figure}

\begin{figure}[h!, width=\pagewidth]
    \centering
    \includegraphics[width=0.4\textwidth]{images/PIRegler2}
    \caption{Blockschaltbild von PI-Regler}
    \label{fig:PIRegler2}
\end{figure}


% ---------------------------------------------------------------------------- %
\subsubsection*{PID-Regler}
% ---------------------------------------------------------------------------- %

Wird      dem     PI-Regler      ein     D-Anteil      parallel     geschaltet
(Abbildung~\ref{fig:PRDRegler1}),  entsteht   der  PID-Regler. Der  PID-Regler
ist   ein  sehr   oft  verwendeter   Regler,   da  durch   den  D-Anteil   die
Regelgr\"osse  rascher   den  Sollwert  erreicht  und   der  Einschwingvorgang
schneller  abgeschlossen  ist. Das   Blockschaltbild  zeigt  dieses  Verhalten
(Abbildung~\ref{fig:PID})  anschaulich. Der  PID-Regler   ist  geeignet  f\"ur
Regelstrecken h\"oherer Ordnung, welche m\"oglichst schnell und ohne bleibende
Regelabweichungen geregelt werden m\"ussen.

\begin{figure}[h!, width=\pagewidth]
    \centering
    \includegraphics[width=0.4\textwidth]{images/PRDRegler1}
    \caption{Parallelschaltung von P-, I-, und D-Regler}
    \label{fig:PRDRegler1}
\end{figure}

\begin{figure}[h!, width=\pagewidth]
    \centering
    \includegraphics[width=0.4\textwidth]{images/PIDRegler2.png}
    \caption{Blockschaltbild des PID-Reglers}
    \label{fig:PID}
\end{figure}
