Die    Formeln   in    Tabelle   \ref{tab:bode_regler_konform}    dienen   zur
Umrechnung  zwischen der  bodekonformen  Darstellung  und der  reglerkonformen
Darstellung. N\"ahere  Informationen  zu den  verschiedenen  Darstellungsarten
k\"onnen der Quelle \cite{regelungstechnik:zellweger} entnommen werden.
\todo{Allenfalls noch  ein paar kurze  S\"atze zum Sinn  dieser \"Ubung? Sonst
wird nirgends  darauf wirklich  Bezug genommen, Abschnitt  ist ein  wenig ohne
Kontext in der Landschaft.}

\begin{longtable}{l|ll}
    \toprule

    %\multicolumn{3}{l}{\large{\textsc{auftragsanalyse und hintergrundinformationen}}} \\
    %\multicolumn{2}{l}{pi}

    &
    bodekonform $\rightarrow$ reglerkonform
    &
    reglerkonform $\rightarrow$ bodekonform
    \\

    \midrule

    \endhead
    \endfoot
    \endlastfoot

    % content here ---------------------------------------------------------- %

    PI
    &
    $T_n = T_{nk} $ %reglerkonform
    &
    $K_{rk} = K_r $ %bodekonform
    \\

    \midrule

    PID
    &
    $T_n = T_{nk}+T_{vk}-T_p$
    &
    $T_{nk}=0.5 \cdot (T_n+T_p) \cdot (1+\epsilon)$
    \\

    &
    $T_v=\frac{T_{nk} \cdot T_{vk}}{T_{nk}+T_{vk}-T_p}-T_p$
    &
    $T_{vk}=0.5 \cdot (T_n+t_p) \cdot (1-\epsilon)$
    \\

    &
    %$k_r=k_{rk} \cdot \frac{1+t_{vk}}{t_{nk}}$
    $K_r=K_{rk} \cdot (1 + \frac{T_{vk}-T_p}{T_{nk}})$
    &
    $K_{rk} = 0.5 \cdot K_r \cdot (1 + \frac{T_p}{T_{nk}}) \cdot (1+\epsilon )$
    \\
    \\

    &
    \multicolumn{2}{l}{wobei $\epsilon^2 = 1-(4 \cdot T_n \cdot \frac{T_v-T_p}{(T_n+T_p)^2})$}
    \\
    \bottomrule
    \caption{Formeln zur Umrechung zwischen bode- zu reglerkonformer Darstellung \cite{regelungstechnik:zellweger}, \cite{regelungstechnik:schumleon}}
    \label{tab:bode_regler_konform}
\end{longtable}


F\"ur die Berechnungen in diesem Projekt wird, wenn nicht anders angegeben, mit $T_p=\frac{1}{10} \cdot T_v$ gerechnet.
