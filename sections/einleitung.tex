% Ausgangslage
Im Rahmen  des Projektes soll  ein Tool  entwickelt werden, welches  einen PI-
respektive  einen  PID-Regler mittels  der  von  Jakob Zellweger  entwickelten
Phasengangmethode dimensioniert. Zum  Vergleich soll der  entsprechende Regler
ebenfalls mittels verschiedener Faustformeln berechnet werden.

% Anforderungen
Die Phasengangmethode ist eine graphische Methode, die bis anhin mit Stift und
Papier  durchgef\"uhrt wurde. Folglich  ist die  Ausf\"uhrung zeitaufw\"andig,
speziell   wenn   Schrittantworten   mit   unterschiedlichen   Parameterwerten
durchgespielt werden sollen, zudem kann das \"Uberschwingen nur grob gew\"ahlt
werden. Das Tool  soll ausgehend  von drei  Parametern aus  der Schrittantwort
der  Strecke  (Verst\"arkung  $K_s$,  Anstiegszeit  $T_g$,  Verz\"ogerungszeit
$T_u$)  mittels   der  Phasengangmethode  m\"oglichst   ideale  Regelparameter
berechnen,  sowie  die  Schrittantwort des  darauf  basierenden  geschlossenen
Regelkreises graphisch darstellen. Die  Benutzeroberfl\"ache der Software soll
intuitiv  sein, sodass  sich  auch  mit dem  Thema  nicht eingehend  vertraute
Regelungstechniker einfach zurechtfinden.

% Umsetzung
Die  erforderlichen  Algorithmen  wurden   zuerst  in  Matlab  als  Prototypen
implementiert   und  anschliessend   vollst\"andig  in   Java  konvertiert. Um
optimale  Wartbarkeit,  \"Ubersichtlichkeit  und Modularit\"at  des  Codes  zu
gew\"ahrleisten,  ist  die  Software gem\"ass  Model  View-Controllern-Pattern
aufgebaut.

% Ergebnisse
Nach  der Implementierung  in Matlab  wurde  klar, dass  die Berechnung  durch
die  hohe Rechenleistung  sehr schnell  durchgef\"uhrt werden  kann und  somit
eine  Dimensionierung des  geschlossenen  Regelkreises  anhand dieser  Methode
von  Zellweger  m\"oglich  ist. Die  Schrittantworten  k\"onnen  zur  Echtzeit
angezeigt werden. Diese  M\"oglichkeit wurde  genutzt indem  zwei Schiebregler
implementiert wurden. Durch einen kann das \"Uberschwingen manuell eingestellt
werden kann und mit dem anderen wird die Kurve optimiert.

% Aufbau des Berichts
Der Bericht gliedert sich in drei Teile: Die ersten zwei Teile erl\"autern die
theoretischen Grundlagen,  der zweite Teil  besch\"aftigt sich mit  dem Aufbau
der Software.
