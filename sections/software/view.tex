Die  \emph{View} ist  aus  zwei \"ubergeordneten  Panels aufgebaut. Im  linken
Panel befinden sich Ein- und  Ausgabefelder f\"ur numerische Werte, das rechte
Panel  beheimtated die  Plots sowie  das Optimierungs-Panel  und kann  mittels
Check-Box "erweiter" ein- und ausgeblendet werden.

\todo{Image Gesamt-GUI}

\todo{Image Panel Schrittantwort vermessen}


Im  Bereich 1  \todo{Image Referenzen}  werden die  Parameter der  vermessenen
Strecke  eingegeben. Darunter  befinden  sich die  Schalftfl\"achen  zur  Wahl
\todo{Image Butttons  PI-, PID-T1-Regler}  zwischen der  Dimensionierung eines
PI- respektive eines PID-T1-Reglers.

Das   Panel    \emph{Reglerwerte}   dient   der   Ausgabe    der   berechneten
Reglerwerte  der   verschiedenen  Berechnungsmethoden. Ebenfalls   kann  f\"ur
die  Phasengang-Methode,   sofern  ein  PID-Regler  dimensioniert   wird,  die
Zeitkonstante  $T_p$  \todo{Check:  korrekter  Begriff}  spezifiziert  werden.
\todo{Image Panel Phasengangmethode}

Das Optimierungs-Panel  beinhaltet zwei  Slider zur Eingabe  des gew\"unschten
\"Uberschwingens   respektive   f\"ur   die  Optimierung   des   Reglers   der
Phasengang-Methode.

Unterhalb des  Optimierungs-Panels werden  die Plots der  mittels Faustformeln
und Phasengangmethode errechneten Schrittantworten graphisch ausgegeben. Diese
k\"onnen   mittels  Check-Boxen   zu-  und   weggeschaltet  werden. Zu   jeder
Faustformel wird die zugeh\"orige Schrittantwort abgebildet. Die Resultate der
Phasengangmethode  werden durch  drei Kurven  dargestellt. Eine Kurve  benutzt
den  Standardwert des  Phasenrands gem\"ass  Zellweger \todo{Einf\"ugen  Wert,
Referenz}, die beiden anderen Kurven basieren auf Benutzereingaben f\"ur einen
oberen  und  unteren  Offset  des Phasenrandes  der  \"uber  den  Schiebregler
\emph{Optimierung} eingestellt werden kann.
