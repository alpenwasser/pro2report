In  der   Regelungstechnik  ist   die  korrekte  Dimensionierung   der  Regler
essenziell. Die  zu  regelnde  Strecke  kann nur  durch  richtig  eingestellte
Regelwerte wie gew\"unscht beeinflusst werden. 
 

Dieses  Projekt  hat sich  zum  Ziel  gesetzt,  eine Software  zu  entwickeln,
welche  aus   den eingegebenen Streckenwerten  $K_s$,  $T_u$  und   $T_g$  PI-
und  PID-Regler  dimensioniert. Das  Tool  berechnet die  Reglerwerte mit  der
sogenannten   Phasengangmethode,  welche   von  Jakob   Zellweger,  ehemaliger
Dozent  an  der  Fachhochschule Nordwestschweiz, stammt. Diese  Methode  wurde
urspr\"unglich  f\"ur die  graphische  Auswertung  des Phasengangs  entwickelt
und  durch Handarbeit mit  Geodreieck und  Bleistift angewendet. Die  Software
erm\"oglicht es, diese bew\"ahrte  Methode effizient durch ein automatisiertes
Verfahren zu verwenden.
 

Das  entwickelte   Softwaretool simuliert  neben  den berechneten Reglerwerten
auch die    Schrittantwort des     geschlossenen    Regelkreises. Zus\"atzlich
wurde      die       Berechnung      dreier       g\"angiger      Faustformeln
implementiert. Zum Vergleich k\"onnen die Schrittantworten der Faustformeln im
gleichen Plot angezeigt werden.


F\"ur die Phasengangmethode werden gleichzeitig drei Graphen abgebildet. Einer
verwendet  den   Standardwert  f\"ur  $\varphi_{start}$   gem\"ass  Zellweger,
die    anderen   beiden    basieren   auf    Benutzereingaben   durch    einen
Schieberegler. Dadurch k\"onnen die Reglerwerte weiter optimiert werden.


Ein  weiterer  Schieberegler  erm\"oglicht  das  Festlegen  des  gew\"unschten
maximalen  \"Uberschwingens. Die Software  optimiert den  zu dimensionierenden
Regler  automatisch  in  Echtzeit,  sodass  der  geschlossene  Regelkreis  das
gew\"unschte   Verhalten   aufweist. Dies  erm\"oglicht   enorme   Zeitgewinne
verglichen mit der manuellen Durchf\"uhrung der Phasengangmethode.
