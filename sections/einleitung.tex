% Ausgangslage
Im Rahmen  des Projektes soll  ein Tool  entwickelt werden, welches  einen PI-
respektive  einen  PID-Regler mittels  der  von  Jakob Zellweger  entwickelten
Phasengangmethode dimensioniert. Zum  Vergleich soll der  entsprechende Regler
ebenfalls mittels verschiedener Faustformeln berechnet werden.

% Anforderungen
Die Phasengangmethode ist eine graphische Methode, die bis anhin mit Stift und
Papier  durchgef\"uhrt wurde. Folglich  ist die  Ausf\"uhrung zeitaufw\"andig,
speziell   wenn   Schrittantworten   mit   unterschiedlichen   Parameterwerten
durchgespielt werden sollen. Zudem kann das \"Uberschwingen nur grob gew\"ahlt
werden. Das Tool soll ausgehend von drei Parametern aus der Schrittantwort der
Strecke  (Verst\"arkung $K_s$,  Verz\"ogerungszeit $T_u$,  Anstiegszeit $T_g$)
mittels  der Phasengangmethode  m\"oglichst  ideale Regelparameter  berechnen,
sowie  die Schrittantwort  des darauf  basierenden geschlossenen  Regelkreises
graphisch  darstellen. Die  Benutzeroberfl\"ache  der Software  soll  intuitiv
sein,  sodass  sich auch  mit  dem  Thema nicht  vertraute  Regelungstechniker
einfach zurechtfinden.

% Umsetzung
Die    erforderlichen    Algorithmen    wurden   zuerst    in    Matlab    als
Prototypen   implementiert    und   anschliessend   vollst\"andig    in   Java
\"ubersetzt. Um  optimale Wartbarkeit,  \"Ubersichtlichkeit und  Modularit\"at
des    Codes     zu    gew\"ahrleisten,    ist    die     Software    gem\"ass
\emph{Model-View-Controller}-Pattern aufgebaut.

% Ergebnisse
Nach  der Implementierung  in Matlab  wurde  klar, dass  die Berechnung  durch
die  hohe Rechenleistung  sehr schnell  durchgef\"uhrt werden  kann und  somit
eine Auswertung  der Schrittantwort  des geschlossenen  Regelkreises m\"oglich
ist.   Diese k\"onnen  in Echtzeit  angezeigt werden. Damit  er\"offneten sich
neue M\"oglichkeiten, was dazu benutzt wurde, zwei Schieberegler zur Anpassung
des  dimensionierten  Reglers  zu   implementieren. Mit  dem  einen  kann  das
gew\"unschte \"Uberschwingen  spezifiziert werden,  der andere  Schieber dient
zur weiteren Anpassung der  Kurvenform des geschlossenen Regelkreises gem\"ass
den W\"unschen des Benutzers.

% Aufbau des Berichts
Der      Bericht      gliedert      sich      in      drei      Hauptteile: In
einem     ersten     Teil     werden    einige     theoretische     Grundlagen
der   Regelungstechnik    erl\"autert   (Kapitel~\ref{sec:grundlagenRegtech}).
Kapitel~\ref{sec:fachlicher_hintergrund}  geht  genauer  auf  die  eigentliche
Dimensionierung  von Reglern  ein, indem  ein PI-  und ein  PID-Regler mittels
Faustformeln  und Phasengangmethode  f\"ur eine  Beispielstrecke dimensioniert
werden.    Kapitel~\ref{sec:software}   und~\ref{sec:test}   beschreiben   die
Software  und  die  zugeh\"origen  Testverfahren;  wichtige  Algorithmen  sind
dabei   in   Anhang~\ref{app:algos}   genauer  dokumentiert. Es   werden   die
Benutzeroberfl\"ache  und  das  Zusammenspiel  der  verschiedenen  Module  der
Software genauer betrachtet.
