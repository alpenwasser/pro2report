Im   Praxiseinsatz  stehen   f\"ur   die  Dimensieung   der  Regler   einfache
Berechnungsformeln f\"ur die Einstellwerte der  Regler anhand von $T_u$, $T_g$
und $K_s$ zur Verf\"ugung.

Einige   dieser  Faustformeln   werden  in   der  Applikation   zum  Vergleich
mitberechnet. Die    dazugeh\"origen    Berechnungen    sind    der    Tabelle
\ref{tab:faustformeln} zu entnehmen.

\begin{longtable}{p{50mm}rrrrr}
    \toprule

    %\multicolumn{3}{l}{\large{\textsc{Auftragsanalyse und Hintergrundinformationen}}} \\

    Faustformel
    &
    \multicolumn{2}{l}{PI-Regler}
    &
    \multicolumn{2}{l}{PID-T1-Regler}
    \\

    &
    $T_n$
    &
    $K_p$
    &
    $T_n$
    &
    $T_v$
    &
    $K_p$
    \\

    \midrule

    \endhead
    \endfoot
    \endlastfoot

    % CONTENT HERE ---------------------------------------------------------- %

    \pbox{45mm}{Chiens, Hrones, Reswick \\ \small{\textbf{(0\% \"Uberschwingen)}} \\ \cite{ref:chiens_tsn}, \cite{ref:chiens_wiki}}
    &
    $1.2\cdot T_g$
    &
    $\frac{0.35}{K_s} \cdot \frac{T_g}{T_u}$
    &
    $T_g$
    &
    $0.5\cdot T_u$
    &
    $ \frac{0.6}{K_s} \cdot \frac{T_g}{T_u} $
    \\

    \addlinespace[1em]

    \pbox{45mm}{Chiens, Hrones, Reswick \small{\textbf{(20\% \"Uberschwingen)}} \\ \cite{ref:chiens_tsn}, \cite{ref:chiens_wiki}}
    &
    $T_g$
    &
    $\frac{0.6}{K_s} \cdot \frac{T_g}{T_u}$
    &
    $1.35\cdot T_g$
    &
    $0.47 \cdot T_u$
    &
    $ \frac{0.95}{K_s} \cdot \frac{T_g}{T_u} $
    \\

    \addlinespace[1em]

    Oppelt \cite{ref:op_ros_zieg}
    &
    $3 \cdot T_u$
    &
    $\frac{0.8}{K_s} \cdot \frac{T_g}{T_u}$
    &
    $2 \cdot T_u$
    &
    $ 0.42 \cdot T_u $
    &
    $ \frac{1.2}{K_s} \cdot \frac{T_g}{T_u} $
    \\

    \addlinespace[1em]

    Rosenberg \cite{ref:op_ros_zieg}
    &
    $3.3 \cdot T_u $
    &
    $ \frac{0.91}{K_s} \cdot \frac{T_g}{T_u} $
    &
    $ 2 \cdot T_u $
    &
    $ 0.45 \cdot T_u $
    &
    $ \frac{1.2}{T_s} \cdot \frac{T_g}{T_u}$
    \\

    \addlinespace[1em]

    Ziegler/Nichols \cite{ref:op_ros_zieg}
    &
    $ 3.33 \cdot T_u $
    &
    $ \frac{0.9}{K_s} \cdot \frac{T_g}{T_u} $
    &
    $ 2 \cdot T_u $
    &
    $ 0.5 \cdot T_u $
    &
    $ \frac{1.2}{K_s} \cdot \frac{T_g}{t_u} $
    \\

    \bottomrule
\caption{Faustformeln zur Reglerdimensionierung}
\label{tab:faustformeln}
\end{longtable}
\todo{Quellen hinzuf\"ugen}
