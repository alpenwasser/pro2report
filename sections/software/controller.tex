Regler  heisst \"ubersetzt  auf Englisch  \emph{Controller}. Aus diesem  Grund
heisst die generische Reglerklasse  in unserer Software \code{Controller}. Die
Klasse,   welche   die   Rolle   des   \emph{Controllers}   im   Kontext   von
\emph{Model-View-Controller} wahrnimmt, heisst  daher \code{GUIController}, um
Namenskonflikte zu vermeiden.


Der \code{GUIController} dient als  Schnittstelle zwischen der \code{View} und
dem  \code{Model}. Werden  Eingaben auf  der  \code{View}  durch den  Benutzer
gemacht, wird  der \code{GUIController}  dar\"uber informiert. Er  pr\"uft die
Daten  auf  Zul\"assigkeit  und   leitet  berechnungsrelevante  Daten,  sofern
sie  die  Pr\"ufung bestanden  haben,  an  das \code{Model}  zur  Verarbeitung
weiter. Zus\"atzlich  \"ubernimmt der  \code{GUIController} die  Steuerung der
\code{View}, indem  er Objekte  ein- und ausblendet  sowie Meldungen/Warnungen
\"uber die \code{View} ausgibt.
