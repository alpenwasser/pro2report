% was laeuft, was laeuft nicht, was kann verbessert werden
Das Projekt  wurde erfolgreich abgeschlossen: Das Tool  ist funktionst\"uchtig
und alle Punkte des urspr\"unglichen Auftrages sowie einige der im Verlauf des
Projekts  optionalen  Erweiterungen  wurden erf\"ullt. Aus  den  Eingabewerten
wird  das dynamische  Verhalten des  geschlossenen Regelkreises  berechnet und
graphisch dargestellt. Die Erweiterungen  erm\"oglichen das Dimensionieren des
Reglers anhand der Phasengangmethode und die Optimierung der Reglerwerte.


Die   im   Pflichtenheft   optional  geplanten,   graphischen   Ausgaben   des
Amplitudengangs und der Strecke sowie  die Schieberegler f\"ur das Ver\"andern
der  Reglerwerte wurden  weggelassen. Diese Zus\"atze  wurden geplant,  um das
Dimensionieren  des  Reglers zu  erleichtern  und  die Eingabewerte  graphisch
darzustellen. Der  Grund f\"ur  den Entscheid,  diese zus\"atzlichen  Features
wegzulassen, war die M\"oglichkeit durch die hohe Rechenleistung den Regler in
Echtzeit  automatisiert zu  optimieren, womit  die Funktionen  \"uberfl\"ussig
wurden.


Alternativ kann  die Schrittantwort mit zwei  Schiebereglern manuell optimiert
werden. Durch  das Ver\"andern  des  \"Uberschwingens und  das Einstellen  der
Optimierung kann die Kurve von  Hand angepasst werden.


Eine m\"oglicher  Bereich, in dem zus\"atzliche  Features implementiert werden
k\"onnten,  ist  die Auswertung  und  Analyse  der Strecke. Es  k\"onnten  die
Kennzahlen  der  \"Ubertragungsfunktion  der Strecke  zur\"uckgegeben  werden,
zusammen mit einer graphischen Darstellung der Strecke.


Einen  bedeutenden   Zusatznutzen  w\"urde  auch  das   automatische  Einlesen
der  Streckenwerte   aus  einer  graphischen  Darstellung   bringen. Das  Tool
w\"urde   an   die   eingelesene   Kurve   selbst\"andig   die   Wendetangente
legen   und  die   zuge\"origen  Kennwerte   auslesen. Das  Ablesen   k\"onnte
mit  der   Monte-Carlo-Studie  verfeinert   werden  (ein  Verfahren   aus  der
Wahrscheinlichkeitsrechung, welches  aus einer endlichen Schar  von Kurven die
numerisch wahrscheinlichste L\"osung findet).


Das  Vergleichen  der  Resultate  von verschiedenen  PI-  und  PID-Reglern  im
gleichen Plot ist  momentan noch nicht m\"oglich, w\"are  sicherlich aber auch
n\"utzlich.


Weiter  w\"are  eine  automatische   Anpassung  der  Werte  f\"ur  $\varphi_s$
denkbar   (siehe  dazu   Gleichungen~\ref{eq:pi:phi_s}  und~\ref{eq:pid:phi_s}
auf   Seite~\pageref{eq:pi:phi_s}    respektive~\pageref{eq:pid:phi_s}   sowie
die   zugeh\"origen   Fussnoten~\footnotemark[4]   und~\footnotemark[7]). Auch
zus\"atzliche Reglertypen k\"onnten noch implementiert werden.
