Die Model-Klassen  beheimaten alle Daten  sowie den Grossteil  der Algorithmen
(zus\"atzliche Algorithmen,  die von den Model-Klassen  verwendet werden, sind
in  der  Klasse  \code{Calc}  des  Package  \code{Utilities}  zu  finden). Das
\code{Model}  ist   observable,  um  das  Aktualisieren   der  \code{View}  zu
erm\"oglichen.

\subsubsection*{\code{Model}}
Wie  im  Klassendiagramm  (siehe  Anhang~\ref{app:classdiagram})  ersichtlich,
erzeugt  die  Klasse \code{Model}  im  Konstruktor  f\"ur jede  Berechnungsart
eine  Instanz der  Klasse \code{ClosedLoop}. Das  \code{Model} hat  ein Objekt
der  Klasse \code{Path}  und \"ubergibt  dieses an  den \code{ClosedLoop}. Das
\code{Model}  enth\"alt  Setter-  und  Getter-Methoden  zur  Verarbeitung  der
Daten  mittels  weiterer  Klassen. Das \code{Model}  nutzt  die  M\"oglichkeit
\code{notifyObservers()} aufzurufen, um \code{update()} auf der \code{View} zu
bewirken.


\subsubsection*{\code{ClosedLoop}}
Die    Instanzen   der    Klasse   \code{ClosedLoop}    repr\"asentieren   die
geschlossenen  Regelkreise. Sie  kennen   ihren  Berechnungstyp  und  besitzen
einen  Regler. Im  Konstruktor  wird  je  nach  Berechnungsart  ein  passender
\code{Controller}   instanziiert. Zus\"atzlich  beinhaltet   die  Klasse   die
Methode  \code{calculate()}. Diese   berechnet  f\"ur   alle  Berechnungsarten
die    Schrittantwort    mittels   \code{calculateStepResponse()}.

F\"ur  die  Phasengangmethode  wird  zus\"atzlich  das  \"Uberschwingverhalten
mittels   der   Methode    \code{overShootOptimization()}   optimiert   (siehe
Anhang~\ref{app:algo:oversh}).  Zudem sind diverse Setter- und Getter-Methoden
Bestandteil dieser Model-Klasse.


\subsubsection*{\code{Controller}}
Der  \code{Controller}  bildet  die  Oberklasse  aller  Faustformeln  und  der
Phasengangmethode. Er  beinhaltet  die   abstrakte  Klasse  \code{calculate()}
sowie  alle  n\"otigen  Setter-  und   Getter-Methoden  um  Werte  zu  setzten
und  auszulesen.  Die  Methode  \code{utfController()},  deren Algorithmus  in
Anhang~\ref{app:algo:utfcontroller}  beschrieben  ist,  wird via  die  Methode
\code{setUTFPoly()} hier ausgel\"ost.


\subsubsection*{\code{Chien20}}
Die     Klasse     \code{Chien20}     stellt    die     Algorithmen     (siehe
Anhang~\ref{app:algo:chien})   zur   Berechnung   der   Reglerwerte   gem\"ass
der  Faustformel   Chien/Hrones/Reswick  (20\%)  zur  Verf\"ugung,   erbt  von
\code{Controller}, instanziiert und setzt eine \"Ubertragungsfunktion.


\subsubsection*{\code{ChienApper}}
Die    Klasse     \code{ChienApper}    stellt    die     Algorithmen    (siehe
Anhang~\ref{app:algo:chien})  zur  Berechnung  der  Reglerwerte  gem\"ass  der
Faustformel  Chien/Hrones/Reswick  (aperiodisch)  zur  Verf\"ugung,  erbt  von
\code{Controller}, instanziiert und setzt eine \"Ubertragungsfunktion.


\subsubsection*{\code{Oppelt}}
Die     Klasse     \code{Oppelt}     stellt     die     Algorithmen     (siehe
Anhang~\ref{app:algo:oppelt})  zur  Berechnung  der Reglerwerte  gem\"ass  der
Faustformel Oppelt  zur Verf\"ugung, erbt von  \code{Controller}, instanziiert
und setzt eine \"Ubertragungsfunktion.


\subsubsection*{\code{Rosenberg}}
Die     Klasse    \code{Rosenberg}     stellt    die     Algorithmen    (siehe
Anhang~\ref{app:algo:rosenberg})  zur  Berechnung   der  Reglerwerte  gem\"ass
der  Faustformel  Rosenberg  zur   Verf\"ugung,  erbt  von  \code{Controller},
instanziiert und setzt eine \"Ubertragungsfunktion.


\subsubsection*{\code{PhaseResponseMethod}}
Die Klasse  \code{PhaseResponseMethod} stellt  die Algorithmen  zur Berechnung
der Reglerwerte  gem\"ass der Phasengangmethode zur  Verf\"ugung. Sie erbt von
der Klasse \code{Controller} und bringt  all deren Setter- und Getter-Methoden
mit.

\"Uber  die  \"uberladene  Methode  \code{setData()}  werden  die  Input-Werte
gesetzt. Dabei wird die Methode \code{calculateOverShoot()}, die das f\"ur die
Berechnung  des  korrekten  \"Uberschwingens ben\"otigt  Attribut  \code{phiU}
setzt,    ausgel\"ost. Zus\"atzlich   wird    \code{calculate()}   aufgerufen.
\code{calculate()}  berechnet  anhand   der  Methode  \code{createOmegaAxis()}
die   diskrete  Frequenzachse   in   Abh\"angigkeit   der  Zeitkonstante   der
Regelstrecke. Die \"Ubertragungsfunktion in s der Regelstrecke wird f\"ur alle
Punkte von \code{omega} berechnet.

\code{calculateTnk()}  wird ausgel\"ost  und berechnet  $T_{nk}$ und  $T_{vk}$
unter   Zuhilfenahme  der   diskreten   Werte.  \code{calculateTnk()}   l\"ost
wiederum   \code{calculateKrk()}  zur   Berechnung   von   $K_{rk}$  aus   und
ruft   zudem   \code{calculateControllerConf()}   und   \code{setUTF()}   auf.
\code{calculateControllerConf()} transformiert die Werte in die reglerkonforme
Darstellung  (siehe  Anhang~\ref{app:algo:boderegler}). \code{setUTF()}  setzt
die \"Ubertragungs-Funktion des Reglers.


\subsubsection*{\code{Path}}
Die   Klasse  \code{Path}   berechnet  in   der  Methode   \code{calculate()},
unter   Zuhilfenahme  der   Sani-Methode  (\code{Calc.sani()}),   die  Strecke
(\code{Path}). Sie besitzt  eine Instanz  der Klasse \code{UTF}  sowie Setter-
und Getter-Methoden zum Setzen und Auslesen von Werten.


\subsubsection*{\code{UTF}}
Setzt  und speichert  die  \"Ubertragungsfunktion. Diese  kann \"uber  diverse
Getter- und Setter-Methoden ausgelesen bzw. geschrieben werden.
