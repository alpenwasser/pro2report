\subsection*{Sani}

\subsubsection{Input}

\begin{tabular}{p{40mm}l}
    \code{Tu} & Verzugszeit \\
    \code{Tg} & Anstiegszeit
\end{tabular}

\subsubsection{Output}
\begin{tabular}{p{40mm}l}
    \code{n} & Ordnung der Regelstrecke \\
    \code{T} & Zeitkonstante
\end{tabular}

\subsubsection{Algorithmus}
\begin{enumerate}
    \item
        Ung\"ultige Eingaben warden abgefangen und ein Fehler zur\"uckgegeben.
    \item
        L\"adt Werte f\"ur Tu und Tg.
    \item
        Erstellt 50 Werte zwischen 0 und 1 f\"ur ri.
    \item
        Bestimmt die Ordnung der Regelstrecke.
    \item
        Spline f\"ur r und w
    \item
        T(n) wird aus w*tg berechnet.
    \item
        Umspeichern \& Sortieren
\end{enumerate}

\subsubsection{Matlab-Code}
\lstinputlisting[style=Matlab-editor-2]{mfiles/p2Sani.m}


\subsection*{Umrechnung von reglerkonformer in bodekonforme Darstellung}
\subsubsection{Input}
\begin{tabular}{p{40mm}l}
    \code{Tv}        & Vorhaltezeit \\
    \code{Tn}        & Nachstellzeit \\
    \code{Tp}        & Periodendauer \\
    \code{Kr}        & Verst\"arkungsfaktor des Reglers \\
    \code{Reglertyp} & Typ des Reglers (P, PI, PID)
\end{tabular}

\subsubsection{Output}
\begin{tabular}{p{40mm}l}
    \code{Tnk} & Nachstellzeit \\
    \code{Tvk} & Vorhaltezeit \\
    \code{Krk} & Verst\"arkungsfaktor des Reglers
\end{tabular}

\subsubsection{Algorithmus}
\begin{enumerate}
    \item
        W\"ahlt je nach Reglertyp die Umrechnungsformel.
    \item
        Falls der I-Regler gew\"ahlt wird, gibt der Algorithmus einen Fehler zur\"uck, da der I-Regler nicht implementiert ist.
    \item
        PI-Regler:
        $T_{nk} = T_n$, $K_{rk} = K_r$, $T_{vk} = 0$
    \item
        F\"ur PID-Regler:
        \begin{equation*}
            \varepsilon =\frac{\sqrt{1-(4 \cdot T_n \cdot (T_v-T_p))}}{(T_n+T_p)^2}
        \end{equation*}

        \begin{equation*}
            T_{nk} = \frac{(T_n+T_p) \cdot (1+\varepsilon)}{2}
        \end{equation*}

        \begin{equation*}
            K_{rk} = \frac{K_r \cdot (\frac{1+T_p}{T_{nk}}) \cdot (1+\varepsilon)}{2}
        \end{equation*}

        \begin{equation*}
            T_{vk} = \frac{(T_n+T_p)*(1+\varepsilon)}{2}
        \end{equation*}
\end{enumerate}
\subsubsection{Matlab-Code}
%\lstinputlisting[style=Matlab-editor-2]{mfiles/p2Sani.m}


%    Algorithmus:  \\
%    1.	Entscheidet je nach Reglertyp die Umrechungformel\\
%    2.	Beim I Regler giebt dies einen Error zur\"uck das dieser nicht Implementiert ist.\\
%    3.	F\"ur PI:\\
%		Tnk=Tn , Krk=Kr , Tvk=0 .  \\
%    4.	F\"ur PID:         \\
%\begin{equation}
%\varepsilon =\frac{
%\sqrt{1-(4*Tn*(Tv-Tp))}}{(Tn+Tp)^2}
%\end{equation}    \\
%\begin{equation}
%Tnk= \frac{(Tn+Tp)\cdot(1+\varepsilon)}{2}
%\end{equation} \\
%\begin{equation}
%Krk= \frac{Kr*(\frac{1+Tp}{Tnk})*(1+\varepsilon)}{2}
%\end{equation} \\
%\begin{equation}
%Tvk = \frac{(Tn+Tp)*(1+\varepsilon)}{2}
%\end{equation}\\ \hline
%Matlabcode: \\
%\includegraphics[width=\textwidth] {bodekonf}   \\ \hline
%\end{tabular}
%
%    \begin{tabular}{| p{13cm} |}
%    \hline
%    Umrechnung von Regel- zu Bodekonform \\ \hline
%    Input:
%    \begin{tabbing}
%    \hspace*{4cm}\=\hspace*{2cm}\= \kill
%    Tp	\>			Periodendauer\\
%    Tnk	\>			Nachstellzeit\\
%    Tvk	\>			Vorhaltzeit\\
%    Krk	\>			Verst\"arkungsfaktor des Reglers\\
%    Reglertyp	\>		Reglertyp(I,Pi,PID)
%    \end{tabbing} \\  \hline
%
%    Output:
%    \begin{tabbing}
%    \hspace*{4cm}\=\hspace*{2cm}\= \kill
%    Tn	\>			Nachstellzeit \\
%    Tv	\>			Vorhaltzeit\\
%    Kr	\>			Verst\"arkungsfaktor des Reglers
%    \end{tabbing} \\ \hline
%
%    Algorithmus:  \\
%    1.	Entscheidet je nach Reglertyp die Umrechungformel\\
%    2.	Beim I Regler giebt dies einen Error zur\"uck das dieser nicht Implementiert ist.\\
%    3.	F\"ur PI:\\
%		Tn=Tnk , Kr=Krk , Tv=0 .  \\
%    4.	F\"ur PID:         \\
%\begin{equation}
%Kr= Krk*\frac{1+Tvk}{Tnk}
%\end{equation} \\
%\begin{equation}
%Tn= Tnk+Tvk-Tp
%\end{equation} \\
%\begin{equation}
%Tv = \frac{Tnk*Tvk}{Tnk+Tvk-Tp)-Tp}
%\end{equation}\\ \hline
%
%Matlabcode: \\
%\includegraphics[width=\textwidth] {regelkonf}
%\\ \hline
%    \end{tabular}
%
%
%     \begin{tabular}{| p{13cm} |}
%    \hline
%    utfController \\ \hline
%    Input:
%    \begin{tabbing}
%    \hspace*{4cm}\=\hspace*{2cm}\= \kill
%    Tp	\>			Periodendauer\\
%    Tnk	\>			Nachstellzeit\\
%    Tvk	\>			Vorhaltzeit\\
%    Krk	\>			Verst\"arkungsfaktor des Reglers\\
%    Reglertyp	\>		Reglertyp(I,Pi,PID)
%    \end{tabbing} \\  \hline
%
%    Output:
%    \begin{tabbing}
%    \hspace*{4cm}\=\hspace*{2cm}\= \kill
%    Z\"ahler	\>			Reeller Z\"ahler \\
%    Nenner \>			Reeller Nenner
%    \end{tabbing} \\ \hline
%
%    Algorithmus:  \\
%    1.	Entscheidet je nach Reglertyp die Umrechungformel\\
%    2.	F\"ur PI:\\
%    \begin{equation}
%    Zaehler = Krk* Matrix
%    \end{equation}  \\
%        \begin{equation}
%    Nenner = Matrix
%    \end{equation}  \\
%    3.	F\"ur PID:         \\
%    \begin{equation}
%    Zaehler = Krk *Convolution(Matrix)
%    \end{equation}  \\
%        \begin{equation}
%    Nenner = Matrix
%    \end{equation}  \\ \hline
%
%Matlabcode: \\
%\includegraphics[width=\textwidth] {utfRegler}
%\\ \hline
%    \end{tabular}
%
%
%     \begin{tabular}{| p{13cm} |}
%    \hline
%    Faustformel Oppelt \\ \hline
%    Input:
%    \begin{tabbing}
%    \hspace*{4cm}\=\hspace*{2cm}\= \kill
%    Tu	\>			Verzugszeit\\
%    Tg	\>			Anstiegszeit \\
%    Ks  \>			Reglerverst\"arkung\\
%    Reglertyp	\>		Reglertyp(I,Pi,PID)
%    \end{tabbing} \\  \hline
%
%    Output:
%    \begin{tabbing}
%    \hspace*{4cm}\=\hspace*{2cm}\= \kill
%    Kp \>	Proportionalfaktor	\\
%    Tn	\>	Nachstellzeit		\\
%    Tv	\>	Vorhaltezeit		\\
%    \end{tabbing} \\ \hline
%
%    Algorithmus:  \\
%    1.	Entscheidet je nach Reglertyp die Umrechungformel\\
%    2.	F\"ur PI:\\
%    \begin{equation}
%     k_p= \frac{0.8}{Ks}*\frac{Tg}{Tu}
%    \end{equation}  \\
%    \begin{equation}
%     t_n=3*t_u
%    \end{equation} \\
%    \begin{equation}
%      t_v=0
%    \end{equation} \\
%    3.	F\"ur PID:         \\
%    \begin{equation}
%    Kp = \frac{1.2}{Ks}*\frac{Tg}{Tu}
%    \end{equation}  \\
%    \begin{equation}
%    Tn=2*Tu
%    \end{equation}
%    \begin{equation}
%    Tv=0.42*Tu;
%    \end{equation}\\ \hline
%
%Matlabcode: \\
%\includegraphics[width=\textwidth] {oppelt}
%\\ \hline
%\end{tabular}
%
%
%
%\begin{tabular}{| p{13cm} |}
%    \hline
%    Faustformel Rosenberg \\ \hline
%    Input:
%    \begin{tabbing}
%    \hspace*{4cm}\=\hspace*{2cm}\= \kill
%    Tu	\>			Verzugszeit\\
%    Tg	\>			Anstiegszeit \\
%    Ks  \>			Reglerverst\"arkung\\
%    Reglertyp	\>		Reglertyp(I,Pi,PID)
%    \end{tabbing} \\  \hline
%
%    Output:
%    \begin{tabbing}
%    \hspace*{4cm}\=\hspace*{2cm}\= \kill
%    Kp \>	Proportionalfaktor	\\
%    Tn	\>	Nachstellzeit		\\
%    Tv	\>	Vorhaltezeit		\\
%    \end{tabbing} \\ \hline
%
%    Algorithmus:  \\
%    1.	Entscheidet je nach Reglertyp die Umrechungformel\\
%    2.	F\"ur PI:\\
%    \begin{equation}
%     k_p= \frac{0.91}{Ks}*\frac{Tg}{Tu}
%    \end{equation}  \\
%    \begin{equation}
%     t_n=3.3*t_u
%    \end{equation} \\
%    \begin{equation}
%      t_v=0
%    \end{equation} \\
%    3.	F\"ur PID:    \\
%    \begin{equation}
%    Kp = \frac{1.2}{Ks}*\frac{Tg}{Tu}
%    \end{equation}  \\
%    \begin{equation}
%    Tn=2*Tu
%    \end{equation}
%    \begin{equation}
%    Tv=0.45*Tu;
%    \end{equation}\\ \hline
%
%Matlabcode: \\
%\includegraphics[width=\textwidth] {rosenberg}
%\\ \hline
%\end{tabular}
%
%
%\begin{tabular}{| p{13cm} |}
%    \hline
%    Faustformel Ziegler \\ \hline
%    Input:
%    \begin{tabbing}
%    \hspace*{4cm}\=\hspace*{2cm}\= \kill
%    Tu	\>			Verzugszeit\\
%    Tg	\>			Anstiegszeit \\
%    Ks  \>			Reglerverst\"arkung\\
%    Reglertyp	\>		Reglertyp(I,Pi,PID)
%    \end{tabbing} \\  \hline
%
%    Output:
%    \begin{tabbing}
%    \hspace*{4cm}\=\hspace*{2cm}\= \kill
%    Kp \>	Proportionalfaktor	\\
%    Tn	\>	Nachstellzeit		\\
%    Tv	\>	Vorhaltezeit		\\
%    \end{tabbing} \\ \hline
%
%    Algorithmus:  \\
%    1.	Entscheidet je nach Reglertyp die Umrechungformel\\
%    2.	F\"ur PI:\\
%    \begin{equation}
%     k_p= \frac{0.9}{Ks}*\frac{Tg}{Tu}
%    \end{equation}  \\
%    \begin{equation}
%     t_n=3.33*t_u
%    \end{equation} \\
%    \begin{equation}
%      t_v=0
%    \end{equation} \\
%    3.	F\"ur PID:    \\
%    \begin{equation}
%    Kp = \frac{1.2}{Ks}*\frac{Tg}{Tu}
%    \end{equation}  \\
%    \begin{equation}
%    Tn=2*Tu
%    \end{equation}
%    \begin{equation}
%    Tv=0.5*Tu;
%    \end{equation}\\ \hline
%
%Matlabcode: \\
%\includegraphics[width=\textwidth] {ziegler}
%\\ \hline
%\end{tabular}
%
%
%\begin{tabular}{| p{13cm} |}
%    \hline
%    Faustformel Chien \\ \hline
%    Input:
%    \begin{tabbing}
%    \hspace*{4cm}\=\hspace*{2cm}\= \kill
%    Tu	\>			Verzugszeit\\
%    Tg	\>			Anstiegszeit \\
%    Ks  \>			Reglerverst\"arkung\\
%    Reglertyp	\>		Reglertyp(I,Pi,PID)\\
%    Ueb		\>		\"Uberschwingen\\
%    \end{tabbing} \\  \hline
%
%    Output:
%    \begin{tabbing}
%    \hspace*{4cm}\=\hspace*{2cm}\= \kill
%    Kp \>	Proportionalfaktor	\\
%    Tn	\>	Nachstellzeit		\\
%    Tv	\>	Vorhaltezeit		\\
%    \end{tabbing} \\ \hline
%
%    Algorithmus:  \\
%    1.	Entscheidet je nach Reglertyp die Umrechungformel\\
%    2.	F\"ur PI ohne \"Uberschwingen:\\
%    \begin{equation}
%     k_p= \frac{0.35*Tg}{Ks*Tu}
%    \end{equation}  \\
%    \begin{equation}
%     t_n=1.2*t_u
%    \end{equation} \\
%    \begin{equation}
%      t_v=0
%    \end{equation} \\
%    3.	F\"ur PI mit 20 Prozent \"Uberschwingen:
%    \begin{equation}
%     k_p= \frac{0.7*Tg}{Ks*Tu}
%    \end{equation}  \\
%    \begin{equation}
%     t_n=3*t_u
%    \end{equation} \\
%    \begin{equation}
%      t_v=0
%    \end{equation} \\
%    4.	F\"ur PID ohne \"Uberschwingen:    \\
%    \begin{equation}
%    Kp = \frac{0.9*Tg}{Ks*Tu}
%    \end{equation}  \\
%    \begin{equation}
%    Tn=2.4*Tu
%    \end{equation}
%    \begin{equation}
%    Tv=0.42*Tu;
%    \end{equation}\\
%    5. F\"ur PID mit 20 Prozent \"Uberschwingen:
%     \begin{equation}
%    Kp = \frac{1.2*Tg}{Ks*Tu}
%    \end{equation}  \\
%    \begin{equation}
%    Tn=2*Tu
%    \end{equation}
%    \begin{equation}
%    Tv=0.42*Tu;
%    \end{equation}\\ \hline
%Matlabcode: \\
%\includegraphics[width=\textwidth] {chien}
%\\ \hline
%\end{tabular}
%
%
