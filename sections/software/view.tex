Die  \emph{View} ist  aus  zwei \"ubergeordneten  Panels aufgebaut. Im  linken
Panel befinden sich Ein- und  Ausgabefelder f\"ur numerische Werte, im rechten
Panel werden die zugeh\"origen Plots dargestellt.

\todo{Image Gesamt-GUI}

\todo{Image Panel Schrittantwort vermessen}

Im  Bereich 1  \todo{Image Referenzen}  werden die  Parameter der  vermessenen
Strecke  eingegeben. Darunter  befinden  sich die  Schalftfl\"achen  zur  Wahl
\todo{Image Butttons  PI-, PID-T1-Regler}  zwischen der  Dimensionierung eines
PI- respektive eines PID-T1-Reglers.

Das Panel \emph{Reglerwerte} dient haupts\"achlich der Ausgabe der berechneten
Reglerwerte  mittels  der  verschiedenen  Berechnungsmethoden. Ebenfalls  kann
f\"ur  die Phasengangmethode  die Zeitkonstante  $T_p$ \todo{Check:  korrekter
Begriff} spezifiziert werden.  \todo{Image Panel Phasengangmethode}

\todo{Image Panel rechts}
Der obere Bereich des rechten Panels beinhaltet zwei Slider zur Eingabe des
gew\"unschten \"Uberschwingens respektive des Phasenrands.

Im   unteren  Bereich   werden  die   Plots  der   mittels  Faustformeln   und
Phasengangmethode  errechneten  Resultate   ausgegeben. Zu  jeder  Faustformel
wird   die   zugeh\"orige   Schrittantwort   abgebildet. Die   Resultate   der
Phasengangmethode  werden durch  drei Kurven  dargestellt. Eine Kurve  benutzt
den  Standardwert des  Phasenrands gem\"ass  Zellweger \todo{Einf\"ugen  Wert,
Referenz}, die beiden anderen Kurven basieren auf Benutzereingaben f\"ur einen
oberen und  unteren Offset  des Phasenrandes im  Bereich von  $-45\degree$ bis
$+45\degree$.
