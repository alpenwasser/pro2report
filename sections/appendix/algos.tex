% ---------------------------------------------------------------------------- %
\subsection{Sani}
% ---------------------------------------------------------------------------- %

\subsubsection*{Input}

\begin{tabular}{p{40mm}l}
    $ T_u $ & Verzugszeit \\
    $ T_g $ & Anstiegszeit
\end{tabular}

\subsubsection*{Output}
\begin{tabular}{p{40mm}l}
    Array & \parbox[t][4em][s]{0.7\textwidth}{Array mit den Zeitkonstanten des Polynoms der \"Ubertragungsfunktion der Strecke}
\end{tabular}

\subsubsection*{Algorithmus}
\begin{enumerate}
    \item
        Ung\"ultige Eingaben warden abgefangen und ein Fehler zur\"uckgegeben.
    \item
        L\"adt Werte f\"ur $T_u$ und $T_g$.
    \item
        Erstellt 50 Werte zwischen 0 und 1 f\"ur $r_i$.
    \item
        Bestimmt die Ordnung der Regelstrecke.
    \item
        Spline f\"ur $r$ und $\omega$ \todo{Ja aber was heisst das nun?}
    \item
        $T(n)$ wird aus $\omega \cdot T_g$ berechnet.
    \item
        Umspeichern und Sortieren
\end{enumerate}

\subsubsection*{Matlab-Code}
\lstinputlisting[style=Matlab-editor-2]{mfiles/p2Sani.m}


\clearpage
% ---------------------------------------------------------------------------- %
\subsection{Umrechnung von reglerkonformer in bodekonforme Darstellung}
% ---------------------------------------------------------------------------- %

\subsubsection*{Input}
\begin{tabular}{p{40mm}l}
    $ T_v $        & Vorhaltezeit \\
    $ T_n $        & Nachstellzeit \\
    $ T_p $        & Periodendauer \\
    $ K_r $        & Verst\"arkungsfaktor des Reglers \\
      Reglertyp    & Typ des Reglers (P, PI, PID)
\end{tabular}

\subsubsection*{Output}
\begin{tabular}{p{40mm}l}
    Array & \parbox[t][4em][s]{0.7\textwidth}{Array mit Nachstellzeit $ T_{nk} $, Vorhaltezeit $ T_{vk} $ und Verst\"arkungsfaktor $ K_{rk} $ des Reglers}
\end{tabular}

\subsubsection*{Algorithmus}
\begin{enumerate}
    \item
        W\"ahlt je nach Reglertyp die Umrechnungsformel.
    \item
        Falls  ein nicht  implementierter Reglertyp  gew\"ahlt wird,  wird ein
        Fehler zur\"uckgegeben.
    \item
        Berechnung             der            Reglerwerte             gem\"ass
        Tabelle~\ref{tab:bode_regler_konform}.
\end{enumerate}
\subsubsection*{Matlab-Code}
\lstinputlisting[style=Matlab-editor-2]{mfiles/p2Bodekonf.m}


% ---------------------------------------------------------------------------- %
\subsection{Umrechnung von bodekonformer in reglerkonforme Darstellung}
\label{app:algo:boderegler}
% ---------------------------------------------------------------------------- %

\subsubsection*{Input}

\begin{tabular}{p{40mm}l}
    $ T_p  $ & Periodendauer \\
    $ T{nk} $ & Nachstellzeit \\
    $ T{vk} $ & Vorhaltezeit \\
    $ K{rk} $ & Verst\"arkungsfaktor des Reglers \\
      Reglertyp   & Reglertyp (P, PI, PID)
\end{tabular}

\subsubsection*{Output}
\begin{tabular}{p{40mm}l}
    Array & \parbox[t][4em][s]{0.7\textwidth}{Array mit Nachstellzeit $T_n$, Vorhaltezeit $T_v$ und Verst\"arkungsfaktor $K_r$ des Reglers}
\end{tabular}

\subsubsection*{Algorithmus}
\begin{enumerate}
    \item
        W\"ahlt je nach Reglertyp die Umrechnungsformel.
    \item
        Falls  ein nicht  implementierter Reglertyp  gew\"ahlt wird,  wird ein
        Fehler zur\"uckgegeben.
    \item
        PI-Regler:
        $T_{n} = T_{nk}$, $K_{k} = K_{rk}$, $T_{v} = 0$
    \item
        Berechnung             der            Reglerwerte             gem\"ass
        Tabelle~\ref{tab:bode_regler_konform}.
\end{enumerate}

\subsubsection*{Matlab-Code}
\lstinputlisting[style=Matlab-editor-2]{mfiles/p2Reglerkonf.m}


\clearpage
% ---------------------------------------------------------------------------- %
\subsection{utfController}
% ---------------------------------------------------------------------------- %

\subsubsection*{Input}

\begin{tabular}{p{40mm}l}
    $ T_p $        & Verzugszeit \\
    $ T_{nk} $     & Nachstellzeit \\
    $ T_{vk} $     & Vorhaltezeit \\
    $ K_{rk} $     & Verst\"arkungsfaktor des Reglers \\
      Reglertyp    & Reglertyp (P, PI, PID)
\end{tabular}

\subsubsection*{Output}
\begin{tabular}{p{40mm}l}
    Array & Array mit reellem Z\"ahler und reellem Nenner
\end{tabular}

\subsubsection*{Algorithmus}
\begin{enumerate}
    \item
        W\"ahlt je nach Reglertyp die korrekten Formeln.
    \item
        \todo{Matrix}
\end{enumerate}

\subsubsection*{Matlab-Code}
\lstinputlisting[style=Matlab-editor-2]{mfiles/p2UTFRegler.m}


% ---------------------------------------------------------------------------- %
\subsection{Faustformel Oppelt}
% ---------------------------------------------------------------------------- %

\subsubsection*{Input}

\begin{tabular}{p{40mm}l}
    $ T_p $        & Verzugszeit \\
    $ T_u $        & Anstiegszeit \\
    $ K_s $        & Verst\"arkung der Strecke \\
      Reglertyp    & Reglertyp (P, PI, PID)
\end{tabular}

\subsubsection*{Output}
\begin{tabular}{p{40mm}l}
    Array & \parbox[t][4em][s]{0.7\textwidth}{Array mit Proportionalit\"atsfaktor $ K_p $, Nachstellzeit $ T_n $ und Vorhaltezeit $ T_v $}
\end{tabular}

\subsubsection*{Algorithmus}
\begin{enumerate}
    \item
        W\"ahlt je nach Reglertyp die korrekten Formeln.
    \item
        F\"ur PI:
        \begin{equation*}
             K_p= \frac{0.8}{K_s} \cdot \frac{T_g}{T_u}
        \end{equation*}

        \begin{equation*}
             T_n=3 \cdot T_u
        \end{equation*}

        \begin{equation*}
              t_v=0
        \end{equation*}
    \item
        F\"ur PID:
        \begin{equation*}
            K_p = \frac{1.2}{K_s} \cdot \frac{T_g}{T_u}
        \end{equation*}

        \begin{equation*}
            T_n=2 \cdot T_u
        \end{equation*}

        \begin{equation*}
            T_v=0.42*T_u
        \end{equation*}
\end{enumerate}

%\subsubsection*{Matlab-Code}
%\lstinputlisting[style=Matlab-editor-2]{mfiles/p2Ffoppelt.m}


\clearpage
% ---------------------------------------------------------------------------- %
\subsection{Faustformel Rosenberg}
% ---------------------------------------------------------------------------- %

\subsubsection*{Input}

\begin{tabular}{p{40mm}l}
    $ T_p $        & Verzugszeit \\
    $ T_u $        & Anstiegszeit \\
    $ K_s $        & Verst\"arkung der Strecke \\
      Reglertyp    & Reglertyp (P, PI, PID)
\end{tabular}

\subsubsection*{Output}
\begin{tabular}{p{40mm}l}
    Array & \parbox[t][4em][s]{0.7\textwidth}{Array mit Proportionalit\"atsfaktor $ K_p $, Nachstellzeit $ T_n $ und Vorhaltezeit $ T_v $}
\end{tabular}

\subsubsection*{Algorithmus}
\begin{enumerate}
    \item
        W\"ahlt je nach Reglertyp die korrekten Formeln.
    \item
    \item
        F\"ur PI:

        \begin{equation*}
            K_p= \frac{0.91}{K_s} \cdot \frac{T_g}{T_u}
        \end{equation*}

        \begin{equation*}
            T_n=3.3 \cdot T_u
        \end{equation*}

        \begin{equation*}
            T_v=0
        \end{equation*}
    \item
        F\"ur PID:

        \begin{equation*}
            Kp = \frac{1.2}{K_s} \cdot \frac{T_g}{T_u}
        \end{equation*}  \\

        \begin{equation*}
            T_n=2 \cdot T_u
        \end{equation*}

        \begin{equation*}
            T_v=0.45 \cdot T_u;
        \end{equation*}
\end{enumerate}

%\subsubsection*{Matlab-Code}
%\lstinputlisting[style=Matlab-editor-2]{mfiles/p2Ffrosenberg.m}


\clearpage
% ---------------------------------------------------------------------------- %
\subsection{Faustformel Ziegler}
% ---------------------------------------------------------------------------- %

\subsubsection*{Input}

\begin{tabular}{p{40mm}l}
    $ T_p $        & Verzugszeit \\
    $ T_u $        & Anstiegszeit \\
    $ K_s $        & Verst\"arkung der Strecke \\
      Reglertyp   & Reglertyp (P, PI, PID)
\end{tabular}

\subsubsection*{Output}
\begin{tabular}{p{40mm}l}
    Array & \parbox[t][4em][s]{0.7\textwidth}{Array mit Proportionalit\"atsfaktor $ K_p $, Nachstellzeit $ T_n $ und Vorhaltezeit $ T_v $}
\end{tabular}

\subsubsection*{Algorithmus}
\begin{enumerate}
    \item
        W\"ahlt je nach Reglertyp die korrekten Formeln.
    \item
        F\"ur PI:
        \begin{equation*}
            K_p= \frac{0.9}{K_s} \cdot \frac{T_g}{T_u}
        \end{equation*}

        \begin{equation*}
            T_n=3.33 \cdot T_u
        \end{equation*}

        \begin{equation*}
            T_v=0
        \end{equation*}
    \item
        F\"ur PID:
        \begin{equation*}
            K_p = \frac{1.2}{K_s} \cdot \frac{T_g}{T_u}
        \end{equation*}

        \begin{equation*}
            T_n=2 \cdot T_u
        \end{equation*}

        \begin{equation*}
            T_v = 0.5 \cdot T_u
        \end{equation*}
\end{enumerate}

%\subsubsection*{Matlab-Code}
%\lstinputlisting[style=Matlab-editor-2]{mfiles/p2Ffziegler.m}

\clearpage
% ---------------------------------------------------------------------------- %
\subsection{Faustformel Chien}
% ---------------------------------------------------------------------------- %

\subsubsection*{Input}

\begin{tabular}{p{40mm}l}
    $ T_p $             & Verzugszeit \\
    $ T_u $             & Anstiegszeit \\
    $ K_s $             & Verst\"arkung der Strecke \\
      Reglertyp         & Reglertyp (P, PI, PID) \\
    \"Uberschwingen     & Flag f\"ur \"Uberschwingeng %\TODO{specifics $
\end{tabular}

\subsubsection*{Output}
\begin{tabular}{p{40mm}l}
    Array & \parbox[t][4em][s]{0.7\textwidth}{Array mit Proportionalit\"atsfaktor $K_p$, Nachstellzeit $T_n$ und Vorhaltezeit $T_v$}
\end{tabular}

\subsubsection*{Algorithmus}
\begin{enumerate}
    \item
        W\"ahlt je nach Reglertyp die korrekten Formeln.
    \item
        F\"ur PI, ohne \"Uberschwingen:
        \begin{equation*}
            K_p= \frac{0.35 \cdot T_g}{K_s \cdot T_u}
        \end{equation*}

        \begin{equation*}
            T_n=1.2 \cdot T_u
        \end{equation*}

        \begin{equation*}
            T_v=0
        \end{equation*}

    \item
        F\"ur PI, 20\% \"Uberschwingen:

        \begin{equation*}
            K_p= \frac{0.7 \cdot T_g}{K_s \cdot T_u}
        \end{equation*}

        \begin{equation*}
            T_n=3 \cdot T_u
        \end{equation*}

        \begin{equation*}
            T_v=0
        \end{equation*}
    \item
        F\"ur PID, ohne \"Uberschwingen:

        \begin{equation*}
            K_p = \frac{0.9 \cdot Tg}{Ks \cdot Tu}
        \end{equation*}

        \begin{equation*}
            T_n=2.4 \cdot T_u
        \end{equation*}

        \begin{equation*}
            T_v=0.42 cdot T_u;
        \end{equation*}
    \item
        F\"ur PID, 20\% \"Uberschwingen:

        \begin{equation*}
            K_p = \frac{1.2 \cdot T_g}{K_s \cdot T_u}
        \end{equation*}

        \begin{equation*}
            T_n=2 \cdot T_u
        \end{equation*}

        \begin{equation*}
            T_v=0.42 \cdot T_u;
        \end{equation*}
\end{enumerate}

%\subsubsection*{Matlab-Code}
%\lstinputlisting[style=Matlab-editor-2]{mfiles/p2Ffchien.m}


\clearpage
% ---------------------------------------------------------------------------- %
\subsection{Steigung einer Funktion}
% ---------------------------------------------------------------------------- %

\code{diskDiff()} berechnet die Steigung einer Funktion (repr\"asentiert durch
zwei Arrays) an einem bestimmten Array-Index.

\subsubsection*{Input}

\begin{tabular}{p{40mm}l}
    x-Array        & Array mit x-Werten \\
    y-Array        & Array mit zugeh\"origen Funktionswerten \\
    Index          & Index, an dem die Steigung berechnet werden soll
\end{tabular}

\subsubsection*{Output}
\begin{tabular}{p{40mm}l}
    Steigung   & Steigung an gesuchter Stelle
\end{tabular}

\subsubsection*{Algorithmus}
\begin{enumerate}
    \item
        Pr\"ufen, ob Index innerhalb des Arrays liegt.
    \item
        Steigungsdreieck  zwischen  Element  and  Index  und  den  unmittelbar
        daneben liegenden Array-Elementen bilden.
    \item
        Durchschnitt der beiden Steigungsdreiecke ausrechnen.
    \item
        Falls Steigung an erster Array-Stelle verlangt ist: Steigungsdreieck
        mit dem zweiten Element bilden und Steigung zur\"uckgeben.
    \item
        Falls Steigung an letzter Array-Stelle verlangt ist: Steigungsdreieck
        mit zweitletztem Array-Element bilden und Steigung zur\"uckgeben.
\end{enumerate}

\subsubsection*{Java-Code}
\lstinputlisting[style=java]{java/diskDiff.java}


\clearpage
% ---------------------------------------------------------------------------- %
\subsection{Inverse Fast Fourier Transform}
% ---------------------------------------------------------------------------- %

\code{schrittIfft()}  berechnet   die  Schrittantwort  im   Zeitbereich  einer
\"Ubertragungsfunktion. \todo{korrekt?}

\subsubsection*{Input}

\begin{tabular}{p{40mm}l}
    Z\"ahler & Z\"ahler der \"Ubertragungsfunktion                             \\
    Nenner   & Nenner der \"Ubertragungsfunktion                               \\
    $f_s$    & Abstastfrequenz                                                 \\
    n        & Granulierung der Frequenzachse
\end{tabular}

\subsubsection*{Output}
\begin{tabular}{p{40mm}l}
    Resultat & \parbox[t][4em][s]{0.7\textwidth}{Zweidimensionales Array mit Zeitachse und zugeh\"origen Funktionswerten}
\end{tabular}

\subsubsection*{Algorithmus}
\begin{enumerate}
    \item
        Array mit Frequenzachse generieren.
    \item
        Frequenzgang der \"Ubertragungsfunktion berechnen.
    \item
        Impulsantwort im Frequenzbereich berechnen.
    \item
        In den Zeitbereich zur\"ucktransformieren.
    \item
        Aus   den   Realteilen    des   Resultat-Arrays   die   Schrittantwort
        zusammensetzen    (aufsummieren    des   Realteils    des    aktuellen
        Array-Elements    mit    den     Realteilen    aller    vorhergehenden
        Array-Elementen).
\end{enumerate}

\subsubsection*{Java-Code}
\lstinputlisting[style=java]{java/schrittIfft.java}


\clearpage
% ---------------------------------------------------------------------------- %
\subsection{Optimieren des \"Uberschwingens}
\label{app:algo:oversh}
% ---------------------------------------------------------------------------- %

\code{overShootOptimisation()}   optimiert   das  \"Uberschwingverhalten   des
generierten Reglers.

\subsubsection*{Input}

\begin{tabular}{p{40mm}l}
    Keine Eingabewerte &
\end{tabular}

\subsubsection*{Output}
\begin{tabular}{p{40mm}l}
    Keine R\"uckgabewerte &
\end{tabular}

\subsubsection*{Algorithmus}
\begin{enumerate}
    \item
        Das Maximum in der Schrittantwort des geschlossenen Regelkreises finden.
    \item
        Diesen Wert mit dem vom Benutzer gew\"unschten maximalen \"Uberschwingen vergleichen.
    \item
        Die  Optimierung des  \"Uberschwingens  erfolgt  in zwei  Stufen: Eine
        rasche, aber  grobe, erste Stufe,  und eine langsamere,  aber feinere,
        zweite Stufe.   Dies erlaubt, rasch in  den gew\"unschten Wertebereich
        zu kommen, ohne dass dabei das Endergebnis an Genauigkeit verliert. In
        beiden Stufen ist das Konzept identisch:
    \begin{enumerate}
        \item
            Falls  zu  starkes  \"Uberschwingen: Reglerverst\"arkung  $K_{rk}$
            schrittweise reduzieren,  bis gew\"unschtes  Verhalten eingehalten
            wird.
        \item
            Falls  zu schwaches  \"Uberschwingen: Reglerverst\"arkung $K_{rk}$
            schrittweise  erh\"ohen, bis  gew\"unschtes Verhalten  eingehalten
            wird.
    \end{enumerate}
\end{enumerate}

\subsubsection*{Java-Code}
\lstinputlisting[style=java]{java/overshootoptimisation.java}


\clearpage
% ---------------------------------------------------------------------------- %
\subsection{Berechnung von $T_{vk}$ und $T_{nk}$}
% ---------------------------------------------------------------------------- %
\label{app:algo:tnktvk}

\code{calculateTnkTvk()}    bestimmt   $T_{nk}$    und   $T_{vk}$. Kern    des
Algorithmus  ist  die  Automatisierung   der  Berechnung  von  $\beta$  (siehe
Anhang~\ref{app:beta} f\"ur die manuelle Rechnung);

\subsubsection*{Input}

\begin{tabular}{p{40mm}l}
    Keine Eingabewerte &
\end{tabular}

\subsubsection*{Output}
\begin{tabular}{p{40mm}l}
    Keine R\"uckgabewerte &
\end{tabular}

\subsubsection*{Algorithmus}
\begin{enumerate}
        \item
            Bestimmen der Steigung der Strecke bei Frequenz $\omega_{pid}$
        \item
            Unteren  Wert   f\"ur  $\beta$   festlegen: $10^{-12}$. Wert  muss
            gr\"sser  als null,  aber  sehr  klein sein,  damit  der Rest  des
            Algorithmus funktioniert.
        \item
            Obere Grenze f\"ur $\beta$ auf 1 legen.
        \item
            \"Uber 20 Iterationen folgende Arbeitsschritte ausf\"uhren:
            \begin{enumerate}
                \item
                    Aktuellen Testwert f\"ur $\beta$ definieren: $\beta_{oben}
                    - \beta_{unten} \cdot 0.5 + \beta_{unten}$.
                \item
                    Mit diesem  Wert f\"ur  $\beta$ nun $T_{nk}$  und $T_{vk}$
                    berechnen gem\"ass Gleichung~\ref{eq:pid:beta:start}.
                \item
                    $T_{nk}$  und  $T_{vk}$   in  die  Reglergleichung  (siehe
                    Gleichung~\ref{eq:pid:t_nk_t_vk_initial_results}) einsetzen.
                    Man  beachte,   dass  der  Algorithmus   nicht  symbolisch
                    rechnet, sondern mit  Werte-Arrays f\"ur die Frequenzachse
                    und Funktionswerte der \"Ubertragungsfunktion.
                \item
                    Steigung    des   Phasengangs    des   Reglers    an   der
                    Stelle  $\omega_{pid}$  bestimmen,  aufsummieren  mit  der
                    Steigung  der  Strecke   (bereits  bekannt)  zur  Steigung
                    des   Phasengangs   des    offenen   Regelkreises   (siehe
                    Gleichung~\ref{eq:pid:phi_sum}).
                \item
                    Falls   die    Steigung   $\varphi_o$   an    der   Stelle
                    $\omega_{pid}$  kleiner   ist  als   $-\frac{1}{2}$,  muss
                    $\beta$ vergr\"ossert  werden.  In diesem Fall  die untere
                    Grenze $\beta_{unten}$ auf  den aktuellen Testwert $\beta$
                    setzen.
                \item
                    Falls   die    Steigung   $\varphi_o$   an    der   Stelle
                    $\omega_{pid}$  gr\"osser  ist  als  $-\frac{1}{2}$,  muss
                    $\beta$   verkleinert    werden. Daher   neue   Obergrenze
                    $\beta_{oben}$ auf den aktuellen Testwert $\beta$ setzen.
            \end{enumerate}
\end{enumerate}

\subsubsection*{Java-Code}
\lstinputlisting[style=java]{java/calculateTnkTvk.java}
