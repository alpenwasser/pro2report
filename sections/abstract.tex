\todo{Vermutlich zu knapp G fragen}
Die Aufgabe eines geschlossenen Regelkreises  ist es der Fehler zwischen Soll-
und Ist-Wert m\"oglichst klein zu  halten. Um dies zu gew\"ahrleisten wird der
Regler dimensioniert. Die Aufgabe des Tools ist, der PI und den PID-T1 richtig
zu dimensionieren.

Die  Anforderung  an  die  Eingabe   sind  die  numerischen  Werte  von  einer
vorliegenden  und  ausgemessenen  Schrittantwort. Das Ziel  der  Ausgabe  sind
sowohl die  numerischen Regelwerte der Phasengangmethode  und der Faustformeln
sowie die Plots der Schrittantworten.

Die    Berechungen   wurden    in   Matlab    durchgef\"uhrt   und    in   das
Model-View-Controller-Plattern \"ubertragen. Als Basis  dient die Klasse Calc,
welche die Grundrechenfunktionen von Matlab beinhaltet.

Ein   benutzerfreundliches   Tool   um   in   der   Praxis   die   Reglerwerte
einzustellen. Das Neue an  dieser L\"osung ist, dass  die Phasengangmethode in
ein Berechungstool integriert ist.

Das     Tool    ist     in     der    Regeldimensionierung     einsetzbar. Das
Verbesserungspotential liegt in der Optimierung der Rechengeschwindigkeit.
