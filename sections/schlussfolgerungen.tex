%  was laeuft,  was  laeuft  nicht, was  kann  verbessert  werden Das  Projekt
wurde  erfolgreich abgeschlossen: Das  Tool  ist  funktionst\"uchtig und  alle
Punkte  des  urspr\"unglichen  Auftrages  sowie  einige  der  im  Verlauf  des
Projekts  dazugekommenen optionalen  Anforderungen  wurden erf\"ullt. Aus  den
Eingabewerten  wird das  dynamische Verhalten  des geschlossenen  Regelkreises
berechnet und graphisch  dargestellt.  Die Erweiterungen {\color{red}{\emph{Ist
Erweiterungen hier korrekt? War das Dimensionieren  des Reglers anhand der PGM
nicht  Grundvoraussetzung  des  Projekts?}}} erm\"oglichen  das  Dimensionieren
des   Reglers   anhand  der   Phasengangmethode   und   die  Optimierung   der
Reglerwerte {\color{red}{\emph{Dies  scheint mir  die Erweiterung  zu sein. Wenn
ich das falsch in Erinnerung habe, sorry.}}}.

Die vor  der Erweiterung des Auftrags {\color{red}{\emph{Vielleicht ganz kurz,
allenfalls  in Fussnote,  erl\"autern, was  hier genau  passiert ist. So  ganz
ohne  Kontext  von  der  Erweiterung   des  Auftrags  zu  reden,  scheint  mir
etwas  verwirrend, ausser  wir  erw\"ahnen es  sonst noch  irgendwo? Ansonsten
vielleicht einfach die Auftragserweiterung weglassen und lediglich erw\"ahnen,
dass  gewisse   anfangs  geplante   Elemente  nicht  mehr   notwendig  sind.}}}
optional geplanten  graphischen Ausgaben  des Amplitudengangs und  der Strecke
sowie  die   Schiebregler  f\"ur   das  Ver\"andern  der   Reglerwerte  wurden
weggelassen. Urspr\"unglich  waren diese  geplant, um  das Dimensionieren  des
Reglers zu erleichtern und die Eingabewerte graphisch darzustellen.

Da die  Algorithmen zur  Reglerdimensionierung jedoch so  implementiert werden
konnten, dass eine Optimierung des Reglers in Echtzeit m\"oglich wurde, wurden
diese Zusatzelemente \"uberf\"ussig.

Alternativ kann  die Schrittantwort mit zwei  Schiebereglern manuell optimiert
werden. Durch  das  Ver\"andern  des gew\"unschten  \"Uberschwingens  und  das
Einstellen der Optimierung kann die Kurve von Hand angepasst werden.

Als  Erweiterung  w\"are  es  m\"oglich, die  Streckenwerte  auszugegeben  und
die   Strecke  graphisch   zur\"uckzugeben. Das   automatische  Einlesen   der
Streckenwerte  aus einer  graphisch  Form  k\"onnte hinzugef\"ugt  werden. Das
Tool  w\"urde  an  die  eingelesene  Kurve die  Wendetangente  legen  und  die
Werte  ablesen. Das Ablesen  k\"onnte  mit  der Monte-Carlo-Studie  verfeinert
werden.{\color{red}{\emph{Regelungstechniker  m\"ogen  das  vielleicht  kennen,
aber wir haben  ja auch bei den absoluten Grundlagen  angefangen zu erkl\"aren
in vorhergehenden  Kapiteln. Jemand, der  diese Einf\"hrung  ben\"otigt, weiss
wohl kaum, was dies ist. Allenfalls einfach Verweis auf eine Quelle?}}}
