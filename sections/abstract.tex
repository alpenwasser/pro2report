In  der   Regelungstechnik  ist   die  korrekte  Dimensionierung   der  Regler
essentiell. Die  zu  regelnde  Strecke  kann nur  durch  richtig  eingestellte
Regelwerte wie gew\"unscht beeinflusst werden. 
 

Dieses  Projekt  hat sich  zum  Ziel  gesetzt,  eine Software  zu  entwickeln,
welche  aus   den eingegebenen Streckenwerten  $K_s$,  $T_u$  und   $T_g$  PI-
und  PID-Regler  dimensioniert. Das  Tool  berechnet die  Reglerwerte mit  der
sogenannten   Phasengangmethode,  welche   von  Jakob   Zellweger,  ehemaliger
Dozent   der  Fachhochschule   Nordwestschweiz, stammt. Diese  Methode   wurde
urspr\"unglich  f\"ur die  graphische  Auswertung  des Phasengangs  entwickelt
und  durch Handarbeit mit  Geodreieck und  Bleistift angewendet. Die  Software
erm\"oglicht,  die  bew\"ahrte  Methode  effizient durch  ein  automatisiertes
Verfahren anzuwenden.
 

Das  entwickelte   Softwaretool simuliert  neben  den berechneten Reglerwerten
auch die    Schrittantwort des     geschlossenen    Regelkreises. Zus\"atzlich
wurde      die       Berechnung      dreier       g\"angiger      Faustformeln
implementiert. Zum Vergleich k\"onnen die Schrittantworten der Faustformeln im
gleichen Plot angezeigt werden.


F\"ur die Phasengangmethode werden gleichzeitig drei Graphen abgebildet. Einer
mit   der   Standard-Reglerknickfrequenz    gem\"ass   Phasengangmethode   und
zwei weitere, welche sich im  Vorzeichen unterscheiden und manuell einstellbar
sind.
 

Zus\"atzlich  kann die  Reglerknickfrequenz  und  das \"Uberschwingen  mittels
eines   Schiebereglers  manuell   ver\"andert  werden. Dies   erm\"oglicht  im
Vergleich zu  der  urspr\"unglichen  Ausf\"uhrung  der  Phasengangmethode  per
Hand  enorme  Zeitersparnisse.   Somit  wird  eine  Optimierung  der  mit  der
Phasengangmethode erzielten Ergebnisse in Echtzeit m\"oglich.
