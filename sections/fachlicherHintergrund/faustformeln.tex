Im  Praxiseinsatz  stehen  f\"ur  die  Dimensionierung  von  Reglern  einfache
Berechnungsformeln  zur  Verf\"ugung  (siehe  Tabelle~\ref{tab:faustformeln}).
Diese  liefern Einstellwerte  anhand  von $T_u$,  $T_g$  und $K_s$. An  dieser
Stelle wird  daher unsere  Beispielstrecke zuerst  mit einigen  der g\"angigen
Faustformeln dimensioniert, um das Ergebnis anschliessend mit dem Resultat der
Phasengangmethode vergleichen zu k\"onnen.

\begin{longtable}{p{50mm}rrrrr}
    \toprule

    %\multicolumn{3}{l}{\large{\textsc{Auftragsanalyse und Hintergrundinformationen}}} \\

    Faustformel
    &
    \multicolumn{2}{l}{PI-Regler}
    &
    \multicolumn{2}{l}{PID-T1-Regler}
    \\

    &
    $T_n$
    &
    $K_p$
    &
    $T_n$
    &
    $T_v$
    &
    $K_p$
    \\

    \midrule

    \endhead
    \endfoot
    \endlastfoot

    % CONTENT HERE ---------------------------------------------------------- %

    \pbox{45mm}{Chiens, Hrones, Reswick \\ \small{\textbf{(0\% \"Uberschwingen)}} \\ \cite{ref:chiens_tsn}, \cite{ref:chiens_wiki}}
    &
    $1.2\cdot T_g$
    &
    $\frac{0.35}{K_s} \cdot \frac{T_g}{T_u}$
    &
    $T_g$
    &
    $0.5\cdot T_u$
    &
    $ \frac{0.6}{K_s} \cdot \frac{T_g}{T_u} $
    \\

    \addlinespace[1em]

    \pbox{45mm}{Chiens, Hrones, Reswick \small{\textbf{(20\% \"Uberschwingen)}} \\ \cite{ref:chiens_tsn}, \cite{ref:chiens_wiki}}
    &
    $T_g$
    &
    $\frac{0.6}{K_s} \cdot \frac{T_g}{T_u}$
    &
    $1.35\cdot T_g$
    &
    $0.47 \cdot T_u$
    &
    $ \frac{0.95}{K_s} \cdot \frac{T_g}{T_u} $
    \\

    \addlinespace[1em]

    Oppelt \cite{ref:op_ros_zieg}
    &
    $3 \cdot T_u$
    &
    $\frac{0.8}{K_s} \cdot \frac{T_g}{T_u}$
    &
    $2 \cdot T_u$
    &
    $ 0.42 \cdot T_u $
    &
    $ \frac{1.2}{K_s} \cdot \frac{T_g}{T_u} $
    \\

    \addlinespace[1em]

    Rosenberg \cite{ref:op_ros_zieg}
    &
    $3.3 \cdot T_u $
    &
    $ \frac{0.91}{K_s} \cdot \frac{T_g}{T_u} $
    &
    $ 2 \cdot T_u $
    &
    $ 0.45 \cdot T_u $
    &
    $ \frac{1.2}{T_s} \cdot \frac{T_g}{T_u}$
    \\

    \addlinespace[1em]

    \bottomrule
\caption{Faustformeln zur Reglerdimensionierung}
\label{tab:faustformeln}
\end{longtable}

Setzt man  die Werte f\"ur $K_s$,  $T_u$, $T_g$ in diese  Formeln ein, ergeben
sich die Werte aus Tabelle~\ref{tab:ff_results}.

\begin{longtable}{p{50mm}rrrrr}
    \toprule

    %\multicolumn{3}{l}{\large{\textsc{Auftragsanalyse und Hintergrundinformationen}}} \\

    Faustformel
    &
    \multicolumn{2}{l}{PI-Regler}
    &
    \multicolumn{2}{l}{PID-T1-Regler}
    \\

    &
    $T_n$
    &
    $K_p$
    &
    $T_n$
    &
    $T_v$
    &
    $K_p$
    \\

    \midrule

    \endhead
    \endfoot
    \endlastfoot

    % CONTENT HERE ---------------------------------------------------------- %

    \pbox{45mm}{Chiens, Hrones, Reswick \\ \small{\textbf{(0\% \"Uberschwingen)}} \\ \cite{ref:chiens_tsn}, \cite{ref:chiens_wiki}}
    &
    $\SI{10.68}{\second}$
    &
    $1.42$
    &
    $\SI{8.9}{\second}$
    &
    $\SI{0.55}{\second}$
    &
    $2.43$
    \\

    \addlinespace[1em]

    \pbox{45mm}{Chiens, Hrones, Reswick \small{\textbf{(20\% \"Uberschwingen)}} \\ \cite{ref:chiens_tsn}, \cite{ref:chiens_wiki}}
    &
    $\SI{8.9}{\second}$
    &
    $2.43$
    &
    $\SI{12.02}{\second}$
    &
    $\SI{052}{\second}$
    &
    $3.84$
    \\

    \addlinespace[1em]

    Oppelt \cite{ref:op_ros_zieg}
    &
    $\SI{3.3}{\second}$
    &
    $3.24$
    &
    $\SI{2.2}{\second}$
    &
    $\SI{0.46}{\second}$
    &
    $4.85$
    \\

    \addlinespace[1em]

    Rosenberg \cite{ref:op_ros_zieg}
    &
    $\SI{3.63}{\second}$
    &
    $3.68$
    &
    $\SI{2.2}{\second}$
    &
    $\SI{0.50}{\second}$
    &
    $4.85$
    \\

    \addlinespace[1em]

    \bottomrule
    \caption{Reglerparameter bestimmt mit Faustformeln aus Tabelle~\ref{tab:faustformeln}}
\label{tab:ff_results}
\end{longtable}

Die Algorithmen  zu diesen Faustformeln sind  in Anhang~\ref{app:algo:oppelt},
\ref{app:algo:rosenberg},   \ref{app:algo:ziegler}  und   \ref{app:algo:chien}
genauer beschrieben.
