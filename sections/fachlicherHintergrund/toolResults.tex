Zum     Vergleich     mit    den     in     Abschnitt~\ref{subs:phasengang:pi}
und~\ref{subs:phasengang:pid}  dimensionierten PI-  und PID-Reglern  sind hier
die Resultate unseres Tools gezeigt. Wie zu  erwarten, weisen die mit dem Tool
berechneten Werte f\"ur die Faustformeln keine signifikanten Abweichenungen zu
den Werten in Tabelle~\ref{tab:ff_results} auf.

Verwendet        man         in        den         Berechnungen        mittels
Phasengangmethode      den      Standardwert      f\"ur      $\varphi_{start}$
(siehe      Gleichungen~\ref{eq:pi:phi_s}      und~\ref{eq:pid:phi_s}      auf
Seite~\pageref{eq:pi:phi_s}  respektive~\pageref{eq:pid:phi_s}), erh\"alt  man
Zahlen, die ziemlich nahe bei den von Hand berechneten Resultaten liegen. Wird
jedoch  von  den  Optimierungsm\"oglichkeiten   des  Tools  Gebrauch  gemacht,
\"andern  sich  die  Reglerparameter   und  das  Verhalten  des  zugeh\"origen
geschlossenen Regelkreises teilweise betr\"achtlich.

Die        Zahlenwerte       sind        in       Tabelle~\ref{tab:allresults}
zusammengefasst,       einige       Plots        zum       Vergleich       der
Schrittantworten    der   zugeh\"origen    geschlossenen   Regelkreise    sind
in    den    Abbildungen~\ref{fig:comparisonPI015},~\ref{fig:comparisonPID015}
und~\ref{fig:comparisonPID002optimisations}   zu   sehen.   Alle   Werte   und
Plots  beziehen   sich  auf   die  in   Abschnitt~\ref{subs:frequenzgang}  auf
Seite~\pageref{subs:frequenzgang} vermessene Strecke.

\begin{longtable}{p{85mm}rrrrr}
    \toprule

    %\multicolumn{3}{l}{\large{\textsc{Auftragsanalyse und Hintergrundinformationen}}} \\

    Berechnungsmethode
    &
    \multicolumn{2}{l}{PI-Regler}
    &
    \multicolumn{2}{l}{PID-T1-Regler}
    \\

    &
    $T_n$
    &
    $K_p$
    &
    $T_n$
    &
    $T_v$
    &
    $K_p$
    \\

    \midrule

    \endhead
    \endfoot
    \endlastfoot

    % CONTENT HERE ---------------------------------------------------------- %

    \pbox{84mm}{Chiens, Hrones, Reswick (manuell) \\ \small{\textbf{(0\% \"Uberschwingen)}}~\cite{ref:chiens_tsn},~\cite{ref:chiens_wiki}}
    &
    $\SI{10.68}{\second}$
    &
    $1.42$
    &
    $\SI{8.9}{\second}$
    &
    $\SI{0.55}{\second}$
    &
    $2.43$
    \\

    \addlinespace[0.5em]

    \pbox{84mm}{Chiens, Hrones, Reswick (Software) \\ \small{\textbf{(0\% \"Uberschwingen)}}}
    &
    $\SI{10.68}{\second}$
    &
    $1.42$
    &
    $\SI{8.9}{\second}$
    &
    $\SI{0.55}{\second}$
    &
    $2.43$
    \\

    \addlinespace[0.5em]

    \pbox{84mm}{Chiens, Hrones, Reswick (manuell) \\ \small{\textbf{(20\% \"Uberschwingen)}}~\cite{ref:chiens_tsn},~\cite{ref:chiens_wiki}}
    &
    $\SI{8.9}{\second}$
    &
    $2.43$
    &
    $\SI{12.02}{\second}$
    &
    $\SI{0.52}{\second}$
    &
    $3.84$
    \\

    \addlinespace[0.5em]

    \pbox{84mm}{Chiens, Hrones, Reswick (Software) \\ \small{\textbf{(20\% \"Uberschwingen)}}}
    &
    $\SI{8.9}{\second}$
    &
    $2.43$
    &
    $\SI{12.02}{\second}$
    &
    $\SI{0.517}{\second}$
    &
    $3.84$
    \\

    \addlinespace[0.5em]

    Oppelt (manuell)~\cite{ref:op_ros_zieg}
    &
    $\SI{3.3}{\second}$
    &
    $3.24$
    &
    $\SI{2.2}{\second}$
    &
    $\SI{0.46}{\second}$
    &
    $4.85$
    \\

    \addlinespace[0.5em]

    Oppelt (Software)
    &
    $\SI{3.3}{\second}$
    &
    $3.24$
    &
    $\SI{2.20}{\second}$
    &
    $\SI{0.462}{\second}$
    &
    $4.85$
    \\

    \addlinespace[0.5em]

    Rosenberg (manuell)~\cite{ref:op_ros_zieg}
    &
    $\SI{3.63}{\second}$
    &
    $3.68$
    &
    $\SI{2.2}{\second}$
    &
    $\SI{0.50}{\second}$
    &
    $4.85$
    \\

    \addlinespace[0.5em]

    Rosenberg (Software)
    &
    $\SI{3.63}{\second}$
    &
    $3.68$
    &
    $\SI{2.20}{\second}$
    &
    $\SI{0.495}{\second}$
    &
    $4.85$
    \\

    \addlinespace[0.5em]

    \pbox{84mm}{Phasengangmethode (manuell) \\ \small{\textbf{(16.3\% \"Uberschwingen)}} \\}
    &
    $\SI{3.29}{\second}$
    &
    $0.52$
    &
    $\SI{5.37}{\second}$
    &
    $\SI{0.41}{\second}$
    &
    $1.83$
    \\

    \addlinespace[0.5em]

    \pbox{84mm}{Phasengangmethode (Software) \\ \small{\textbf{(15\% \"Uberschwingen)}} \\}
    &
    $\SI{3.29}{\second}$
    &
    $0.47$
    &
    $\SI{5.74}{\second}$
    &
    $\SI{0.35}{\second}$
    &
    $1.78$
    \\

    \addlinespace[0.5em]

    \pbox{84mm}{Phasengangmethode (Software, $\varphi_r$ Standard) \\ \small{\textbf{(2\% \"Uberschwingen)}} \\}
    &
    $\SI{3.29}{\second}$
    &
    $0.17$
    &
    $\SI{5.74}{\second}$
    &
    $\SI{0.35}{\second}$
    &
    $0.76$
    \\

    \addlinespace[0.5em]

    \pbox{84mm}{Phasengangmethode (Software, Optimierungen positiv) \\ \small{\textbf{(2\% \"Uberschwingen)}} \\}
    &
    $\SI{7.50}{\second}$
    &
    $1.05$
    &
    $\SI{13.02}{\second}$
    &
    $\SI{0.636}{\second}$
    &
    $2.70$
    \\

    \addlinespace[0.5em]

    \pbox{84mm}{Phasengangmethode (Software, Optimierungen negativ) \\ \small{\textbf{(2\% \"Uberschwingen)}} \\}
    &
    $\SI{1.62}{\second}$
    &
    $0.058$
    &
    $\SI{5.74}{\second}$
    &
    $\SI{0.35}{\second}$
    &
    $0.76$
    \\

    \addlinespace[0.5em]


    \bottomrule
    \caption{%
        Zusammenfassung     und     Vergleich    der     mit     verschiedenen
        Methoden     berechneten    Regelparameter.
    }
\label{tab:allresults}
\end{longtable}


\begin{minipage}[c][][b]{.75\textwidth}
    \includegraphics[width=\textwidth]{images/comparisonPI015.png}
\end{minipage}
\begin{minipage}[c][][b]{.22\textwidth}
    \captionof{figure}{%
        Schrittantworten    des   geschlossenen    Regelkreises (PI-Regler):
        \textbf{Pink}: Chiens, Hrones, Reswick (0\% \"Uberschwingen),
        \textbf{dunkelgrau}: Chiens, Hrones, Reswick (20\% \"Uberschwingen),
        \textbf{blau}: Oppelt,
        \textbf{gr\"un}: Rosenberg,
        \textbf{rot}: Phasengangmethode.
    }
    \label{fig:comparisonPI015}
\end{minipage}

\begin{minipage}[c][][b]{.75\textwidth}
    \includegraphics[width=\textwidth]{images/comparisonPID015.png}
\end{minipage}
\begin{minipage}[c][][b]{.22\textwidth}
    \captionof{figure}{%
        Schrittantworten    des   geschlossenen    Regelkreises (PID-Regler):
        \textbf{Pink}: Chiens, Hrones, Reswick (0\% \"Uberschwingen),
        \textbf{dunkelgrau}: Chiens, Hrones, Reswick (20\% \"Uberschwingen),
        \textbf{blau}: Oppelt,
        \textbf{gr\"un}: Rosenberg,
        \textbf{rot}: Phasengangmethode.
    }
    \label{fig:comparisonPID015}
\end{minipage}

\begin{minipage}[c][][b]{.75\textwidth}
    \includegraphics[width=\textwidth]{images/comparisonPID002optimisations.png}
\end{minipage}
\begin{minipage}[c][][b]{.22\textwidth}
    \captionof{figure}{%
        Schrittantworten    des   geschlossenen    Regelkreises (PID-Regler),
        \"Uberschwingen auf 2\% begrenzt, mit Optimierungen:
        \textbf{rot}: Phasengangmethode (Standardwerte),
        \textbf{braun}: Phasengangmethode (positiv optimiert),
        \textbf{gelb}: Phasengangmethode (negativ optimiert).
    }
    \label{fig:comparisonPID002optimisations}
\end{minipage}
