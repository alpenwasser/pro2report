% was laeuft, was laeuft nicht, was kann verbessert werden
Das   Projekt   wurde   erfolgreich   abgeschlossen  und   alle   Punkte   des
urspr\"unglichen   Auftrages  erf\"ullt. Aus   den   Eingabewerten  wird   das
dynamische Verhalten  des geschlossenen  Regelkreises berechnet  und graphisch
dargestellt. Die Auftragserweiterungen wurden eingebaut, sodass man den Regler
mit  der  Phasengangmethode  dimensionieren  und  die  Knickfrequenz  und  das
\"Uberschwingen  manuell ver\"andert  werden kann. \\Die  vor der  Erweiterung
des  Auftrags  geplanten  graphische   Ausgabe  des  Amplitudengangs  und  der
Strecke sowie  die Schiebregler f\"ur  das Ver\"andern der  Reglerwerte wurden
weggelassen. Der Grund f\"ur diesen Entscheid  war die M\"oglichkeit durch die
hohe Rechenleistung den Regler echtzeit zu dimensioniern.

Als  Erweiterung  k\"onnten  die  Streckenwerte  ausgegeben  und  die  Strecke
gezeichnet   werden. Ausserdem   k\"onnte   das  automatische   Einlesen   der
Streckenwerte aus einer graphisch  Form hinzugef\"ugt werden. Das Tool w\"urde
an die  eingelesene Kurve die  Wendetangente legen und die  Werte ablesen. Das
Ablesen k\"onnte mit der Monte-Carlo-Studie verfeinert werden.
