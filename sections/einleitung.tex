% Ausgangslage
Im Rahmen  des Projektes soll  ein Tool  entwickelt werden, welches  einen PI-
respektive  einen  PID-Regler mittels  der  von  Jakob Zellweger  entwickelten
Phasengangmethode dimensioniert. Zum  Vergleich soll der  entsprechende Regler
ebenfalls mittels verschiedener Faustformeln berechnet werden.

% Anforderungen
Die Phasengangmethode ist eine graphische Methode, die bis anhin mit Stift und
Papier  durchgef\"uhrt wurde. Folglich  ist die  Ausf\"uhrung zeitaufw\"andig,
speziell   wenn   Schrittantworten   mit   unterschiedlichen   Parameterwerten
durchgespielt werden sollen, zudem kann das \"Uberschwingen nur grob gew\"ahlt
werden. Das Tool  soll ausgehend  von drei  Parametern aus  der Schrittantwort
der  Strecke  (Verst\"arkung  $K_s$,  Anstiegszeit  $T_g$,  Verz\"ogerungszeit
$T_u$)  mittels   der  Phasengangmethode  m\"oglichst   ideale  Regelparameter
berechnen,  sowie  die  Schrittantwort des  darauf  basierenden  geschlossenen
Regelkreises graphisch darstellen. Die  Benutzeroberfl\"ache der Software soll
intuitiv  sein, sodass  sich  auch  mit dem  Thema  nicht eingehend  vertraute
Regelungstechniker einfach zurechtfinden.

% Umsetzung
Die    erforderlichen    Algorithmen    wurden   zuerst    in    Matlab    als
Prototypen   implementiert    und   anschliessend   vollst\"andig    in   Java
konvertiert. Um  optimale Wartbarkeit,  \"Ubersichtlichkeit und  Modularit\"at
des    Codes     zu    gew\"ahrleisten,    ist    die     Software    gem\"ass
\emph{Model-View-Controller}-Pattern aufgebaut.

% Ergebnisse
Nach der Implementierung  in Matlab wurde klar, dass die  Berechnung durch die
hohe Rechenleistung  sehr schnell  durchgef\"uhrt werden  kann und  somit eine
Dimensionierung des geschlossenen Regelkreises anhand dieser Methode m\"oglich
ist. Die  Schrittantworten  k\"onnen   zur  Echtzeit  angezeigt  werden. Diese
M\"oglichkeit   wurde   genutzt,   indem  zwei   Schieberegler   implementiert
wurden. Durch den  einen kann das \"Uberschwingen  manuell eingestellt werden,
der andere erlaubt das weitere Anpassen  der Form der Kurve der Schrittantwort
an das gew\"unschte Ziel.

% Aufbau des Berichts
Der  Bericht   gliedert  sich  in   drei  Hauptteile: In  einem   ersten  Teil
werden  einige   theoretische  Grundlagen  der   Regelungstechnik  erl\"autert
(Abschnitt~\ref{sec:grundlagenRegtech}).

Abschnitt~\ref{sec:fachlicher_hintergrund}  geht genauer  auf die  eigentliche
Dimensionierung  von Reglern  ein, indem  ein PI-  und ein  PID-Regler mittels
Faustformeln  und Phasengangmethode  f\"ur eine  Beispielstrecke dimensioniert
werden. Die  Ergebnisse   werden  auch   mit  den  Resultaten   unseres  Tools
verglichen.

In    Abschnitt~\ref{sec:software}    wird    der    Aufbau    der    Software
erkl\"art;   wichtige  Algorithmen   sind   dabei  in   Anhang~\ref{app:algos}
genauer   dokumentiert. Dabei   werden   die  Benutzeroberfl\"ache   und   das
innere   Zusammenspiel  der   verschiedenen   Module   der  Software   genauer
betrachtet. Ebenfalls  sind  in   Abschnitt~\ref{sec:test}  einige  Worte  zum
Testverfahren zu finden.
