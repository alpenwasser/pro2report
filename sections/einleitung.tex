Das  Projekt besch\"aftigt  sich mit  den Problemen  der Regelungstechnik. Der
Auftrag beinhaltet ein Tool, welches aus  den eingegebenen Ks, Tg und Tu einen
PI- und ein PID-Regler dimensioniert wird.

Das  Tool soll  benutzerfreundlich, das  heisst auch  f\"ur einen  unge\"ubten
Regelungstechniker geeignet sein. Das  \"Uberschwingen der Schrittantwort soll
gew\"ahlt  werden  k\"onnen. Der  geschlossene   Regelkreis  soll  mit  seinem
dynamischen Verhalten dargestellt/visualisiert und berechnet werden.

F\"ur die  Berechung wird  neben den  Faustformeln auch  die Phasengangmethode
angewendet.   Die  Phasengangmethode  ist ein  graphisches  Werkzeug,  welches
normalerweise mit Bleistift, Papier und Lineal durchgef\"uhrt wird. Unser Ziel
ist  es,  dass  diese  Methode  durch  Implementierung  in  Java  eine  ebenso
brauchbare L\"osung liefert.

Nach  der Berechung  in  Matlab  wurde klar,  dass  die  Berechnung durch  die
hohe Rechenleistung  sehr schnell  durchgef\"uhrt werden  kann und  somit eine
Dimensionierung des geschlossenen Regelkreises anhand dieser Methode von Herrn
Zellweger m\"oglich ist.

Das Tool  ist in zwei  Bereiche aufgeteilt: Die  graphische Seite und  die der
Ein- und Ausgabe der Zahlen.
