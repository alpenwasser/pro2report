Ein Regler  kann mit unterschiedlichen Gleichungen  dargestellt werden.  F\"ur
die Berechnungen  in diesem  Projekt sind  zwei von  besonderer Bedeutung: Die
bodekonforme und die reglerkonforme Darstellung.

Dabei gilt  es zu  beachten, dass  sich die  einzelnen Darstellungen  in ihrer
mathematischen Gleichung  unterscheiden, jedoch den  selben Informationsinhalt
haben. Deshalb   ist  es   auch  m\"oglich,  mittels  einfacher   Umrechnungen
die  Darstellungsweise   zu  wechseln.

Es  existieren  zwei  haupts\"achliche   Gr\"unde  f\"ur  die  Verwendung  von
zwei  verschiedenen Darstellungsarten: Einerseits  geben gewisse  Berechnungen
automatisch  die  eine oder  andere  Form  zur\"uck, andererseits  bieten  die
beiden Varianten je  nach Situation bestimmte Vorteile,  was das Verst\"andnis
anbelangt.

Bei der  bodekonformen Darstellung wie  in Gleichungen~\ref{eq:pi:bodekonform}
und~\ref{eq:pid:bodekonform}   ist   die  \"Ubertragungsfunktion   leicht   zu
interpretieren,  was ihren  Amplitduden- und  Frequenzgang betrifft.   Deshalb
wird  die  bodekonforme  Darstellung  \"uberall dort  verwendet,  wo  mit  den
Reglerwerten Berechnungen ausgef\"uhrt werden.

\begin{equation} \label{eq:pi:bodekonform}
    H_{rpi} = K_{rk} \cdot \biggl[ \frac{1 + s \cdot T_{nk}}{s \cdot T_{nk}} \biggr]
\end{equation}

\begin{equation} \label{eq:pid:bodekonform}
    H_{rpid} = K_{rk} \cdot \biggl[ \frac{(1 + s \cdot T_{nk}) \cdot (1 + s \cdot T_{vk}) }{ s \cdot T_{nk} } \biggr]
\end{equation}


Die reglerkonforme  Darstellung aus  den Gleichungen~\ref{eq:pi:reglerkonform}
und~\ref{eq:pid:reglerkonform}  hingegen ist  so  aufgebaut,  dass die  Werte,
welche  an  einem realen  Regler  einstellbar  sind, direkt  abgelesen  werden
k\"onnen. Diese Darstellung  wird verwendet,  um die Reglerwerte  dem Benutzer
zur Verf\"ugung zu stellen.

\begin{equation} \label{eq:pi:reglerkonform}
    H_{rpi} = K_{r} \cdot \biggl[ 1 + \frac{1}{s \cdot T_{n}} \biggr]
\end{equation}

\begin{equation} \label{eq:pid:reglerkonform}
    H_{rpid} = K_{r} \cdot \biggl[ 1 + \frac{1}{s \cdot T_n} + \frac{s \cdot T_v}{s \cdot T_p} \biggr]
\end{equation}

Zudem   liegt   das   Ergebnis   der    Faustformeln   auch   immer   in   der
reglerkonformen  Darstellung  vor. F\"ur  die  Umrechnung  zwischen  den  zwei
Darstellungsarten ergeben sich durch  einen Koeffizientenvergleich die Formeln
in  Tabelle~\ref{tab:bode_regler_konform}.  F\"ur  die Berechnungen  in diesem
Projekt wird,  wenn nicht anders  angegeben, mit $T_p=\frac{1}{10}  \cdot T_v$
gerechnet.

Die algorithmischen Umsetzungen  dieser Formeln f\"ur unsere  Software sind in
Anhang~\ref{app:algo:regbode} und \ref{app:algo:boderegler} zu finden.

\begin{longtable}{l|ll}
    \toprule

    %\multicolumn{3}{l}{\large{\textsc{auftragsanalyse und hintergrundinformationen}}} \\
    %\multicolumn{2}{l}{pi}

    &
    bodekonform $\rightarrow$ reglerkonform
    &
    reglerkonform $\rightarrow$ bodekonform
    \\

    \midrule

    \endhead
    \endfoot
    \endlastfoot

    % content here ---------------------------------------------------------- %

    PI
    &
    $T_n = T_{nk} $ %reglerkonform
    &
    $K_{rk} = K_r $ %bodekonform
    \\

    \midrule

    PID
    &
    $T_n = T_{nk}+T_{vk}-T_p$
    &
    $T_{nk}=0.5 \cdot (T_n+T_p) \cdot (1+\epsilon)$
    \\

    &
    $T_v=\frac{T_{nk} \cdot T_{vk}}{T_{nk}+T_{vk}-T_p}-T_p$
    &
    $T_{vk}=0.5 \cdot (T_n+t_p) \cdot (1-\epsilon)$
    \\

    &
    %$k_r=k_{rk} \cdot \frac{1+t_{vk}}{t_{nk}}$
    $K_r=K_{rk} \cdot (1 + \frac{T_{vk}-T_p}{T_{nk}})$
    &
    $K_{rk} = 0.5 \cdot K_r \cdot (1 + \frac{T_p}{T_{nk}}) \cdot (1+\epsilon )$
    \\
    \\

    &
    \multicolumn{2}{l}{wobei $\epsilon^2 = 1-(4 \cdot T_n \cdot \frac{T_v-T_p}{(T_n+T_p)^2})$}
    \\
    \bottomrule
    \caption{Formeln zur Umrechung zwischen bode- zu reglerkonformer Darstellung \cite{regelungstechnik:zellweger}, \cite{regelungstechnik:schumleon}}
    \label{tab:bode_regler_konform}
\end{longtable}


