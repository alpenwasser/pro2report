Das  Kernst\"uck  dieser Arbeit  und  des  zugeh\"origen Softwaretools  stellt
die   so  genannte   ``Phasengangmethode   zur  Reglerdimensionierung''   von
Jakob   Zellweger   dar~\cite{regelungstechnik:zellweger_short}. Diese   wurde
urspr\"unglich  als  vereinfachte  grafische  Methode  zur  Approximation  der
-20dB/Dek Methode erarbeitet und im Rahmen dieses Projektes in einem Java-Tool
automatisiert. Als  Vergleich   wertet  die  Software  ebenfalls   einige  der
g\"angigen Faustformeln aus.

Das Tool f\"uhrt grob vereinfacht folgende Schritte aus:
\begin{itemize}
    \item
        Bestimmung des  Frequenzgangs der Regelstrecke  aus Verz\"ogerungszeit
        $T_u$,     Anstiegszeit     $T_g$      und     Verst\"arkung     $K_s$
        (Abschnitt~\ref{subs:frequenzgang})
    \item
        Dimensionierung       des      Reglers       mittels      Faustformeln
        (Abschnitt~\ref{subs:faustformeln})
    \item
        Dimensionierung    des    Reglers    durch    die    Phasengangmethode
        (Abschnitte~\ref{subs:phasengang:pi} und~\ref{subs:phasengang:pid})
    \item
        Umrechung   der   Regler-Darstellung    zwischen   bodekonformer   und
        reglerkonformer Darstellung (Abschnitt~\ref{subs:bode_regler})
    \item
        Berechnung   der   Schrittantwort   des   geschlossenen   Regelkreises
        (Abschnitt~\ref{subs:geschlossen})\todo{reference abschnitt anita}
\end{itemize}

Im   folgenden   Kapitel   wird   auf   diese   Punkte   genauer   eingegangen
und   das    Vorgehen   anhand    eines   konkreten    Beispiels   rechnerisch
und    grafisch   erl\"autert. Die    Durchrechnung   der    Phasengangmethode
orientiert     sich     an     den     Rezepten,     welche     bereits     im
fachlichen     Teil     des     Pflichtenheftes    dieses     Projektes     zu
finden  sind~\cite{ref:pflichtenheft}. Genauere  Hintergrundinformationen  zur
Phasengangmethode  selbst  sind  dem  Vorlesungs-Skript  von  J. Zellweger  zu
entnehmen~\cite{regelungstechnik:zellweger_short}.

Das  \"Uberschwingverhalten  kann  im  Software-Tool vom  Benutzer  auf  einen
Zielwert  zwischen 0\%  und 30\%  eingestellt werden. Das  Tool optimiert  den
resultierenden Regler  dann entsprechend,  um dieser Vorgabe  m\"oglichst nahe
zu  kommen. Dazu  wird die  Reglerverst\"arkung  $K_{rk}$  angepasst, bis  ein
passendes Resultat erzielt ist.

\clearpage
\subsection{Frequenzgang der Regelstrecke}
\label{subs:frequenzgang}
Als  Ausgangspunkt  der  Reglerdimensionierung dient  die  Schrittantwort  der
Strecke. Durch  Einzeichnen  der  Wendetangente\footnotemark[1]  ergeben  sich
Schnittpunkte  der  Wendetangente mit  der  Zeitachse  $[T_u,0]$ und  mit  dem
Zielwert $[T_u+T_g,K_s]$.   Es k\"onnen nun also  die Verz\"ogerungszeit $T_u$
und  die  Anstiegszeit   $T_g$  aus  Abbildung~\ref{fig:plant_step}  abgelesen
werden.

\footnotetext[1]{%
    Die Wendetangante ist die Tangente an den Wendepunkt in der Anstiegs-Phase
    der Schrittantwort.
}

Wir werden in diesem Bericht folgende Strecke als Beispiel nehmen:
\begin{figure}[h! width=\pagewidth]
    \includegraphics[width=\textwidth]{images/streckeSchrittantwort.png}
    \caption{%
    Schrittantwort der  Beispielstrecke (schwarz), Wendetangende  (rot), $T_u$
    und $T_g$ (blau)
    }
    \label{fig:plant_step}
\end{figure}

Ausmessen der Schrittantwort ergibt:
\begin{itemize}
    \item
        $K_s = 2$\footnotemark[2]
    \item
        $T_u = \SI{1.1}{\second}$
    \item
        $T_g = \SI{8.9}{\second}$
\end{itemize}

\footnotetext[2]{%
    Abbildung~\ref{fig:plant_step}  ist  auf  1  normiert,  die  Verst\"arkung
    unserer  Beispielstrecke   betr\"agt  $2$.    An  den  Werten   f\"ur  die
    Verz\"ogerungs-  und Anstiegszeit  oder  am  Ausmessen der  Schrittantwort
    \"andert sich dadurch nichts
}

Da  die  Reglerdimensionierung  mit  der  Phasengangmethode  vom  Frequenzgang
einer  Strecke  ausgeht und  nicht  von  deren  Schrittantwort, wird  aus  den
obigen Werten  nun der Frequenzgang  der Strecke bestimmt.  Dies  erledigt die
Methode  \code{p\_sani()}\footnotemark[3],  welche  uns die  Werte  f\"ur  die
\"Ubertragungsfunktion  der  Strecke liefert.

\footnotetext[3]{%
    Die  Methode  \code{p\_sani()} wurde  zu  Beginn  des Projektes  in  einer
    Matlab-Implementation  vom  Auftraggeber   zur  Verf\"ugung  gestellt  und
    anschliessend f\"ur unser Tool in Java \"ubersetzt.

    Sie kann aus der Verz\"ogerungszeit, der Anstiegszeit und der Verst\"arkung
    der Strecke ein Polynom f\"ur deren \"Ubertragungsfunktion vom Grad 1 bis 8
    ausrechnen.

    Als Eingabeparameter werden  die Werte $T_u$, $T_g$  und $K_s$ ben\"otigt,
    als R\"uckgabewert erh\"alt  man ein Array mit den Zeiten  $T_i$ f\"ur die
    Nenner der Faktoren des Polynoms (siehe Gleichung~\ref{eq:transfer:plant}).

    Genauere  Informationen  zur   Funktionsweise  von  \code{p\_sani()}  sind
    Anhang~\ref{app:sani} zu entnehmen.
}

\clearpage
In unserem Fall ergibt dies folgendes Polynom:

\begin{gather} \label{eq:transfer:plant}
    \begin{split}
        H_s (s) & = K_s
                  \cdot \frac{1}{1 + s \cdot T_1}
                  \cdot \frac{1}{1 + s \cdot T_2}
                  \cdot \frac{1}{1 + s \cdot T_2}                     \\
                & = 2
                  \cdot \frac{1}{1 + s \cdot \SI{0.4134}{\second}}
                  \cdot \frac{1}{1 + s \cdot \SI{1.4894}{\second}}
                  \cdot \frac{1}{1 + s \cdot \SI{5.3655}{\second}}
    \end{split}
\end{gather}

Mit einem geeigneten Tool (z.B. Matlab) kann man sich den dazugeh\"origen Plot
(Abbildung~\ref{fig:plant_freq}) erstellen lassen.

\begin{figure}[h! width=\pagewidth]
    \includegraphics[width=\textwidth]{images/streckeFrequenzgang.png}
    \caption{%
        Frequenzgang der Strecke, mit Amplitudengang und Phasengang
    }
    \label{fig:plant_freq}
\end{figure}

Somit ist der Frequenzgang der Strecke bekannt und man hat alle erforderlichen
Informationen, um den Regler mit der Phasengangmethode zu dimensionieren.


\subsection{Reglerdimensionierung mittels Faustformeln}
\label{subs:faustformeln}
Im   Praxiseinsatz  stehen   f\"ur   die  Dimensieung   der  Regler   einfache
Berechnungsformeln f\"ur die Einstellwerte der  Regler anhand von $T_u$, $T_g$
und $K_s$ zur Verf\"ugung.

Einige   dieser  Faustformeln   werden  in   der  Applikation   zum  Vergleich
mitberechnet. Die    dazugeh\"origen    Berechnungen    sind    der    Tabelle
\ref{tab:faustformeln} zu entnehmen.

\begin{longtable}{p{50mm}rrrrr}
    \toprule

    %\multicolumn{3}{l}{\large{\textsc{Auftragsanalyse und Hintergrundinformationen}}} \\

    Faustformel
    &
    \multicolumn{2}{l}{PI-Regler}
    &
    \multicolumn{2}{l}{PID-T1-Regler}
    \\

    &
    $T_n$
    &
    $K_p$
    &
    $T_n$
    &
    $T_v$
    &
    $K_p$
    \\

    \midrule

    \endhead
    \endfoot
    \endlastfoot

    % CONTENT HERE ---------------------------------------------------------- %

    \pbox{45mm}{Chiens, Hrones, Reswick \\ \small{\textbf{(0\% \"Uberschwingen)}} \\ \cite{ref:chiens_tsn}, \cite{ref:chiens_wiki}}
    &
    $1.2\cdot T_g$
    &
    $\frac{0.35}{K_s} \cdot \frac{T_g}{T_u}$
    &
    $T_g$
    &
    $0.5\cdot T_u$
    &
    $ \frac{0.6}{K_s} \cdot \frac{T_g}{T_u} $
    \\

    \addlinespace[1em]

    \pbox{45mm}{Chiens, Hrones, Reswick \small{\textbf{(20\% \"Uberschwingen)}} \\ \cite{ref:chiens_tsn}, \cite{ref:chiens_wiki}}
    &
    $T_g$
    &
    $\frac{0.6}{K_s} \cdot \frac{T_g}{T_u}$
    &
    $1.35\cdot T_g$
    &
    $0.47 \cdot T_u$
    &
    $ \frac{0.95}{K_s} \cdot \frac{T_g}{T_u} $
    \\

    \addlinespace[1em]

    Oppelt \cite{ref:op_ros_zieg}
    &
    $3 \cdot T_u$
    &
    $\frac{0.8}{K_s} \cdot \frac{T_g}{T_u}$
    &
    $2 \cdot T_u$
    &
    $ 0.42 \cdot T_u $
    &
    $ \frac{1.2}{K_s} \cdot \frac{T_g}{T_u} $
    \\

    \addlinespace[1em]

    Rosenberg \cite{ref:op_ros_zieg}
    &
    $3.3 \cdot T_u $
    &
    $ \frac{0.91}{K_s} \cdot \frac{T_g}{T_u} $
    &
    $ 2 \cdot T_u $
    &
    $ 0.45 \cdot T_u $
    &
    $ \frac{1.2}{T_s} \cdot \frac{T_g}{T_u}$
    \\

    \addlinespace[1em]

    Ziegler/Nichols \cite{ref:op_ros_zieg}
    &
    $ 3.33 \cdot T_u $
    &
    $ \frac{0.9}{K_s} \cdot \frac{T_g}{T_u} $
    &
    $ 2 \cdot T_u $
    &
    $ 0.5 \cdot T_u $
    &
    $ \frac{1.2}{K_s} \cdot \frac{T_g}{t_u} $
    \\

    \bottomrule
\caption{Faustformeln zur Reglerdimensionierung}
\label{tab:faustformeln}
\end{longtable}
\todo{Quellen hinzuf\"ugen}


\subsection{Reglerdimensionierung mittels Phasengangmethode: PI-Regler}
\label{subs:phasengang:pi}
Es  werden nun  anhand  der  Phasengangmethode sowohl  ein  PI-  wie auch  ein
PID-Regler f\"ur die in Abschnitt~\ref{subs:frequenzgang} ausgemessene Strecke
dimensioniert (siehe n\"achster Abschnitt f\"ur PID-Regler).

Tabelle~\ref{tab:terms}  fasst  die  h\"aufig verwendeten  Begriffe  in  einer
\"Ubersicht zusammen:

\begin{longtable}{lp{60mm}}
    \toprule
    \endhead
    \endfoot
    \endlastfoot

    % CONTENT HERE ---------------------------------------------------------- %

    $H_s(j\omega)                                                                   $ &  \"Ubertragungsfunktion der Regelstrecke \\
    $A_s(j\omega)=|H_s(j\omega)|                                                    $ &  Amplitudengang der Regelstrecke \\
    $\varphi_s(j\omega)=arg(H_s(j\omega))                                           $ &  Phasengang der Regelstrecke \\
    $H_r(j\omega)                                                                   $ &  \"Ubertragungsfunktion des Reglers \\
    $A_r(j\omega)=|H_r(j\omega)|                                                    $ &  Amplitudengang des Reglers \\
    $\varphi_r(j\omega)=arg(H_r(j\omega))                                           $ &  Phasengang des Reglers \\
    $H_o(j\omega)=H_s \cdot H_r(j\omega)                                            $ &  \"Ubertragungsfunktion des offenen Regelkreises \\
    $A_o(j\omega)=|H_o(j\omega)|                                                    $ &  Amplitudengang des offenen Regelkreises \\
    $\varphi_o(j\omega)=arg(H_o(j\omega))=\varphi_s(j\omega)+\varphi_r(j\omega)     $ &  Phasengang des offenen Regelkreises \\
    $H_{rpid}= K_{rk}\Big[ \frac{(1+sT_{nk})(1+sT_{vk})}{sT_{nk}}\Big]              $ & \"Ubertragungsfunktion des PID-Reglers \\
    $H_{rpi} = K_{rk}\Big[ 1 + \frac{1}{sT_{nk}} \Big]                              $ & \"Ubertragungsfunktion des PI-Reglers \\

    \bottomrule
    \caption{Die wichtigsten Begriffsdefinitionen}
    \label{tab:terms}
\end{longtable}


% ---------------------------------------------------------------------------- %
\subsubsection*{Ziel}
% ---------------------------------------------------------------------------- %
Das  Ziel ist  die  Bestimmung  der Parameter  $K_{rk}$  und  $T_{nk}$ in  der
\"Ubertragungsfunktion des Reglers:

\begin{equation} \label{eq:pi:target}
    H_{rpi} = K_{rk} \cdot \biggl[ 1 + \frac{1}{s \cdot T_{nk}} \biggr]
\end{equation}


% ---------------------------------------------------------------------------- %
\subsubsection{Bestimmung der Reglerfrequenz $\mathbf{\boldsymbol{\omega}_{pi}}$}
% ---------------------------------------------------------------------------- %

Zuerst  wird  im  Phasengang  der  Strecke  gem\"ass  Formel~\ref{eq:pi:phi_s}
die     Frequenz     $\omega_{pi}$      bestimmt,     f\"ur     welche     die
Phase     der    Strecke     $-90\degree$     betr\"agt,    ersichtlich     in
Abbildung~\ref{fig:pi:omega_pi}\footnotemark[4].

\begin{equation} \label{eq:pi:phi_s}
    \varphi_s(\omega_{pi}) = -90 \degree
\end{equation}

\footnotetext[4]{%
    Der  Winkel  stellt  keinen   endg\"ultigen  Wert  dar. Dieser  wurde  von
    Jakob  Zellweger  fixiert, um  eine  grafische  Evaluation \"uberhaupt  zu
    erm\"oglichen. Durch Anpassung dieses Wertes kann je nach Regelstrecke das
    Regelverhalten weiter optimiert werden.
}

\begin{figure}[h! width=\pagewidth]
    \includegraphics[width=\textwidth]{images/piStreckeOmegaPI.png}
    \caption{%
        Amplituden- und  Phasengang der Strecke mit  $\omega_{pi}$ eingetragen
        (vertikale gestrichelte Linie).
    }
    \label{fig:pi:omega_pi}
\end{figure}

Wie    man   aus    Abbildung~\ref{fig:pi:omega_pi}   ablesen    kann,   liegt
dieser   Wert  f\"ur   $\omega_{pi}$  in   unserem  Beispiel   bei  ungef\"ahr
$\SI{0.3}{\per\second}$. Die Kontrollrechnung mittels Matlab ergibt:

\begin{equation} \label{eq:pi:omega_pi}
    \omega_{pi} = \SI{0.3039}{\per\second}
\end{equation}


% ---------------------------------------------------------------------------- %
\subsubsection{Bestimmung von $\mathbf{T_{nk}}$}
% ---------------------------------------------------------------------------- %
Damit kann nun $T_{nk}$ direkt berechnet werden\footnotemark[5]:

\begin{equation} \label{eq:pi:omega_pi}
    T_{nk} = \frac{1}{\omega_{pi}} = \frac{1}{\SI{0.3039}{\per\second}} = \SI{3.2902}{\second}
\end{equation}

\footnotetext[5]{%
    Um die  Akkumulation von Ungenauigkeiten  zu minimieren, werden bei diesen
    Berechnungen  die  genauen  Werte  aus  Matlab  verwendet  und  nicht  die
    gerundeten  Zwischenresultate,  was  zu   Abweichungen  zu  den  von  Hand
    berechneten Ergebnissen f\"uhren kann.
}


% ---------------------------------------------------------------------------- %
\subsubsection{Bestimmung der Durchtrittsfrequenz $\mathbf{\boldsymbol{\omega}_d}$}
% ---------------------------------------------------------------------------- %

Die   Durchtrittsfrequenz  ist   die   Frequenz,  bei   der  die   betrachtete
\"Ubertragungsfunktion $H(j\omega)$ eine Verst\"arkung von $\SI{0}{\decibel} =
1$  aufweist. In der  Phasengangmethode soll  sie so  festgelegt werden,  dass
der  offene  Regelkreis  Gleichung~\ref{eq:phi_o} erf\"ullt. Dabei  ist  f\"ur
$\varphi_s$ abh\"angig  vom gew\"unschten \"Uberschwingverhalten ein  Wert aus
Tabelle~\ref{tab:phi_s}  auszuw\"ahlen\footnotemark[6].   Nach dem  Festlegen
der Durchtrittsfrequenz  wird dann  im n\"achsten Abschnitt  die Verst\"arkung
des Reglers noch angepasst.

\footnotetext[6]{%
    Die  Werte f\"ur  $\varphi_s$  aus  Tabelle~\ref{tab:phi_s} stellen  keine
    abschliessende  Auflistung dar  und  sind lediglich  als Anhaltspunkte  zu
    betrachten. Weicht das Verhalten des geschlossenen Regelkreises am Schluss
    zu stark  vom gew\"unschten  Ergebnis ab, besteht  durch die  Wahl anderer
    Werte f\"ur $\varphi_s$ die M\"oglichkeit weiterer Optimierung.
}

\begin{equation} \label{eq:phi_o}
    \varphi_o(\omega_d)=\varphi_s.
\end{equation}

\begin{longtable}{llll}
    \toprule
    \endhead
    \endfoot
    \endlastfoot

    % CONTENT HERE ---------------------------------------------------------- %

    \"Uberschwingen & 0\%              & 16.3\%           & 23.3\% \\
    $\varphi_s$        & $-103.7 \degree$ & $-128.5 \degree$ & $-135 \degree$ \\

    \bottomrule
    \caption{Werte f\"ur $\varphi_s$}
    \label{tab:phi_s}
\end{longtable}

Um  Gleichung~\ref{eq:phi_o}  auswerten  zu   k\"onnen,  wird  der  Phasengang
des    \emph{offenen    Regelkreises}    ben\"otigt. Dazu    wird    der    in
Gleichung~\ref{eq:pi:omega_pi}    erhaltene    Wert    f\"ur    $T_{nk}$    in
die   \"Ubertragungsfunktion    des   Reglers   (Gleichung~\ref{eq:pi:target})
eingesetzt. $K_{rk}$  ist  noch  unbekannt,  hat aber  auf  die  Phase  keinen
Einfluss und wird somit vorerst einfach auf 1 gesetzt.

\begin{gather} \label{eq:pi:target:inserted}
    \begin{split}
        H_{rpi} & = K_{rk} \cdot \biggl[ 1 + \frac{1}{s \cdot T_{nk}} \biggr] \\
                & = 1      \cdot \biggl[ 1 + \frac{1}{s \cdot \SI{3.2902}{\second}} \biggr]
    \end{split}
\end{gather}

Daraus    kann    nun    der    Frequenzgang    des    offenen    Regelkreises
(\"Ubertragungsfunktion $H_o$,  Amplitudengang $A_o$,  Phasengang $\varphi_o$)
identifiziert werden.

\begin{gather} \label{eq:pi:h_open}
    \begin{split}
        H_o (s) & = H_{rpi} (s) \cdot H_s (s) \\
            & = \Biggl(
                    K_{rk} \cdot \biggl[ 1 + \frac{1}{s \cdot T_{nk}} \biggr]
                \Biggr)
                \cdot
                K_s
                \cdot
                \Biggl(
                        \frac{1}{1 + s \cdot T_1}
                  \cdot \frac{1}{1 + s \cdot T_2}
                  \cdot \frac{1}{1 + s \cdot T_2}
                \Biggr) \\
            & = \Biggl(
                    1 \cdot \biggl[ 1 + \frac{1}{s \cdot \SI{3.2902}{\second}} \biggr]
                \Biggr)
                \cdot
                2
                \cdot
                \Biggl(
                          \frac{1}{1 + s \cdot \SI{0.4134}{\second}}
                    \cdot \frac{1}{1 + s \cdot \SI{1.4894}{\second}}
                    \cdot \frac{1}{1 + s \cdot \SI{5.3655}{\second}}
                \Biggr)
    \end{split}
\end{gather}


Von  besonderem  Interesse  ist  der  Phasengang  $\varphi_o(j\omega)$  dieser
\"Ubertragungsfunktion. Wie   Anfangs   spezifiziert,    soll   ein   maximals
\"Uberschwingen  von   ca. $16.3\%$  angestrebt  werden. Dazu   muss  gem\"ass
Tabelle~\ref{tab:phi_s}  die Durchtrittsfrequenz  $\omega_d$ gefunden  werden,
an  welcher der  offene  Regelkreis eine  Phase  von $-128.5\degree$  aufweist
(Gleichung~\ref{eq:phi_o}). In   Abbildung~\ref{fig:pi:omega_d}    kann   dies
grafisch verifiziert werden.

\begin{figure}[h! width=\pagewidth]
    \includegraphics[width=\textwidth]{images/piOffenerRegelkreisPhasengang.png}
    \caption{%
        Phasengang   $\varphi_o(j\omega)$   des   offenen   Regelkreises   mit
        eingetragener Durchtrittsfrequenz $\omega_{d}$ (vertikale gestrichelte
        Linie). Wie man  sieht, weist der offene  Regelkreis unseres Beispiels
        bei dieser Kreisfrequenz eine Phase von $-128.5\degree$ auf
        (etwa $\SI{-2.24}{\radian}$).
    }
    \label{fig:pi:omega_d}
\end{figure}

Dies ergibt:
\begin{equation} \label{eq:pi:omega_d}
    \omega_d = \SI{0.2329}{\per\second}
\end{equation}


% ---------------------------------------------------------------------------- %
\subsubsection{Bestimmung der Reglerverst\"arkung $\mathbf{K_{rk}}$}
% ---------------------------------------------------------------------------- %

Im  letzten  Schritt  muss  nun, wie im  vorherigen  Abschnitt  erw\"ahnt, die
Verst\"arkung  $K_{rk}$   des  Reglers   noch  angepasst  werden,   damit  der
offene   Regelkreis  bei   der  angestrebten   Durchtrittsfrequenz  $\omega_d$
auch  effektiv  eine Verst\"arkung  von  1  aufweist.  Dazu  wird  $j\omega_d$
in  Gleichung~\ref{eq:pi:h_open}  f\"ur  den   Parameter  $s$  eingesetzt  und
$|H_o(j\omega_d)| = 1$ gesetzt.

\begin{gather} \label{eq:pi:A_o_set_to_one}
    \begin{split}
        A_o & = | H_o (j\omega_d) | = | H_{rpi} (j\omega) \cdot H_s (j\omega)| \\
            & = \abs*{
                    \bigg(
                        K_{rk} \cdot \biggl[ 1 + \frac{1}{j \cdot \omega_d \cdot T_{nk}} \biggr]
                    \bigg)
                    \cdot
                    K_s
                    \cdot
                    \bigg(
                            \frac{1}{1 + j \cdot \omega_d \cdot T_1}
                      \cdot \frac{1}{1 + j \cdot \omega_d \cdot T_2}
                      \cdot \frac{1}{1 + j \cdot \omega_d \cdot T_2}
                      \bigg)} \\
              & = 1
    \end{split}
\end{gather}

Mit den Werten
\begin{equation} \label{eq:pi:values}
    \begin{split}
        K_s      & = 2                    \\
        T_{nk}   & = \SI{3.2902}{\second} \\
        T_1      & = \SI{0.4134}{\second} \\
        T_2      & = \SI{1.4894}{\second} \\
        T_3      & = \SI{5.3655}{\second} \\
        \omega_d & = \SI{0.2329}{\radian\per\second}
    \end{split}
\end{equation}

l\"ost  man Gleichung  \ref{eq:pi:A_o_set_to_one}  nun nach  $K_{rk}$ auf  und
erh\"alt:

\begin{equation} \label{eq:pi:k_rk_result}
    K_{rk} = 0.517577
\end{equation}


% ---------------------------------------------------------------------------- %
\subsubsection{Resultat}
% ---------------------------------------------------------------------------- %

Somit ist der PI-Regler vollst\"andig bestimmt und hat folgende Form:

\begin{equation} \label{eq:pi:result}
    H_{rpi} = 0.518 \cdot \biggl[ 1 + \frac{1}{s \cdot \SI{3.29}{\second}} \biggr]
\end{equation}

In Abbildung~\ref{fig:pi:all} sind die  wichtigsten Werte f\"ur diesen Prozess
nochmals in einer \"Ubersicht zusammengefasst.

\begin{figure}[h! width=\pagewidth]
    \includegraphics[width=\textwidth]{images/piBode.png}
    \caption{%
        Frequenzgang des Reglers (gr\"un), der  Strecke (blau) und des offenen
        Regelkreises (rot).
    }
    \label{fig:pi:all}
\end{figure}


\subsection{Reglerdimensionierung mittels Phasengangmethode: PID-Regler}
\label{subs:phasengang:pid}
\subsubsection*{Ziel}
Das Ziel ist  die Bestimmung der Parameter $K_{rk}$, $T_{nk}$  und $T_{vk}$ in
der \"Ubertragungsfunktion des Reglers:

\begin{equation} \label{eq:pid:target}
    H_{rpid} = K_{rk} \cdot \biggl[ \frac{(1 + s \cdot T_{nk}) \cdot (1 + s \cdot T_{vk}) }{ s \cdot T_{nk} } \biggr]
\end{equation}


\subsubsection{Bestimmung der Reglerfrequenz $\omega_{pid}$}

Analog  zum PI-Regler  wird  zuerst  im Phasengang  der  Strecke die  Frequenz
$\omega_{pid}$  bestimmt,  f\"ur  welche   die  Phase  einen  bestimmten  Wert
aufweist, nur wird hier $-135\degree$ benutzt~\footnotemark[7]:

\begin{equation} \label{eq:pid:phi_s}
    \varphi_s(\omega_{pid}) = -135 \degree
\end{equation}

\footnotetext[7]{%
    Wie   auch  beim   PI-Regler   stellt  diese   Frequenz  lediglich   einen
    Ausgangspunkt dar  und kann  zur weiteren  Optimierung des  Resultats noch
    angepasst werden.
}

In unserem Beispiel ergibt dies:

\begin{equation} \label{eq:pid:omega_pid}
    \omega_{pid} = \SI{0.6714}{\per\second}
\end{equation}

Eine       grafische        \"Uberpr\"ufung       kann        anhand       von
Abbildung~\ref{fig:pid:complete} durchgef\"uhrt werden.


\subsubsection{Steigung des Phasengangs bei der Reglerfrequenz}

Anschliessend wird  die Steigung des  Phasengangs $\varphi_s$ der  Strecke bei
der  Frequenz  $\omega_{pid}$  bestimmt. Ausgangspunkt  daf\"ur  ist  die  von
\code{p\_sani} bestimmte  \"Ubertragungsfunktion der Strecke  (siehe Gleichung
\ref{eq:transfer:plant}).

\begin{equation} \label{eq:transfer:plant:derivative}
    \frac{d\varphi_s}{d\omega} \biggr \rvert_{\omega=\omega_{pid}}
        = \frac{d(arg(H_s(j\omega)))}{d\omega} \biggr \rvert_{\omega=\omega_{pid}}
        = \SI{-1.5124}{\second}
\end{equation}
\todo{Einheit \"uberpr\"ufen}


\subsubsection{Hilfsparameter $\beta$}

Zwischen  den Steigungen  der Phasen  des offenen  Regelkreises ($\varphi_o$),
der  Strecke  ($\varphi_s$)  und   des  Reglers  ($\varphi_r$)  gilt  gem\"ass
Tabelle~\ref{tab:terms} folgende Beziehung:

\begin{equation} \label{eq:pid:phi_sum}
    \varphi_o = \varphi_s + \varphi_r
\end{equation}

Da die Ableitung eine lineare Funktion ist, gilt somit auch:
\begin{equation} \label{eq:pid:dphi_sum}
    \frac{d\varphi_o}{d\omega} = \frac{d\varphi_s}{d\omega} + \frac{d\varphi_r}{d\omega}
\end{equation}

Diese Beziehungen  k\"onnen auch  gut in Abbildung  \ref{fig:pid:complete} von
Hand \"uberpr\"uft werden.

Es soll nun gelten:

\begin{equation} \label{eq:pid:dphi_o_target}
    \frac{d\varphi_o}{d\omega} \biggr \rvert_{\omega=\omega_{pid}} = - \frac{1}{2}
\end{equation}

Da   $\frac{d\varphi_s}{d\omega}$  durch   die  Strecke   gegeben  und   somit
unver\"anderlich ist, kann lediglich der Wert von $\frac{d\varphi_r}{d\omega}$
angepasst werden, damit Gleichung~\ref{eq:pid:dphi_o_target} erf\"ullt wird.

Dazu f\"uhrt man den Hilfsparameter $\beta$ ein, f\"ur den gilt:

\begin{gather} \label{eq:pid:beta:start}
    \begin{split}
        \frac{1}{T_{vk}} & = \frac{\omega_{pid}}{\beta} \\
        \frac{1}{T_{nk}} & = \omega_{pid} \cdot \beta  \\
                       0 & <  \beta \leq 1
    \end{split}
\end{gather}

Wie in Abbildung ~\ref{fig:pid:complete} gesehen werden kann~\footnotemark[8],
liegen  die   beiden  Frequenzen  $\frac{1}{T_{vk}}$   und  $\frac{1}{T_{nk}}$
symmetrisch um  den Faktor $\beta$ respektive  $\frac{1}{\beta}$ oberhalb bzw.
unterhalb der Frequenz $\omega_{pid}$.

\footnotetext[8]{%
    Man  beachte  dabei,  dass  der Plot  logarithmisch  skaliert  ist.   Eine
    identische Wegstrecke  zwischen zwei  Punkte-Paaren auf  der Frequenzachse
    bedeutet also,  dass diese um denselben  \emph{Faktor} auseinander liegen,
    und nicht,  dass die Differenz  zwischen den jeweiligen  Punkten identisch
    ist. Im  Falle  der  Punkte-Paare $[\frac{1}{T_{nk}},  \omega_{pid}]$  und
    $[\omega_{pid}, \frac{1}{T_{vk}}]$  ist  dieser  Faktor  $\beta$,  wie  in
    Gleichung~\ref{eq:pid:beta:start} ersichtlich.
}


Will man $\beta$ von Hand  berechnen, trifft man zuerst eine ``vern\"unftige''
Annahme, zum Beispiel:

\begin{equation} \label{eq:pid:beta:initial_value}
    \beta = 0.5
\end{equation}

Mit  diesem Startwert  bestimmt  man nun  $T_{nk}$  und ${T_{vk}}$:

\begin{gather} \label{eq:pid:t_nk_t_vk_initial_results}
    \begin{split}
        {T_{vk}} & = \frac{\beta}{\omega_{pid}}  = \frac{0.5}{\SI{0.6714}{\per\second}}                   = \SI{0.7447}{\second} \\
        {T_{nk}} & = \frac{1}{\omega_{pid} \cdot \beta} = \frac{1}{\SI{0.6714}{\per\second} \cdot 0.5 }  = \SI{2.9789}{\second} \\
    \end{split}
\end{gather}

Die  somit  erhaltenen  Werte   setzt  man  in  Gleichung  \ref{eq:pid:target}
ein,   zusammen   mit   dem    Wert   f\"ur   $\omega_{pid}$   aus   Gleichung
\ref{eq:pid:omega_pid}. Da  $K_{rk}$   noch  unbekannt   ist,  aber   auf  den
Phasengang  keinen  Einfluss   hat,  setzt  man  vorerst  $K_{rk}   =  1$,  um
weiterrechnen zu k\"onnen.

\begin{gather} \label{eq:pid:t_nk_t_vk_initial_results}
    \begin{split}
        H_{rpid} & = K_{rk} \cdot \biggl[ \frac{(1 + j\omega \cdot T_{nk}) \cdot (1 + j\omega \cdot T_{vk}) }{ j\omega \cdot T_{nk} } \biggr] \\
                 & = 1      \cdot \biggl[ \frac{(1 + j\omega \cdot \SI{2.9789}{\second}) \cdot (1 + j\omega \cdot \SI{0.7447}{\second}) }{ j\omega \cdot  \SI{2.9789}{\second}} \biggr]
    \end{split}
\end{gather}

Von dieser Gleichung bestimmt man nun  den Phasengang und wertet danach dessen
Ableitung an der Stelle $\omega = \omega_{pid}$ aus. Die zugeh\"orige Rechnung
kann in Anhang~\ref{app:beta} gefunden werden.

\begin{gather} \label{eq:pid:phi_r_first_iteration}
    \begin{split}
        \varphi_r (j\omega)                                            & = arg(H_{rpid}(j\omega))        \\
        \frac{d\varphi_r}{d\omega} \biggr \rvert_{\omega=\omega_{pid}} & = \SI{1.1920}{\second}
    \end{split}
\end{gather}


Setzt man dies in Gleichung \ref{eq:pid:phi_sum} ein, erh\"alt man:
\begin{gather} \label{eq:pid:phi_sum_result_iteration_one}
    \begin{split}
    \frac{d\varphi_o}{d\omega}       \biggr \rvert_{\omega=\omega_{pid}, \beta=0.5}
        & = \frac{d\varphi_s}{d\omega} \biggr \rvert_{\omega=\omega_{pid}}
        + \frac{d\varphi_r}{d\omega} \biggr \rvert_{\omega=\omega_{pid}, \beta=0.5} \\
        & = \SI{-1.5124}{\second} + \SI{1.1920}{\second} \\
        & = \SI{-0.3204}{\second} \\
        & > -\frac{1}{2}
    \end{split}
\end{gather}

Mit  $\beta  = 0.5$  erh\"alt  man  also eine  zu  hohe  Steigung des  offenen
Regelkreises   an    der   Stelle  $\omega_{pid}$,   folglich   muss   $\beta$
\emph{verkleinert} werden.   Diese Berechnungen  werden nun mit  jeweils neuen
Werten  f\"ur  $\beta$  solange  wiederholt,  bis  die  Steigung  des  offenen
Regelkreises die gew\"unschte N\"ahe zu $-\frac{1}{2}$ aufweist.

Da die manuelle Iterierung dieses Prozesses enorm viel Zeit in Anspruch nimmt,
bietet  sich  hier  eine  Automatisierung  an. Die  Berechnung  mittels  eines
geeigneten Algorithmus in Matlab liefert schlussendlich folgendes Ergebnis:

\begin{gather} \label{eq:pid:beta_result}
    \begin{split}
        \beta    & = 0.2776 \\
        {T_{vk}} & = \frac{\beta}{\omega_{pid}}           = \SI{0.4134}{\second} \\
        {T_{nk}} & = \frac{1}{\omega_{pid} \cdot \beta}   = \SI{5.3656}{\second} \\
    \end{split}
\end{gather}

Diese Werte sind in ebenfalls in~\ref{fig:pid:complete} eingetragen.

Sollte  man  f\"ur  $\beta$  einen komplexen  Wert  erhalten,  wird  $\beta=1$
gesetzt.


\subsubsection{Durchtrittsfrequenz $\omega_d$}

Als  letzte Unbekannte  verbleibt  die Verst\"arkung  $K_{rk}$. Wie auch  beim
PI-Regler ist zum Finden  der Verst\"arkung die Durchtrittsfrequenz $\omega_d$
zu bestimmen, um anschliessend mit deren Hilfe $K_{rk}$ auszurechnen.

Die    Resultate    aus     Gleichung~\ref{eq:pid:beta_result}    werden    in
Gleichung~\ref{eq:pid:target} eingesetzt. $K_{rk}$  ist immer  noch unbekannt,
und wird daher vorerst bei 1 belassen.

\begin{gather} \label{eq:pid:h_rpid_beta_result}
    \begin{split}
        H_{rpid} & = K_{rk} \cdot \biggl[ \frac{(1 + s \cdot T_{nk}               ) \cdot (1 + s \cdot T_{vk}               ) }{ s \cdot T_{nk}               } \biggr]
                   = 1      \cdot \biggl[ \frac{(1 + s \cdot \SI{5.3656}{\second} ) \cdot (1 + s \cdot \SI{0.4134}{\second} ) }{ s \cdot \SI{5.3656}{\second} } \biggr]
    \end{split}
\end{gather}


Es interessiert hier der Phasengang des offenen Regelkreises (auch eingetragen
in   Abbildung~\ref{fig:pid:complete}),    wozu   die   \"Ubertragungsfunktion
der  Strecke   (siehe  Gleichung   \ref{eq:transfer:plant})  mit   der  soeben
bestimmten  provisorischen   \"Ubertragungsfunktion  des   Reglers  (Gleichung
\ref{eq:pid:h_rpid_beta_result}) multipliziert wird.

\begin{equation} \label{eq:pid:h_o_k_rk_one}
    H_{o}(j\omega) = H_{rpid}(j\omega) \cdot H_s(j\omega)
\end{equation}

Nun wird die  Durchtrittsfrequenz $\omega_d$ berechnet, an  welcher der offene
Regelkreis eine  Verst\"arkung von $\SI{0}{\decibel} =  1$ aufweisen soll. Wie
auch  beim  PI-Regler  werden  wir   hier  ein  \"Uberschwingen  von  $16.3\%$
anstreben, womit gem\"ass Tabelle \ref{tab:phi_s} gilt:

\begin{equation} \label{eq:pid:omega_d_target}
    \varphi_s(\omega_d) = \varphi_s = -128.5\degree
\end{equation}

Dieser   Wert   wird   analog   zum   PI-Regler   aus   dem   Phasengang   des
offenen Regelkreises  abgelesen (siehe Abbildung~\ref{fig:pid:complete}). Eine
Nachrechnung mittels Matlab ergibt:

\begin{equation} \label{eq:pid:omega_d_target}
    \omega_d = \SI{0.5341}{\per\second}
\end{equation}


\subsubsection{Bestimmung der Reglerverst\"arkung $K_{rk}$}

Im letzten  Schritt wird  nun der Amplitudengang  des offenen  Regelkreises an
der  Stelle $\omega_d$  gleich 1  gesetzt  und diese  Gleichung nach  $K_{rk}$
aufgel\"ost:

\begin{equation} \label{eq:pid:h_o_k_rk_one}
    \begin{split}
        A_{o}(j\omega_d)    & = | H_{o}(j\omega_d) | = | H_{rpid}(j\omega_d) \cdot H_s(j\omega_d) |    \\
                            & = \Biggl \rvert
                                    K_{rk}
                                    \cdot
                                    \biggl[ \frac{(1 + j\omega_d \cdot T_{nk}) \cdot (1 + j\omega_d \cdot T_{vk}) }{ j\omega_d \cdot T_{nk} } \biggr] \Biggr \rvert \\
                            & \cdot
                                \Biggl \rvert
                                    K_s
                                    \cdot \frac{1}{1 + j\omega_d \cdot T_1}
                                    \cdot \frac{1}{1 + j\omega_d \cdot T_2}
                                    \cdot \frac{1}{1 + j\omega_d \cdot T_2}
                                    \Biggr \rvert \\
                            & = 1
    \end{split}
\end{equation}

Die einzusetzenden Werte sind:
\begin{gather} \label{eq:pid:h_o_k_rk_one}
    \begin{split}
        K_s         & = 2                        \\
        T_1         & = \SI{0.4134}{\second}     \\
        T_2         & = \SI{1.4894}{\second}     \\
        T_3         & = \SI{5.3655}{\second}     \\
        T_{nk}      & = \SI{5.3656}{\second}     \\
        T_{vk}      & = \SI{0.4134}{\second}     \\
        \omega_d    & = \SI{0.5341}{\per\second}
    \end{split}
\end{gather}

Womit man f\"ur die Verst\"arkung den Wert

\begin{equation} \label{eq:pid:k_rk_result}
    K_{rk} = 1.83084
\end{equation}

erh\"alt.

\subsubsection{Resultat}

Somit   ist  der   Regler   vollst\"andig  dimensioniert   und  hat   folgende
\"Ubertragungsfunktion:

\begin{equation} \label{eq:pid:result}
    H_{rpid}(s) = 1.83084 \cdot \biggl[ \frac{(1 + s \cdot \SI{5.3656}{\second} ) \cdot (1 + s \cdot \SI{0.4134}{\second} ) }{ s \cdot \SI{5.3656}{\second} } \biggr]
\end{equation}

Zusammenfassend sind in Abbildung~\ref{fig:pid:complete} die verschiedenen
Frequenzg\"ange und Frequenzen eingetragen.

\begin{figure}[h! width=\pagewidth]
    \includegraphics[width=\textwidth]{images/pidCompletePlot.png}
    \caption{%
        Frequenzgang der Strecke (blau), des  Reglers (gr\"un) und des offenen
        Regelkreises  (rot).  Ebenfalls  eingetragen  sind die  Reglerfrequenz
        $\omega_{pid}$,   die   beiden   Frequenzen   $\frac{1}{T_{vk}}$   und
        $\frac{1}{T_{nk}}$ sowie die Durchtrittsfrequenz $\omega_d$.
    }
    \label{fig:pid:complete}
\end{figure}
\clearpage


\subsection{Umrechnung zwischen bodekonformer und reglerkonformer Darstellung}
\label{subs:bode_regler}
Die    Formeln   in    Tabelle   \ref{tab:bode_regler_konform}    dienen   zur
Umrechnung  zwischen der  bodekonformen  Darstellung  und der  reglerkonformen
Darstellung. N\"ahere  Informationen  zu den  verschiedenen  Darstellungsarten
k\"onnen  der   Quelle  \cite{regelungstechnik:zellweger}   entnommen  werden.
\todo{Allenfalls noch  ein paar kurze  S\"atze zum Sinn  dieser \"Ubung? Sonst
wird nirgends  darauf wirklich  Bezug genommen, Abschnitt  ist ein  wenig ohne
Kontext in der Landschaft.}

\begin{longtable}{l|ll}
    \toprule

    %\multicolumn{3}{l}{\large{\textsc{auftragsanalyse und hintergrundinformationen}}} \\
    %\multicolumn{2}{l}{pi}

    &
    bodekonform $\rightarrow$ reglerkonform
    &
    reglerkonform $\rightarrow$ bodekonform
    \\

    \midrule

    \endhead
    \endfoot
    \endlastfoot

    % content here ---------------------------------------------------------- %

    PI
    &
    $T_n = T_{nk} $ %reglerkonform
    &
    $K_{rk} = K_r $ %bodekonform
    \\

    \midrule

    PID
    &
    $T_n = T_{nk}+T_{vk}-T_p$
    &
    $T_{nk}=0.5 \cdot (T_n+T_p) \cdot (1+\epsilon)$
    \\

    &
    $T_v=\frac{T_{nk} \cdot T_{vk}}{T_{nk}+T_{vk}-T_p}-T_p$
    &
    $T_{vk}=0.5 \cdot (T_n+t_p) \cdot (1-\epsilon)$
    \\

    &
    %$k_r=k_{rk} \cdot \frac{1+t_{vk}}{t_{nk}}$
    $K_r=K_{rk} \cdot (1 + \frac{T_{vk}-T_p}{T_{nk}})$
    &
    $K_{rk} = 0.5 \cdot K_r \cdot (1 + \frac{T_p}{T_{nk}}) \cdot (1+\epsilon )$
    \\
    \\

    &
    \multicolumn{2}{l}{wobei $\epsilon^2 = 1-(4 \cdot T_n \cdot \frac{T_v-T_p}{(T_n+T_p)^2})$}
    \\
    \bottomrule
    \caption{Formeln zur Umrechung zwischen bode- zu reglerkonformer Darstellung \cite{regelungstechnik:zellweger}, \cite{regelungstechnik:schumleon}}
    \label{tab:bode_regler_konform}
\end{longtable}


F\"ur die Berechnungen in diesem Projekt wird, wenn nicht anders angegeben, mit $T_p=\frac{1}{10} \cdot T_v$ gerechnet.

