In  der   Regelungstechnik  ist   die  korrekte  Dimensionierung   der  Regler
grundlegend. Die  zu  regelnde Strecke  kann  nur  durch richtig  eingestellte
Regelwerte wie gew\"unscht ver\"andert werden.

Dieses  Projekt hat  sich zum  Ziel  gesetzt ein  Softwaretool zu  entwickeln,
welches  aus  den  eingegebenen  Streckenwerten $K_s$,  $T_u$  und  $T_g$  PI-
und  PID-Regler  dimensioniert. Das Tool  berechnet  die  Reglerwerte mit  der
sogenannten   Phasengangmethode,  welche   von  Jakob   Zellweger,  ehemaliger
Dozent  der   Fachhochschule  Nordwestschweiz,  stammt. Diese   Methode  wurde
urspr\"unglich f\"ur die graphische  Auswertung des Phasengangs entwickelt und
durch  Handarbeit mit  Geodreieck und  Bleistift angewendet. Das  Softwaretool
erm\"oglicht  die  bew\"ahrte  Methode  effizient  durch  ein  automatisiertes
Verfahren anzuwenden.

Das  entwickelte Softwaretool  simuliert  neben  den berechneten  Reglerwerten
auch  die Schrittantwort  des  geschlossenen Regelkreises. Zus\"atzlich  wurde
die   Berechnung  dreier   g\"angigen   Faustformeln  mit   implementiert. Zum
Vergleich  k\"onnen die  Schrittantworten  der Faustformeln  im gleichen  Plot
angezeigt  werden.   F\"ur  die  Phasengangmethode  werden  gleichzeitig  drei
Graphen  abgebildet. Einer   mit  der   Standart-Reglerknickfrequenz  gem\"ass
Phasengangmethode und zwei weitere welche sich in der Reglerknickfrequenz um ±
10° unterscheiden.

Zus\"atzlich  kann die  Reglerknickfrequenz  und  das \"Uberschwingen  mittels
Schiebregler manuell  ver\"andert werden. Dies erm\"oglicht die  im Vergleich,
zu  der urspr\"unglichen  Handarbeit, enorm  hohe Geschwindigkeit. Somit  wird
eine Echtzeit Optimierung der bestehenden Phasengangmethode get\"atigt.
