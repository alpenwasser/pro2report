% was laeuft, was laeuft nicht, was kann verbessert werden
Das   Projekt   wurde   erfolgreich   abgeschlossen  und   alle   Punkte   des
urspr\"unglichen   Auftrages  erf\"ullt. Aus   den   Eingabewerten  wird   das
dynamische Verhalten  des geschlossenen  Regelkreises berechnet  und graphisch
dargestellt. Die einige Erweiterungen wurden  eingebaut, sodass man den Regler
mit der Phasengangmethode dimensioniert und zus\"atzlich.


Die vor der  Erweiterung des Auftrags optional  geplanten graphischen Ausgaben
des  Amplitudengangs  und  der  Strecke   sowie  die  Schiebregler  f\"ur  das
Ver\"andern  der  Reglerwerte  wurden   weggelassen. Der  Grund  f\"ur  diesen
Entscheid war  die M\"oglichkeit durch  die hohe Rechenleistung den  Regler in
Echtzeit zu optimieren womit diese Funktionen \"uberfl\"ussig wurden.


Alternativ  kann die  Schrittantwort mit  zwei Schiebregler  manuell optimiert
werden. Durch  das Ver\"andern  des  \"Uberschwingens und  das Einstellen  der
Optimierung kann  die Kurve  per der Hand  angepasst werden.   Als Erweiterung
w\"are es m\"oglich  die Streckenwerte auszugegeben und  die Strecke graphisch
zur\"uckgegeben wird. Das  automatische Einlesen  der Streckenwerte  aus einer
graphisch Form k\"onnte hinzugru\“ werden. Das Tool w\"urde an die eingelesene
Kurve die Wendetangente legen und  die Werte ablesen. Das Ablesen k\"onnte mit
der Monte-Carlo-Studie verfeinert werden.
