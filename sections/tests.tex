Damit  die Software  von  Anfang an  den Anforderungen  des  Kunden sowie  dem
Pflichtenheft entspricht, wurden die Berechnungen sowie die Funktionsweise der
einzelnen Komponenten laufend mittels verschiedenster Varianten \"uberpr\"uft,
welche nachfolgend erl\"autert werden.

\subsection{Berechnungen}
Die  Berechnungen und  Algorithmen wurden  vorerst in  Matlab geschrieben  und
durch  ein  weiteres  Teammitglied \"uberpr\"uft. Die  in  Matlab  berechneten
Test-Resultate konnten mit  dem Fachcoach/Auftraggeber abgeglichen werden. Die
so entstandenen Matlab-Files  dienten als direkte Vorlage  f\"ur die Umsetzung
in Java-Code. Zus\"atzlich  konnten die berechneten  Werte in Java  direkt mit
den Resultaten aus Matlab verglichen werden.

\subsection{Intern (Team)}
Um  die  Funktionalit\"at   und  das  Verhalten  des  GUIs   zu  testen  wurde
das  Programm  regelm\"assig  durch   das  Team  getestet. Dabei  wurden  alle
Mitglieder dazu  aufgefordert nebst den  zu erwartenden Werten  auch explizite
Fehleingaben/-ausgaben zu provozieren.

Fehler     wurden     umgehend     an     den     Programmier-Verantwortlichen
weitergeleitet. Sobald  der Fehler  behoben  wurde, wurde  das Team  dar\"uber
informiert und zur erneuten Verifikation aufgefordert.

\subsection{Extern}
Das Tool wurde auf Computer  aussenstehender Personen auf Funktionalit\"at und
korrekte  Darstellungsweise  getestet. Dabei  wurde ein  spezielles  Augenmerk
darauf  gelegt, dass  die  Testpersonen  unterschiedliche Betriebssysteme  und
Bildschirmaufl\"osungen verwenden.


\subsection{Auftraggeber}
Die Software wurde in enger Zusammenarbeit mit dem Auftraggeber entwickelt. So
konnte  er  die  Software  immer   wieder  testen  und  direktes  Feedback  an
die  Entwickler   geben. Eine  m\"oglichst  kundenorientierte   L\"osung,  die
den  Vorstellungen  des  Auftraggebers  entspricht,  konnte  somit  erarbeitet
werden. Eine  Vorabversion  wurde  dem Auftraggeber  zur  \"Uberpr\"ufung  der
Aufl\"osung und Lauff\"ahigkeit auf seinem System zugestellt.
