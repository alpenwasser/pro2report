Das  Kernst\"uck  dieser  Arbeit  und des  zuge\"origen  Softwaretools  stellt
die   so  genannte   ``Phasengang-Methode   zur  Reglerdimensionierung''   von
Jakob   Zellweger~\cite{regelungstechnik:zellweger_short}   dar. Diese   wurde
urspr\"unglich  als  vereinfachte  grafische  Methode  zur  Approximation  der
-20dB/Dek Methode erarbeitet und im Rahmen dieses Projektes in einem Java-Tool
automatisiert.

Genauere  Informationen  zu  den  Hingergr\"unden  der  beiden  aufgef\"uhrten
Rezepte sind dem Skript~\cite{regelungstechnik:zellweger_short} zu entnehmen.

Das \"Uberschwingverhalten des Regelkreises soll  f\"ur dieses Projekt in drei
Stufen  berechnet werden. Verwendet  wird dazu  folgende Abstufung:  \todo{Ist
diese  Einteilung des  \"Uberschwingens noch  aktuell  oder wurde  das in  der
Auftragserweiterung angepasst?}


\begin{itemize}
    \item
        wenig \"Uberschwingen (ca. 0\%)
    \item
        mittleres \"Uberschwingen (ca. 16\%)
    \item
        starkes \"Uberschwingen (ca. 23\%)
\end{itemize}

\subsubsection{Rezept}
Als Erstes sollten folgende Begriffe definiert werden:
\begin{longtable}{lp{60mm}}
    \toprule
    \endhead
    \endfoot
    \endlastfoot

    % CONTENT HERE ---------------------------------------------------------- %

    $H_s(j\omega)                                                                   $ &  \"Ubertragungsfunktion der Regelstrecke \\
    $A_s(j\omega)=|H_s(j\omega)|                                                    $ &  Amplitudengang der Regelstrecke \\
    $\varphi_s(j\omega)=arg(H_s(j\omega))                                           $ &  Phasengang der Regelstrecke \\
    $H_r(j\omega)                                                                   $ &  \"Ubertragungsfunktion des Reglers \\
    $A_r(j\omega)=|H_r(j\omega)|                                                    $ &  Amplitudengang des Reglers \\
    $\varphi_r(j\omega)=arg(H_r(j\omega))                                           $ &  Phasengang des Reglers \\
    $H_o(j\omega)=H_s \cdot H_r(j\omega)                                            $ &  \"Ubertragungsfunktion des nicht geschlossenen Regelkreises \\
    $A_o(j\omega)=|H_o(j\omega)|                                                    $ &  Amplitudengang des nicht geschlossenen Regelkreises \\
    $\varphi_o(j\omega)=arg(H_o(j\omega))=\varphi_s(j\omega)+\varphi_r(j\omega)     $ &  Phasengang des nicht geschlossenen Regelkreises \\
    $H_{rpid}= K_{rk}\Big[ \frac{(1+sT_{nk})(1+sT_{vk})}{sT_{nk}}\Big]              $ & \"Ubertragungsfunktion des PID-Reglers \\
    $H_{rpi} = K_{rk}\Big[ 1 + \frac{1}{sT_{nk}} \Big] $ & \"Ubertragungsfunktion des PI-Reglers \\


    \bottomrule
    %\caption{Caption here}
\end{longtable}

\clearpage
\subsubsection{Rezept PID-Regler}
\todo{Grafische Illustration der Rezepte? W\"urde es der Leserin erleichtern,
sich ein Bild des Prozesses zu machen, speziell da die Phasengangmethode
ja im Kern eine grafische Methode ist.}


\begin{enumerate}
    \item
        Im  Phasengang muss  die  Frequenz  $\omega_{pid}$ gem\"ass  Gleichung
        \ref{eq:phi_s} bestimmt werden \footnotemark[1].
        \begin{equation} \label{eq:phi_s}
            \varphi_s(\omega_{pid}) = -135 \degree.
        \end{equation}
    \item
        Mittels  Ableitung  wird  die  Steigung des  Phasengangs  im  Punkt  $
        \omega_{pid} $ bestimmt.
    \item
        $\beta$ ist so w\"ahlen, dass Gleichung \ref{eq:dphi_o} erf\"ullt ist

        \begin{equation} \label{eq:dphi_o}
            \frac{d\varphi_o}{d\omega_{pid}}= -\frac{1}{2}.
        \end{equation}

        wobei:
        \vspace*{1em}

        \begin{center}
            $\frac{\omega_{pid}}{\beta}=\frac{1}{T_{vk}}$,
            $\omega_{pid} \cdot \beta=\frac{1}{T_{nk}}$,
            $T_p = 0$,
            $K_{rk} = 1$
        \end{center}

        \vspace*{1em}
        {\em{Man Beachte: Falls $\beta$ komplex werden sollte, muss $\beta=1$ gesetzt werden.}}
        \vspace*{1em}
    \item
        Nun   werden    $T_{vk}$,   $T_{nk}$   sowie   $K_{rk}=1$    in   die
        \"Ubertragungsfunktion  $H_r$  eingesetzt. Daraus folgen  $\varphi_o$
        sowie $A_o$.   Je nach  gew\"unschtem \"Uberschwingverhalten  wird der
        entsprechende Wert  f\"ur $\varphi_s$ aus der  Tabelle \ref{tab:phi_s}
        herausgelesen.

        Durch Suchen des Punktes  $\omega_d$ gem\"ass Gleichung \ref{eq:phi_o}
        wird berechnet, an welchem Punkt  f\"ur $A_o$ eine Verst\"arkung von 1
        herrschen muss.

        \begin{equation} \label{eq:phi_o}
            \varphi_o(\omega_d)=\varphi_s.
        \end{equation}

        \begin{longtable}{llll}
            \toprule
            \endhead
            \endfoot
            \endlastfoot

            % CONTENT HERE ---------------------------------------------------------- %

            \"Uberschwingen & 0\%              & 16.3\%           & 23.3\% \\
            $\varphi_s$        & $-103.7 \degree$ & $-128.5 \degree$ & $-135 \degree$ \\

            \bottomrule
            \caption{Werte f\"ur $\varphi_s$}
            \label{tab:phi_s}
        \end{longtable}

    \item
        Durch geeignete Wahl von  $K_{rk}$ mithilfe von Gleichung \ref{eq:A_o}
        eine Verst\"arkung von 1 erzwingen:

        \begin{equation} \label{eq:A_o}
            A_o(\omega_d) \cdot K_{rk} = 1
        \end{equation}

    \item
        Alle  Freiheitsgrade des  PID-Reglers  sind hiermit  bestimmt und  der
        Regler nach der Phasengang-Methode vollst\"andig dimensioniert.
\end{enumerate}

\footnotetext[1]{Der Winkel stellt keinen endg\"ultigen Wert dar. Dieser wurde von Jakob Zellweger fixiert, um eine grafische Evaluation \"uberhaupt zu erm\"oglichen. Durch \"andern dieses Wertes kann je nach Regelstrecke das
Regelverhalten weiter optimiert werden.}


\subsubsection{Rezept PI-Regler}

\begin{enumerate}
    \item
        Im  Phasengang  muss  die Frequenz  $\omega_{pi}$  gem\"ass  Gleichung
        \ref{eq:phi_s_pi} bestimmt werden \footnotemark[1].

        \begin{equation} \label{eq:phi_s_pi}
            \varphi_s(\omega_{pi})=-90 \degree.
        \end{equation}

    \item
        $T_{nk}$ kann dadurch direkt bestimmt werden.
        \begin{equation} \label{eq:Tnk_pi}
            T_{nk}=\frac{1}{\omega_{pi}}.
        \end{equation}

    \item
        Anschliessend    werden    $T_{nk}$     und    $K_{rk}=1$    in    die
        \"Ubertragungsfunktion $H_r$  eingesetzt, was $\varphi_o$  sowie $A_o$
        liefert. Jetzt  muss  je  nach gew\"ahltem  \"Uberschwingverhalten  der
        entsprechende Wert  f\"ur $\varphi_s$ aus der  Tabelle \ref{tab:phi_s}
        herausgelesen werden.

        Durch Suchen des Punktes  $\omega_d$ gem\"ass Gleichung \ref{eq:phi_o}
        wird festgelegt an  welchem Punkt f\"ur $A_o$ eine  Verst\"arkung von 1
        definiert werden muss.

    \item
        $K_{rk}$ wird  so gew\"ahlt, dass mithilfe  von Gleichung \ref{eq:A_o}
        eine Verst\"arkung von 1 erzwungen wird.

    \item
        Somit sind alle Freiheitsgrade des  PI-Reglers bestimmt und der Regler
        nach der Phasengang-Methode komplett dimensioniert.
\end{enumerate}
