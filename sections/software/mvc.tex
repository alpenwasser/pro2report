Die Software wurde nach dem \emph{MVC-Pattern} aufgebaut. Somit wurde auch das
\emph{Obersever/Obeservable}-Prinzip  eingebunden. Das Klassendiagramm  ist in
Anhang~\ref{app:classdiagram} zu finden.

In   der    Klasse   \code{Application}    werden   das    \code{Model},   der
\code{GUIController}   sowie   die   \code{View}   erzeugt   und   miteinander
verkn\"upft. Dem   \code{GUIcontroller}   wird   das  \code{Model}   und   der
\code{View}  der  \code{GUIController}  \"ubergeben. Somit  k\"onnen  Eingaben
auf  der   \code{View}  direkt   an  den   \code{GUIController}  weitergegeben
werden. Dieser wiederum kann direkten Einfluss auf das \code{Model} nehmen.

Die  \code{View}  wird  als   Observer  von  \code{Model}  registriert. Sobald
im  \code{Model}  \code{notifyObservers()}  ausgel\"ost   wird,  wird  in  der
\code{View} die Methode \code{update()}  aufgerufen und das ganze \code{Model}
als Objekt \"ubergeben. Die  \code{View} kann nun die  gew\"unschten Daten aus
dem \code{Model} auslesen und f\"ur die Darstellung verwenden.

Die ganze  Applikation basiert auf  einem JFrame, auf welchem  die \code{View}
platziert wird.
